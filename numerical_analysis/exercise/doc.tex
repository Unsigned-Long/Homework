\documentclass[12pt, onecolumn]{article}

% 引入相关的包
\usepackage{amsmath, listings, fontspec, geometry, graphicx, ctex, color, subfigure, amsfonts, amssymb}
\usepackage{multirow}
\usepackage[table,xcdraw]{xcolor}
\usepackage[ruled]{algorithm2e}
\usepackage[hidelinks]{hyperref}

		\usepackage{graphicx}
		\usepackage[most]{tcolorbox}
\hypersetup{
	colorlinks=true,
	linkcolor=red,
	citecolor=red,
}
\usepackage{booktabs}
\usepackage{multirow}
\usepackage{picins}

% 设定页面的尺寸和比例
\geometry{left = 1.5cm, right = 1.5cm, top = 1.5cm, bottom = 1.5cm}

% 设定两栏之间的间距
\setlength\columnsep{1cm}

% 设定字体,为代码的插入作准备
\newfontfamily\ubuntu{Ubuntu Mono}
\newfontfamily\consolas{Consolas}

% 头部信息
\title{\normf{编程:观测值逐次更新的扩展卡尔曼滤波器}}
\author{\normf 姓名:陈烁龙\;\;\;学号:2023202140019\;\;\;学院:测绘学院}
\date{\normf{\today}}

% 代码块的风格设定
\lstset{
	language=C++,
	basicstyle=\small\ubuntu,
	keywordstyle=\textbf,
	stringstyle=\itshape,
	commentstyle=\itshape,
	numberstyle=\scriptsize\ubuntu,
	showstringspaces=false,
	numbers=left,
	numbersep=8pt,
	tabsize=2,
	frame=single,
	framerule=1pt,
	columns=fullflexible,
	breaklines,
	frame=shadowbox, 
	backgroundcolor=\color[rgb]{0.97,0.97,0.97}
}

% 字体族的定义
% \fangsong \songti \heiti \kaishu
\newcommand\normf{\fangsong}
\newcommand\boldf{\heiti}
\newcommand\keywords[1]{\bfseries{关键词:} \normf #1}

\newcommand\liehat[1]{\left[ #1 \right]_\times}
\newcommand\lievee[1]{\left[ #1 \right]^\vee}
\newcommand\liehatvee[1]{\left[ #1 \right]^\vee_\times}
\newcommand\bsm[1]{\boldsymbol{\mathrm{#1}}}
\newcommand\mlcomment[1]{\iffalse #1 \fi}
%\newcommand\mlcomment[1]{ #1 }

\newcounter{problemname}
\newenvironment{problem}{\stepcounter{problemname}\par\noindent\normf\textbf{\textcolor[rgb]{1,0,0}{题目\arabic{problemname}.} }}{\leavevmode\\\par}
\newenvironment{solution}{\par\noindent\normf\textbf{解答: }}{\leavevmode\\\par}
\newenvironment{note}{\par\noindent\normf\textbf{题目\arabic{problemname}的注记: }}{\leavevmode\\\par}
\begin{document}
	\begin{titlepage}
	    \centering
	    \includegraphics[width=0.4\textwidth]{whu_red.png}\par\vspace{1cm}
	    \vspace{4cm}
	    {\huge\kaishu\bfseries 数值分析习题 \par}
	    \vspace{3cm}
	    {\Large\kaishu 
	    \begin{center}\begin{tabular}{l}
	    姓名:陈烁龙\\
	    学号:\bfseries 2023202140019\\
	    学院:测绘学院
	    \end{tabular}\end{center}
	     \par}
	    
	
	    \vfill
	
	% Bottom of the page
	    {\large\kaishu\bfseries \today\par}
	\end{titlepage}
		% 换页
 		\thispagestyle{empty}
		\clearpage
		
		% 插入目录、图、表并换页
		\pagenumbering{roman}
		\tableofcontents
		\newpage
		\listoffigures
%		\newpage
%		\listoftables
		% 罗马字母形式的页码
		
		\clearpage
		% 从该页开始计数
		\setcounter{page}{1}
		% 阿拉伯数字形式的页码
		\pagenumbering{arabic}
	
	
	\section{\normf{绪论}}
	\normf\bfseries
	\problem{...}
	\solution{
	已知:
	\begin{equation}
		\arctan x\approx x-\frac{1}{3}x^3+\frac{1}{5}x^5
	\end{equation}
	所以
	\begin{equation}
		4\times\left( \arctan \frac{1}{2}+\arctan\frac{1}{3}\right) \approx
		4\times\left( \frac{1}{2}-\frac{1}{3}\times\frac{1}{8}
		+\frac{1}{5}\times\frac{1}{32}
		+\frac{1}{3}-\frac{1}{3}\times\frac{1}{27}+\frac{1}{5}\times\frac{1}{243}
		\right) =3.145576132
	\end{equation}
	而$\pi=3.141592653...$,所以有绝对误差:
	\begin{equation}
	e_{abs}= x^*-x\approx 3.145576132-3.141592653=0.003983479
	\end{equation}
	相对误差为:
	\begin{equation}
	e_{rel}=\frac{x^*-x}{x}\approx\frac{0.003983479}{\pi}\approx 0.001267980=0.1267980\%
	\end{equation}
	}
	
	\problem{...}
	\solution{
	当$\tilde{\bsm{x}}_1=\begin{pmatrix}
	1&0
	\end{pmatrix}^\top$时,有:
	\begin{equation}
	\bsm{b}-\bsm{A}\tilde{\bsm{x}_1}=\begin{pmatrix}
	1\\1
	\end{pmatrix}-\begin{pmatrix}
	1&0.99\\0.99&0.98
	\end{pmatrix}\begin{pmatrix}
		1\\0
		\end{pmatrix}=\begin{pmatrix}
		0\\0.01
		\end{pmatrix}
	\qquad \Vert\bsm{b}-\bsm{A}\tilde{\bsm{x}_1}\Vert_{\infty}=\max\vert x_i\vert=0.01
	\end{equation}
	所以有:
	\begin{equation}
		\frac{\Vert\bsm{b}-\bsm{A}\tilde{\bsm{x}_1}\Vert_{\infty}}{\Vert\bsm{b}\Vert_{\infty}}=\frac{0.01}{1}=0.01
		\qquad
		\frac{\Vert\delta\tilde{\bsm{x}_1}\Vert_{\infty}}{\Vert\bsm{x}^*\Vert_{\infty}}=\frac{100}{100}=1
	\end{equation}
	同理可得:
	\begin{equation}
	\frac{\Vert\bsm{b}-\bsm{A}\tilde{\bsm{x}_2}\Vert_{\infty}}{\Vert\bsm{b}\Vert_{\infty}}=\frac{0.995}{1}=0.995
			\qquad
			\frac{\Vert\delta\tilde{\bsm{x}_2}\Vert_{\infty}}{\Vert\bsm{x}^*\Vert_{\infty}}=\frac{0.5}{100}=0.005
	\end{equation}
	
	另外有:
	\begin{equation}
	\mathrm{cond}_{\infty}(\bsm{A})=\Vert\bsm{A}\Vert_{\infty}\cdot\Vert\bsm{A}^{-1}\Vert_{\infty}=\Vert
	\frac{1}{0.0001}
	\begin{pmatrix}
	0.98&-0.99\\
	-0.99&1
	\end{pmatrix}
	\Vert_{\infty}=10000
	\end{equation}
	}
	空间矩阵$\bsm{A}$的条件数大于1。条件数比较大时,矩阵是病态矩阵,相应的方程组是病态方程组。因为有:
	\begin{equation}
	\begin{aligned}
	\bsm{A}(\bsm{x}+\delta \bsm{x})&=b+\delta \bsm{b}
	\\
	\delta \bsm{x}&=\bsm{A}^{-1}\cdot\delta\bsm{b}
	\\
	\Vert\delta\bsm{x}\Vert_{\infty}&\le\Vert\bsm{A}^{-1}\Vert_{\infty}\cdot\Vert\delta\bsm{b}\Vert_{\infty}
	\\
	since\;\;\Vert\bsm{b}\Vert_{\infty}&=\Vert\bsm{Ax}\Vert_{\infty}\le\Vert\bsm{A}\Vert_{\infty}\cdot\Vert\bsm{x}\Vert_{\infty}
	\\
	\frac{1}{\Vert\bsm{x}\Vert_{\infty}}&\le\frac{\Vert\bsm{A}\Vert_{\infty}}{\Vert\bsm{b}\Vert_{\infty}}
	\\
	thus\;\;
	\frac{\Vert\delta\bsm{x}\Vert_{\infty}}{\Vert\bsm{x}\Vert_{\infty}}&\le
	\Vert\bsm{A}\Vert_{\infty}\cdot\Vert\bsm{A}^{-1}\Vert_{\infty}\cdot\frac{\Vert\delta\bsm{b}\Vert_{\infty}}{\Vert\bsm{b}\Vert_{\infty}}
	\end{aligned}
	\end{equation}
	所以有:
	\begin{equation}
	1\le 10000\times 0.01=100
	\qquad
	0.005\le 10000\times 0.995=9950
	\end{equation}
	
	\section{\normf{直接法解线性方程组}}
	\problem{...}
	\solution{
	列主元消去法:在第k步消元前,在系数矩阵第k列的对角线以下的元素
	中找出绝对值最大的元。交换方程后,再继续消去计算.
	\begin{equation*}
	\begin{pmatrix}
	3&2&1\\
	1&0&1\\
	12&-3&3
	\end{pmatrix}\begin{pmatrix}
	x_1\\x_2\\x_3
	\end{pmatrix}=\begin{pmatrix}
	4\\2\\15
	\end{pmatrix}
	\qquad
		\begin{pmatrix}
		12&-3&3\\
		3&2&1\\
		1&0&1\\
		\end{pmatrix}\begin{pmatrix}
		x_3\\x_2\\x_1
		\end{pmatrix}=\begin{pmatrix}
		15\\4\\2
		\end{pmatrix}
	\end{equation*}
	\begin{equation*}
		\begin{pmatrix}
		1&-\frac{1}{4}&\frac{1}{4}\\
		3&2&1\\
		1&0&1\\
		\end{pmatrix}\begin{pmatrix}
		x_3\\x_2\\x_1
		\end{pmatrix}=\begin{pmatrix}
		\frac{5}{4}\\4\\2
		\end{pmatrix}
		\qquad
		\begin{pmatrix}
		1&-\frac{1}{4}&\frac{1}{4}\\
		0&\frac{5}{4}&\frac{1}{4}\\
		0&\frac{1}{4}&\frac{3}{4}\\
		\end{pmatrix}\begin{pmatrix}
		x_3\\x_2\\x_1
		\end{pmatrix}=\begin{pmatrix}
		\frac{5}{4}\\\frac{1}{4}\\\frac{3}{4}
		\end{pmatrix}
	\end{equation*}
	\begin{equation*}
		\begin{pmatrix}
		1&-\frac{1}{4}&\frac{1}{4}\\
		0&1&\frac{1}{5}\\
		0&0&\frac{7}{10}\\
		\end{pmatrix}\begin{pmatrix}
		x_3\\x_2\\x_1
		\end{pmatrix}=\begin{pmatrix}
		\frac{5}{4}\\\frac{1}{5}\\\frac{7}{10}
		\end{pmatrix}
		\qquad
		\begin{pmatrix}
		1&-\frac{1}{4}&\frac{1}{4}\\
		0&1&\frac{1}{5}\\
		0&0&1\\
		\end{pmatrix}\begin{pmatrix}
		x_3\\x_2\\x_1
		\end{pmatrix}=\begin{pmatrix}
		\frac{5}{4}\\\frac{1}{5}\\1
		\end{pmatrix}
	\end{equation*}
	所以有:
	\begin{equation}
	x_1=1\qquad x_2=\frac{1}{5}-\frac{1}{5}\times x_1=0
	\qquad
	x_3=\frac{5}{4}+\frac{1}{4}\times x_2-\frac{1}{4}\times x_1=1
	\end{equation}
	}
	
	\problem{...}
	\solution{
	}
	
	
	
	
	
	
	
	
	
	
	
	
	
	
	
	
	
	
	
		
	\newpage
	\section*{ACKNOWLEDGMENT}
	\begin{tcolorbox}[colback=white,colframe=white!70!black,title={\bfseries Author Information}]
	\par\noindent
		\parbox[t]{\linewidth}{
	 \noindent\parpic{\includegraphics[height=2in,width=1in,clip,keepaspectratio]{ShuolongChen_grey.jpg}}
	 \noindent{\bfseries Shuolong Chen}\emph{
	 received the B.S. degree in geodesy and geomatics engineering from Wuhan University, Wuhan China, in 2023.
	 He is currently a master candidate at the school of Geodesy and Geomatics, Wuhan University. His area of research currently focuses on integrated navigation systems and multi-sensor fusion.
	 Contact him via e-mail: shlchen@whu.edu.cn.
	 }}
	\end{tcolorbox}
		
		
\end{document}
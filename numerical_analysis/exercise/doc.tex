\documentclass[12pt, onecolumn]{article}

% 引入相关的包
\usepackage{amsmath, listings, fontspec, geometry, graphicx, ctex, color, subfigure, amsfonts, amssymb}
\usepackage{multirow}
\usepackage[table,xcdraw]{xcolor}
\usepackage[ruled]{algorithm2e}
\usepackage[hidelinks]{hyperref}

		\usepackage{graphicx}
		\usepackage[most]{tcolorbox}
\hypersetup{
	colorlinks=true,
	linkcolor=red,
	citecolor=red,
}
\usepackage{booktabs}
\usepackage{multirow}
\usepackage{picins}

% 设定页面的尺寸和比例
\geometry{left = 1.5cm, right = 1.5cm, top = 1.5cm, bottom = 1.5cm}

% 设定两栏之间的间距
\setlength\columnsep{1cm}

% 设定字体,为代码的插入作准备
\newfontfamily\ubuntu{Ubuntu Mono}
\newfontfamily\consolas{Consolas}

% 头部信息
\title{\normf{编程:观测值逐次更新的扩展卡尔曼滤波器}}
\author{\normf 姓名:陈烁龙\;\;\;学号:2023202140019\;\;\;学院:测绘学院}
\date{\normf{\today}}

% 代码块的风格设定
\lstset{
	language=C++,
	basicstyle=\small\ubuntu,
	keywordstyle=\textbf,
	stringstyle=\itshape,
	commentstyle=\itshape,
	numberstyle=\scriptsize\ubuntu,
	showstringspaces=false,
	numbers=left,
	numbersep=8pt,
	tabsize=2,
	frame=single,
	framerule=1pt,
	columns=fullflexible,
	breaklines,
	frame=shadowbox, 
	backgroundcolor=\color[rgb]{0.97,0.97,0.97}
}

% 字体族的定义
% \fangsong \songti \heiti \kaishu
\newcommand\normf{\fangsong}
\newcommand\boldf{\heiti}
\newcommand\keywords[1]{\bfseries{关键词:} \normf #1}

\newcommand\liehat[1]{\left[ #1 \right]_\times}
\newcommand\lievee[1]{\left[ #1 \right]^\vee}
\newcommand\liehatvee[1]{\left[ #1 \right]^\vee_\times}
\newcommand\bsm[1]{\boldsymbol{\mathrm{#1}}}
\newcommand\mlcomment[1]{\iffalse #1 \fi}
%\newcommand\mlcomment[1]{ #1 }

\newcounter{problemname}
\newenvironment{problem}{\stepcounter{problemname}\par\noindent\normf\textbf{\textcolor[rgb]{1,0,0}{题目\arabic{problemname}.} }}{\leavevmode\\\par}
\newenvironment{solution}{\par\noindent\normf\textbf{解答: }}{\leavevmode\\\par}
\newenvironment{note}{\par\noindent\normf\textbf{题目\arabic{problemname}的注记: }}{\leavevmode\\\par}
\begin{document}
	\begin{titlepage}
	    \centering
	    \includegraphics[width=0.4\textwidth]{whu_red.png}\par\vspace{1cm}
	    \vspace{4cm}
	    {\huge\kaishu\bfseries 数值分析习题 \par}
	    \vspace{3cm}
	    {\Large\kaishu 
	    \begin{center}\begin{tabular}{l}
	    姓名:陈烁龙\\
	    学号:\bfseries 2023202140019\\
	    学院:测绘学院
	    \end{tabular}\end{center}
	     \par}
	    
	
	    \vfill
	
	% Bottom of the page
	    {\large\kaishu\bfseries \today\par}
	\end{titlepage}
		% 换页
 		\thispagestyle{empty}
		\clearpage
		
		% 插入目录、图、表并换页
		\pagenumbering{roman}
		\tableofcontents
		\newpage
		\listoffigures
%		\newpage
%		\listoftables
		% 罗马字母形式的页码
		
		\clearpage
		% 从该页开始计数
		\setcounter{page}{1}
		% 阿拉伯数字形式的页码
		\pagenumbering{arabic}
	
	
	\section{\normf{绪论}}
	\normf\bfseries
	\problem{...}
	\solution{
	已知:
	\begin{equation}
		\arctan x\approx x-\frac{1}{3}x^3+\frac{1}{5}x^5
	\end{equation}
	所以
	\begin{equation}
		4\times\left( \arctan \frac{1}{2}+\arctan\frac{1}{3}\right) \approx
		4\times\left( \frac{1}{2}-\frac{1}{3}\times\frac{1}{8}
		+\frac{1}{5}\times\frac{1}{32}
		+\frac{1}{3}-\frac{1}{3}\times\frac{1}{27}+\frac{1}{5}\times\frac{1}{243}
		\right) =3.145576132
	\end{equation}
	而$\pi=3.141592653...$,所以有绝对误差:
	\begin{equation}
	e_{abs}= x^*-x\approx 3.145576132-3.141592653=0.003983479
	\end{equation}
	相对误差为:
	\begin{equation}
	e_{rel}=\frac{x^*-x}{x}\approx\frac{0.003983479}{\pi}\approx 0.001267980=0.1267980\%
	\end{equation}
	}
	
	\problem{...}
	\solution{
	当$\tilde{\bsm{x}}_1=\begin{pmatrix}
	1&0
	\end{pmatrix}^\top$时,有:
	\begin{equation}
	\bsm{b}-\bsm{A}\tilde{\bsm{x}_1}=\begin{pmatrix}
	1\\1
	\end{pmatrix}-\begin{pmatrix}
	1&0.99\\0.99&0.98
	\end{pmatrix}\begin{pmatrix}
		1\\0
		\end{pmatrix}=\begin{pmatrix}
		0\\0.01
		\end{pmatrix}
	\qquad \Vert\bsm{b}-\bsm{A}\tilde{\bsm{x}_1}\Vert_{\infty}=\max\vert x_i\vert=0.01
	\end{equation}
	所以有:
	\begin{equation}
		\frac{\Vert\bsm{b}-\bsm{A}\tilde{\bsm{x}_1}\Vert_{\infty}}{\Vert\bsm{b}\Vert_{\infty}}=\frac{0.01}{1}=0.01
		\qquad
		\frac{\Vert\delta\tilde{\bsm{x}_1}\Vert_{\infty}}{\Vert\bsm{x}^*\Vert_{\infty}}=\frac{100}{100}=1
	\end{equation}
	同理可得:
	\begin{equation}
	\frac{\Vert\bsm{b}-\bsm{A}\tilde{\bsm{x}_2}\Vert_{\infty}}{\Vert\bsm{b}\Vert_{\infty}}=\frac{0.995}{1}=0.995
			\qquad
			\frac{\Vert\delta\tilde{\bsm{x}_2}\Vert_{\infty}}{\Vert\bsm{x}^*\Vert_{\infty}}=\frac{0.5}{100}=0.005
	\end{equation}
	
	另外有:
	\begin{equation}
	\mathrm{cond}_{\infty}(\bsm{A})=\Vert\bsm{A}\Vert_{\infty}\cdot\Vert\bsm{A}^{-1}\Vert_{\infty}=\Vert
	\frac{1}{0.0001}
	\begin{pmatrix}
	0.98&-0.99\\
	-0.99&1
	\end{pmatrix}
	\Vert_{\infty}=10000
	\end{equation}
	}
	空间矩阵$\bsm{A}$的条件数大于1。条件数比较大时,矩阵是病态矩阵,相应的方程组是病态方程组。因为有:
	\begin{equation}
	\begin{aligned}
	\bsm{A}(\bsm{x}+\delta \bsm{x})&=b+\delta \bsm{b}
	\\
	\delta \bsm{x}&=\bsm{A}^{-1}\cdot\delta\bsm{b}
	\\
	\Vert\delta\bsm{x}\Vert_{\infty}&\le\Vert\bsm{A}^{-1}\Vert_{\infty}\cdot\Vert\delta\bsm{b}\Vert_{\infty}
	\\
	since\;\;\Vert\bsm{b}\Vert_{\infty}&=\Vert\bsm{Ax}\Vert_{\infty}\le\Vert\bsm{A}\Vert_{\infty}\cdot\Vert\bsm{x}\Vert_{\infty}
	\\
	\frac{1}{\Vert\bsm{x}\Vert_{\infty}}&\le\frac{\Vert\bsm{A}\Vert_{\infty}}{\Vert\bsm{b}\Vert_{\infty}}
	\\
	thus\;\;
	\frac{\Vert\delta\bsm{x}\Vert_{\infty}}{\Vert\bsm{x}\Vert_{\infty}}&\le
	\Vert\bsm{A}\Vert_{\infty}\cdot\Vert\bsm{A}^{-1}\Vert_{\infty}\cdot\frac{\Vert\delta\bsm{b}\Vert_{\infty}}{\Vert\bsm{b}\Vert_{\infty}}
	\end{aligned}
	\end{equation}
	所以有:
	\begin{equation}
	1\le 10000\times 0.01=100
	\qquad
	0.005\le 10000\times 0.995=9950
	\end{equation}
	
	\section{\normf{直接法解线性方程组}}
	\problem{...}
	\solution{
	列主元消去法:在第k步消元前,在系数矩阵第k列的对角线以下的元素
	中找出绝对值最大的元。交换方程后,再继续消去计算.
	\begin{equation*}
	\begin{pmatrix}
	3&2&1\\
	1&0&1\\
	12&-3&3
	\end{pmatrix}\begin{pmatrix}
	x_1\\x_2\\x_3
	\end{pmatrix}=\begin{pmatrix}
	4\\2\\15
	\end{pmatrix}
	\qquad
		\begin{pmatrix}
		12&-3&3\\
		3&2&1\\
		1&0&1\\
		\end{pmatrix}\begin{pmatrix}
		x_3\\x_2\\x_1
		\end{pmatrix}=\begin{pmatrix}
		15\\4\\2
		\end{pmatrix}
	\end{equation*}
	\begin{equation*}
		\begin{pmatrix}
		1&-\frac{1}{4}&\frac{1}{4}\\
		3&2&1\\
		1&0&1\\
		\end{pmatrix}\begin{pmatrix}
		x_3\\x_2\\x_1
		\end{pmatrix}=\begin{pmatrix}
		\frac{5}{4}\\4\\2
		\end{pmatrix}
		\qquad
		\begin{pmatrix}
		1&-\frac{1}{4}&\frac{1}{4}\\
		0&\frac{5}{4}&\frac{1}{4}\\
		0&\frac{1}{4}&\frac{3}{4}\\
		\end{pmatrix}\begin{pmatrix}
		x_3\\x_2\\x_1
		\end{pmatrix}=\begin{pmatrix}
		\frac{5}{4}\\\frac{1}{4}\\\frac{3}{4}
		\end{pmatrix}
	\end{equation*}
	\begin{equation*}
		\begin{pmatrix}
		1&-\frac{1}{4}&\frac{1}{4}\\
		0&1&\frac{1}{5}\\
		0&0&\frac{7}{10}\\
		\end{pmatrix}\begin{pmatrix}
		x_3\\x_2\\x_1
		\end{pmatrix}=\begin{pmatrix}
		\frac{5}{4}\\\frac{1}{5}\\\frac{7}{10}
		\end{pmatrix}
		\qquad
		\begin{pmatrix}
		1&-\frac{1}{4}&\frac{1}{4}\\
		0&1&\frac{1}{5}\\
		0&0&1\\
		\end{pmatrix}\begin{pmatrix}
		x_3\\x_2\\x_1
		\end{pmatrix}=\begin{pmatrix}
		\frac{5}{4}\\\frac{1}{5}\\1
		\end{pmatrix}
	\end{equation*}
	所以有:
	\begin{equation}
	x_1=1\qquad x_2=\frac{1}{5}-\frac{1}{5}\times x_1=0
	\qquad
	x_3=\frac{5}{4}+\frac{1}{4}\times x_2-\frac{1}{4}\times x_1=1
	\end{equation}
	}
	
	\problem{...}
	\solution{
	对矩阵$\bsm{A}$进行Cholesky分解,得到:
	\begin{equation}
	\begin{aligned}
		\bsm{A}&=\begin{pmatrix}
		1&0&0\\
		0&1&0\\
		0&0&1
		\end{pmatrix}\begin{pmatrix}
		4&-2&-4\\
		-2&17&10\\
		-4&10&9
		\end{pmatrix}=\begin{pmatrix}
			1&0&0\\
			-\frac{1}{2}&1&0\\
			-1&0&1
			\end{pmatrix}\begin{pmatrix}
			4&-2&-4\\
			0&16&8\\
			0&8&5
			\end{pmatrix}
		\\
		&=\begin{pmatrix}
		1&0&0\\
		-\frac{1}{2}&1&0\\
		-1&0&1
		\end{pmatrix}\begin{pmatrix}
				1&0&0\\
				0&1&0\\
				0&\frac{1}{2}&1
				\end{pmatrix}\begin{pmatrix}
		4&-2&-4\\
		0&16&8\\
		0&0&1
		\end{pmatrix}=\begin{pmatrix}
		1&0&0\\
		-\frac{1}{2}&1&0\\
		-1&\frac{1}{2}&1
		\end{pmatrix}\begin{pmatrix}
				4&-2&-4\\
				0&16&8\\
				0&0&1
				\end{pmatrix}
				\\
				&=\begin{pmatrix}
						1&0&0\\
						-\frac{1}{2}&1&0\\
						-1&\frac{1}{2}&1
						\end{pmatrix}\begin{pmatrix}
								4&0&0\\
								0&16&0\\
								0&0&1
						\end{pmatrix}
			\begin{pmatrix}
					1&-\frac{1}{2}&-1\\
					0&1&\frac{1}{2}\\
					0&0&1
					\end{pmatrix}
	\end{aligned}
	\end{equation}
	所以有:
	\begin{equation}
	\begin{pmatrix}
							1&0&0\\
							-\frac{1}{2}&1&0\\
							-1&\frac{1}{2}&1
							\end{pmatrix}\begin{pmatrix}
									4&0&0\\
									0&16&0\\
									0&0&1
							\end{pmatrix}
				\begin{pmatrix}
						1&-\frac{1}{2}&-1\\
						0&1&\frac{1}{2}\\
						0&0&1
						\end{pmatrix}\bsm{x}=\begin{pmatrix}
						10\\3\\-5
						\end{pmatrix}
	\end{equation}
	\begin{equation}
	\begin{pmatrix}
						4&0&0\\
						0&16&0\\
						0&0&1
				\end{pmatrix}
	\begin{pmatrix}
			1&-\frac{1}{2}&-1\\
			0&1&\frac{1}{2}\\
			0&0&1
			\end{pmatrix}\bsm{x}=\begin{pmatrix}
			10\\8\\1
			\end{pmatrix}
			\qquad
			\begin{pmatrix}
						1&-\frac{1}{2}&-1\\
						0&1&\frac{1}{2}\\
						0&0&1
						\end{pmatrix}\bsm{x}=\begin{pmatrix}
						2.5\\0.5\\1
						\end{pmatrix}
	\end{equation}
	解得:
	\begin{equation}
	\bsm{x}=\begin{pmatrix}
	3.5\\0\\1
	\end{pmatrix}
	\end{equation}
	}
	
	\section{\normf{最速下降法和共轭梯度法}}
	\problem{...}
	\solution{
	最速下降法:
		\begin{equation}
		\bsm{x}_0=\begin{pmatrix}
		0&0
		\end{pmatrix}^\top\quad\bsm{r}_0=\bsm{b}-\bsm{Ax}_0=\begin{pmatrix}
		0&-1
		\end{pmatrix}^\top
		\end{equation}
		\begin{equation}
		\alpha_0=\frac{\bsm{r}_0^\top\bsm{r}_0}{\bsm{r}_0^\top\bsm{A}\bsm{r}_0}=\frac{1}{2}\quad\bsm{x}_1=\bsm{x}_0+\alpha_0\bsm{r}_0=\begin{pmatrix}
		0&-0.5
		\end{pmatrix}^\top
		\quad\bsm{r}_1=\bsm{r}_0-\alpha_0\bsm{Ax}_0=\begin{pmatrix}
		1.5&0
		\end{pmatrix}^\top
		\end{equation}
			\begin{equation}
			\alpha_1=\frac{\bsm{r}_1^\top\bsm{r}_1}{\bsm{r}_1^\top\bsm{A}\bsm{r}_1}=\frac{2.25}{13.5}\quad\bsm{x}_2=\bsm{x}_1+\alpha_1\bsm{r}_1=\begin{pmatrix}
			0.25&-0.5
			\end{pmatrix}^\top
			\end{equation}
	共轭梯度法:
			\begin{equation}
			\bsm{x}_0=\begin{pmatrix}
			0&0
			\end{pmatrix}^\top
			\quad\bsm{p}_0=\bsm{r}_0=\bsm{b}-\bsm{Ax}_0=\begin{pmatrix}
			0&-1
			\end{pmatrix}^\top\quad
			\alpha_0=\frac{\bsm{p}_0^\top\bsm{r}_0}{\bsm{p}_0^\top\bsm{A}\bsm{p}_0}=\frac{1}{2}
			\end{equation}
			\begin{equation}
			\bsm{x}_1=\bsm{x}_0+\alpha_0\bsm{p}_0=\begin{pmatrix}
					0&-0.5
					\end{pmatrix}^\top
			\quad\bsm{r}_1=\bsm{b}-\bsm{Ax}_1=\begin{pmatrix}
			1.5&0
			\end{pmatrix}^\top
			\quad
			\bsm{p}_1=\bsm{r}_1+\frac{\bsm{r}_1^\top\bsm{r}_1}{\bsm{r}_0^\top\bsm{r}_0}\bsm{p}_0=\begin{pmatrix}
			1.5&-2.25
			\end{pmatrix}^\top
			\end{equation}
			\begin{equation}
		\alpha_1=\frac{\bsm{p}_1^\top\bsm{r}_1}{\bsm{p}_1^\top\bsm{A}\bsm{p}_1}=
		\frac{2}{3}\quad
			\bsm{x}_2=\bsm{x}_1+\alpha_1\bsm{p}_1=\begin{pmatrix}
					1&-2
					\end{pmatrix}^\top
			\end{equation}
	}
	
	\section{\normf{特征值问题}}
	
	\problem{...}
	\solution{
	用 Jacobi 方法求实对称矩阵:
	\begin{equation}
	\bsm{A}=\begin{pmatrix}
	2&1&1\\
	1&2&1\\
	1&1&2
	\end{pmatrix}
	\end{equation}
	最大非对角元素为$a_{21}=1$,所以$\tan 2\theta=\frac{2a_{21}}{a_{22}-a_{11}}\Rightarrow\theta=\frac{\pi}{4}$,构造的Gives矩阵为:
	\begin{equation}
	\bsm{G}=\begin{pmatrix}
	\frac{\sqrt{2}}{2}&-\frac{\sqrt{2}}{2}&0\\
	\frac{\sqrt{2}}{2}&\frac{\sqrt{2}}{2}&0\\
	0&0&1
	\end{pmatrix}
	\end{equation}
	有:
	\begin{equation}
	\bsm{G}^\top\bsm{A}\bsm{G}=\begin{pmatrix}
	3.00000& 0.00000& 1.41421\\
	0.00000 &1.00000& 0.00000\\
	1.41421 &0.00000& 2.00000
	\end{pmatrix}
	\end{equation}
	}
	
	
	\problem{...}
	\solution{
	用 Houscholder 变换将$\bsm{A}$化为上 Hessenberg 矩阵:
	\begin{equation}
	\bsm{A}=\begin{pmatrix}
	1&2&3&4\\
	4&5&6&7\\
	2&1&5&0\\
	4&2&1&0
	\end{pmatrix}
	\end{equation}
	可知$\bsm{S}_1=(0,4,2,4)^\top$,$c_1=-6$,$\bsm{u}_1=\bsm{S}_1-c_1\bsm{e}_2=(0,10,2,4)^\top$,所以:
	\begin{equation}
	\bsm{H}_1=\begin{pmatrix}
	1&0&0&0\\
	0&1&0&0\\
	0&0&1&0\\
	0&0&0&1
	\end{pmatrix}-\frac{1}{60}\begin{pmatrix}
		0&0&0&0\\
		0&100&20&40\\
		0&20&4&8\\
		0&40&8&16
		\end{pmatrix}=\begin{pmatrix}
		1.00000 & 0.00000 & 0.00000 & 0.00000\\
		 0.00000& -0.66667& -0.33333& -0.66667\\
		 0.00000& -0.33333&  0.93333& -0.13333\\
		 0.00000 &-0.66667 -0.13333 & 0.73333
		\end{pmatrix}
	\end{equation}
	得到:
	\begin{equation}
	\bsm{H}_1\bsm{A}\bsm{H}_1=\begin{pmatrix}
	1.00000 &-5.00000 & 1.60000 & 1.20000\\
	-6.00000 & 8.55556& -3.62222 & 0.75556\\
	 0.00000 & 1.37778 & 3.00889& -1.38222\\
	 0.00000 & 5.75556 &-2.38222& -1.56444
	\end{pmatrix}
	\end{equation}
	}
	
	\problem{...}
	\solution{
	用乘幂法求矩阵按模最大特征值的近似值:
	\begin{equation}
	\bsm{A}=\begin{pmatrix}
	4&-1&1\\-1&3&2\\
	1&-2&3
	\end{pmatrix}
	\quad
	\bsm{x}_1=\bsm{A}\bsm{x}_0=\cdots
	\qquad \bsm{x}_1=\frac{\bsm{x}_1}{\max(\bsm{x}_1)}=\cdots
	\qquad
	\lambda_1=\max(\bsm{x}_k)\quad\bsm{u}_1=\mathrm{norm}(\bsm{x}_k)
	\end{equation}
	最后得到$\lambda_1=6.0$,$\bsm{x}=(0.57735,-0.57735,0.57735)^\top$
	}
	\problem{...}
	\solution{
	用反幂法求矩阵按模最小特征值的近似值:
		\begin{equation}
		\bsm{A}=\begin{pmatrix}
		4&-1&1\\-1&3&2\\
		1&-2&3
		\end{pmatrix}
		\quad
		\bsm{x}_1=\bsm{A}^{-1}\bsm{x}_0=\cdots
		\qquad \bsm{x}_1=\frac{\bsm{x}_1}{\max(\bsm{x}_1)}=\cdots
			\qquad
			\lambda_n=\frac{1}{\max(\bsm{x}_k)}\quad\bsm{u}_n=\mathrm{norm}(\bsm{x}_k)
		\end{equation}
	}
	
	
	
	\section{\normf{数值微分和积分}}
	\problem{...}
	\solution{
	中心差分法
	\begin{equation}
	f^\prime(x)=\frac{1}{2\sqrt{x}}=\frac{\sqrt{2}}{4}
	\approx\frac{f(x+h)-f(x-h)}{2\times h}=
	\frac{\sqrt{2.1}-\sqrt{1.9}}{0.2}=0.353663997
	\end{equation}
	}
	
	\problem{...}
	\solution{
	使用梯形公式计算积分:
	\begin{equation}
	\int_{0}^{1}\left(1-e^{-x} \right)^{\frac{1}{2}}dx
	\approx
	\frac{1-0}{2}\cdot\left(
	\left(1-e^{0} \right)^{\frac{1}{2}}+
	\left(1-e^{-1} \right)^{\frac{1}{2}}
	\right)  =0.397530048
	\end{equation}
	}
	
	\problem{...}
	\solution{
	使用 Simpson 公式计算积分:
	\begin{equation}
	\int_{0}^{1}e^{-x}dx=\left[-e^{-x} \right]_0^1=0.632120558 
	\approx
	\frac{1-0}{6}\cdot\left(
	e^{0}+4\cdot e^{-\frac{0+1}{2}}+e^{-1}
	\right) =0.63233368
	\end{equation}
	误差为:
	\begin{equation}
	R(f)=-\frac{1}{90}h^5f^{(4)}(\xi)=-\frac{e^{-\xi}}{90} 
	\quad\xi\in[0,1]
	\end{equation}
	}

	
	\problem{...}
	\solution{
	Simpson 公式计算积分:
	\begin{equation}
	\begin{aligned}
	\int_{0.4}^{0.6}\frac{1}{1+x}dx&=\left[\ln(1+x) \right]_{0.4}^{0.6}=0.133531392\\
		&\approx
		\frac{0.6-0.4}{6}\cdot\left(
		\frac{1}{1+0.4}+4\cdot \frac{1}{1+\frac{0.4+0.6}{2}}+\frac{1}{1+0.6}
		\right) =0.133531746
	\end{aligned}
	\end{equation}
	}
	
	\problem{...}
	\solution{
	$h=0.1$,有
	\begin{equation}
	\begin{array}{c|ccccccccccc}
    x&0&0.1&0.2&0.3&0.4&0.5&0.6&0.7&0.8&0.9&1.0
    \\\hline
    -x^2&0&-0.01&-0.04&-0.09&-0.16&-0.25&-0.36&-0.49&-0.64&-0.81&-1.0
    \\\hline
    e^{-x^2}&
	\end{array}
	\end{equation}
	\begin{equation}
	\int_{0}^{1}e^{-x^2}dx\approx
	0.05\cdot\left[ f(0)+f(1)+2\cdot\left( f(0.1)+\cdots+f(0.9)\right) \right] 
	\end{equation}
	}
	
	\problem{...}
	\solution{
	使用复化梯形公式计算下列积分:
	\begin{equation}
	\int_{0}^{1}\frac{x}{4+x^2}dx,n=8
	\qquad
	\int_{0}^{1}\frac{(1-e^{-x})^{1/2}}{x}dx,n=10
	\end{equation}
	}
	
	\problem{...}
	\solution{
	用复化 Simpson 公式计算下列积分:
	\begin{equation}
	\int_{1}^{9}\sqrt{x}dx,n=2
	\qquad
	\int_{0}^{\frac{\pi}{6}}\sqrt{-\sin^2\phi}d\phi,n=3
	\end{equation}
	}
	
	\problem{...}
	\solution{
	用复化 Simpson 公式计算积分近似值,并估计误差:
	\begin{equation}
	\int_{1}^{1.8}\sqrt{x}dx,n=5
	\end{equation}
	}
	
	\problem{...}
	\solution{
	$f(x)$的值如下表所示, 分别用复化梯形公式和复化 Simpson 公式计算积分$\int_{1.0}^{1.8}f(x)dx$的近似值:
	\begin{equation}
	\begin{array}{c|ccccccccc}
	x&1.0&1.1&1.2&1.3&1.4&1.5&1.6&1.7&1.8
	\\\hline
	f(x)&3&5&2&1&-3&-2&1&-1&2
	\end{array}
	\end{equation}
	$h=0.1$,复化梯形公式:
	\begin{equation}
	T_8=0.05\cdot\left( 3+2+2\cdot 3\right) =0.55
	\end{equation}
	$h=0.1$,复化 Simpson 公式:
	\begin{equation}
	S_4=\frac{0.1}{3}\cdot\left(3+2+4\cdot 3+2\cdot 0\right) =0.5666666
	\end{equation}
	}
	
	\problem{...}
	\solution{
	使用龙贝格方法计算积分:
	\begin{equation}
	\int_{0}^{1}\frac{4}{1+x^2}dx
	\end{equation}
	易得:
	\begin{equation}
	T_0(0)=0.5\cdot(4+2)=3\qquad T_0(1)=0.5\cdot3+0.5\cdot f(0.5)=3.1\qquad
	T_1(0)=(4\cdot 3.1-3)/3=3.13333
	\end{equation}
	}
	
	\problem{...}
	\solution{
	使用龙贝格方法计算积分:
	\begin{equation}
	\frac{2}{x}\int_{0}^{1}e^{-x}dx
	\end{equation}
	}
	
	\problem{...}
	\solution{
	使用龙贝格方法计算积分:
		\begin{equation}
		\int_{0}^{1}\sqrt{x}dx
		\end{equation}
	}
	\problem{...}
	\solution{
	分别令$f(x)$等于$1,x,x^2$,得到:
	\begin{equation}
	\int_{0}^{2} xdx=2=x_1+x_2
	\qquad
	\int_{0}^{2}x^2dx=\frac{8}{3}=x_1^2+x_2^2
	\end{equation}
	}
	
	\problem{...}
	\solution{
	分别令$f(x)$等于$1,x,x^2,x^3$,得到:
	\begin{equation}
	\int_{-3}^{3} xdx=0=A\cdot (-\alpha)+C\cdot\alpha
	\qquad
	\int_{-3}^{3}x^2dx=18=A\cdot\alpha^2+C\cdot\alpha^2
	\end{equation}
	\begin{equation*}
	\int_{-3}^{3}x^3dx=0=A\cdot\alpha^3+C\cdot\alpha^3
	\end{equation*}
	}
	
	\problem{...}
	\solution{
	用三点高斯-勒让德公式计算下列积分:
	\begin{equation}
	\int_{-1}^{1}e^{-x^2}dx\approx
	\frac{5}{9}\cdot \mathrm{Exp}\left( {-\left( -\sqrt{\frac{3}{5}}\right)^2 }\right) 
	+\frac{8}{9}+\frac{5}{9}\cdot\mathrm{Exp}\left( {-\left( \sqrt{\frac{3}{5}}\right)^2 }\right) =1.498679596
	\end{equation}
	用三点高斯-勒让德公式计算下列积分:
	\begin{equation}
	\int_{0}^{1}\sin x^2dx
	\end{equation}
	首先进行归化$x=\frac{1}{2}(t-1)$:
	\begin{equation}
	\int_{-1}^{1}\frac{1}{2}\cdot\sin \left( \frac{1}{4}\left(t+1 \right) ^2\right) dt
	\end{equation}
	}

	\problem{...}
	\solution{
	用三点高斯-勒让德公式计算:
	\begin{equation}
	\int_{0}^{1}\frac{\sin x}{x}dx \Rightarrow\int_{-1}^{1}\frac{\sin \frac{1}{2}(t-1)}{t-1}dt
	\end{equation}
	}

		
	\newpage
	\section*{ACKNOWLEDGMENT}
	\begin{tcolorbox}[colback=white,colframe=white!70!black,title={\bfseries Author Information}]
	\par\noindent
		\parbox[t]{\linewidth}{
	 \noindent\parpic{\includegraphics[height=2in,width=1in,clip,keepaspectratio]{ShuolongChen_grey.jpg}}
	 \noindent{\bfseries Shuolong Chen}\emph{
	 received the B.S. degree in geodesy and geomatics engineering from Wuhan University, Wuhan China, in 2023.
	 He is currently a master candidate at the school of Geodesy and Geomatics, Wuhan University. His area of research currently focuses on integrated navigation systems and multi-sensor fusion.
	 Contact him via e-mail: shlchen@whu.edu.cn.
	 }}
	\end{tcolorbox}
		
		
\end{document}
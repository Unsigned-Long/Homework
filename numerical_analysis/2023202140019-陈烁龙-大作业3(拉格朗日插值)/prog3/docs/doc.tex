\documentclass[12pt, onecolumn]{article}

% 引入相关的包
\usepackage{amsmath, listings, fontspec, geometry, graphicx, ctex, color, subfigure, amsfonts, amssymb}
\usepackage{multirow}
\usepackage[table,xcdraw]{xcolor}
\usepackage[ruled]{algorithm2e}
\usepackage[hidelinks]{hyperref}

		\usepackage{graphicx}
		\usepackage[most]{tcolorbox}
\hypersetup{
	colorlinks=true,
	linkcolor=red,
	citecolor=red,
}
\usepackage{booktabs}
\usepackage{multirow}
\usepackage{picins}

% 设定页面的尺寸和比例
\geometry{left = 1.5cm, right = 1.5cm, top = 1.5cm, bottom = 1.5cm}

% 设定两栏之间的间距
\setlength\columnsep{1cm}

% 设定字体,为代码的插入作准备
\newfontfamily\ubuntu{Ubuntu Mono}
\newfontfamily\consolas{Consolas}

% 代码块的风格设定
\lstset{
	language=C++,
	basicstyle=\small\ubuntu,
	keywordstyle=\textbf,
	stringstyle=\itshape,
	commentstyle=\itshape,
	numberstyle=\scriptsize\ubuntu,
	showstringspaces=false,
	numbers=left,
	numbersep=8pt,
	tabsize=2,
	frame=single,
	framerule=1pt,
	columns=fullflexible,
	breaklines,
	frame=shadowbox, 
	backgroundcolor=\color[rgb]{0.97,0.97,0.97}
}

% 字体族的定义
% \fangsong \songti \heiti \kaishu
\newcommand\normf{\fangsong}
\newcommand\boldf{\heiti}
\newcommand\keywords[1]{\boldf{关键词:} \normf #1}

\newcommand\liehat[1]{\left[ #1 \right]_\times}
\newcommand\lievee[1]{\left[ #1 \right]^\vee}
\newcommand\liehatvee[1]{\left[ #1 \right]^\vee_\times}

\newcommand\mlcomment[1]{\iffalse #1 \fi}
%\newcommand\mlcomment[1]{ #1 }

\newcounter{problemname}
\newenvironment{problem}{\stepcounter{problemname}\par\noindent\normf\textbf{\textcolor[rgb]{1,0,0}{题目\arabic{problemname}.} }}{\leavevmode\\\par}
\newenvironment{solution}{\par\noindent\normf\textbf{解答: }}{\leavevmode\\\par}
\newenvironment{note}{\par\noindent\normf\textbf{题目\arabic{problemname}的注记: }}{\leavevmode\\\par}


\begin{document}
	\begin{titlepage}
	    \centering
	    \includegraphics[width=0.4\textwidth]{whu_red.png}\par\vspace{1cm}
	    \vspace{4cm}
	    {\huge\kaishu “数值分析”编程作业\par 多项式插值 \par}
	    \vspace{3cm}
	    {\Large\kaishu 
	    \begin{center}\begin{tabular}{l}
	    姓名:陈烁龙\\
	    学号:\bfseries 2023202140019\\
	    学院:测绘学院
	    \end{tabular}\end{center}
	     \par}
	    
	
	    \vfill
	
	% Bottom of the page
	    {\large\kaishu\bfseries \today\par}
	\end{titlepage}
		% 换页
 		\thispagestyle{empty}
		\clearpage
		
		% 插入目录、图、表并换页
		\pagenumbering{roman}
		\tableofcontents
		\newpage
		\listoffigures
		\newpage
		\listoftables
		% 罗马字母形式的页码
		
		\clearpage
		% 从该页开始计数
		\setcounter{page}{1}
		% 阿拉伯数字形式的页码
		\pagenumbering{arabic}
	
	\section{\normf{代码运行环境说明}}
	
		\normf\bfseries
		表\ref{tab:代码运行环境}显示了本次编程测试的运行环境。本次代码使用\emph{CPP 17}实现,使用到了线性代数库\emph{Eigen}。代码虽然在\emph{Ubuntu}上实现, 但由于是\emph{Cmake}架构的项目,所以具有跨平台的特征,即可以在\emph{Windows}平台和\emph{Mac}上编译运行。
		
		绘图使用\emph{Python}中的\emph{matploylib}库实现,具体的绘图脚本在\emph{script}文件夹下。
		
		% Please add the following required packages to your document preamble:
		% \usepackage{multirow}
		\begin{table}[h]
		\centering
		\caption{\normf 代码运行环境}
		\label{tab:代码运行环境}
		\vspace{2mm}
		\begin{tabular}{ccc|cl}
		\toprule
		\multicolumn{1}{c|}{\multirow{2}{*}{Config.}} & OS name & Ubuntu 20.04.6 LTS & Processor              & 12th Gen Intel® Core™ i9-12900H × 20              \\ \cmidrule{2-5} 
		\multicolumn{1}{c|}{}                         & OS Type & 64-bit             & Graphics               & Mesa Intel® Graphics (ADL GT2)                    \\ \midrule[1pt]\midrule[1pt]
		\multicolumn{3}{c|}{Library}                                                 & \multicolumn{2}{c}{Link}                                                   \\ \midrule
		\multicolumn{3}{c|}{Eigen3}                                                  & \multicolumn{2}{c}{https://eigen.tuxfamily.org/index.php?title=Main\_Page}              \\ \bottomrule
		\end{tabular}
		\end{table}
		
		本次的作业已上传到(\url{https://github.com/Unsigned-Long/Homework/tree/master/numerical_analysis/prog3})。
		
	\section{\normf{拉格朗日重心插值}}
	对于某个待插值函数$y=f(x)$,给定一组观测数据$\{(x_i,y_i)\mid i = 0,1,\cdots,n\}$,则基于拉格朗日多项式插值方法构造的$n$次插值多项式为:
	\begin{equation}
	\large
	\mathcal{L}_n(x_i)=\sum_{i=0}^{n}\left( 
	y_i\cdot l_i(x)
	\right) =\sum_{i=0}^{n}\left( 
		y_i\cdot\prod_{k=0,k\ne i}^{n}\frac{x-x_k}{x_i-x_k}
		\right) 
	\end{equation}
	此即为经典的Lagrange插值算法。
	
	为对其进行改进、增强,有效克服一般插值方法的
	振荡性和向前性,Jean-Paul Berrut 与 Lloyd N.	Trefethen共同提出了重心坐标拉
	格郎日插值(Barycentric Lagrange Interpolation)。该方法具有计算精度高、不用划分网格、程序易实现等诸多优点,且当插值点较大时仍具有良好的数值稳定性。具体来说,令:
	\begin{equation}
	\large
	\omega_{n+1}(x)=\prod_{k=0}^{n}(x-x_k)
	\qquad
	\omega_i=\prod_{k=0,k\ne i}^{n}\frac{1}{x_i-x_k}
	\end{equation}
	则有:
	\begin{equation}
	\large
	\mathcal{L}_n(x_i)=\sum_{i=0}^{n}\left( 
		y_i
		\cdot\frac{\omega_{n+1}(x)}{x-x_i}
		\cdot\prod_{k=0,k\ne i}^{n}\frac{1}{x_i-x_k}
		\right) =\omega_{n+1}(x)\cdot\sum_{i=0}^{n}\left( 
				\frac{\omega_i}{x-x_i}\cdot y_i
				\right)
	\end{equation}
	此即为第一类重心公式。注意到对于这种格式,当给定观测数据后,$\omega_i$被唯一确定。
	
	另外,由于拉格朗日基函数和为1,有:
	\begin{equation}
	\large
	1=\sum_{i=0}^{n}l_i(x)=\omega_{n+1}(x)\cdot\sum_{i=0}^{n}\left( 
					\frac{\omega_i}{x-x_i}
					\right)
					\Rightarrow
	\omega_{n+1}(x)=\frac{1}{\begin{aligned}
	\sum_{i=0}^{n}\left( 
						\frac{\omega_i}{x-x_i}
						\right)
	\end{aligned}}
	\end{equation}
	所以有:
	\begin{equation}
	\large
	\mathcal{L}_n(x_i)=\frac{\begin{aligned}
	\sum_{i=0}^{n}\left( 
					\frac{\omega_i}{x-x_i}\cdot y_i
					\right)
	\end{aligned}}{\begin{aligned}
		\sum_{i=0}^{n}\left( 
							\frac{\omega_i}{x-x_i}
							\right)
		\end{aligned}}
	\end{equation}
	此即为第二类重心公式。相较于第一类重心公式,更加常用。其不仅有对称性,还有数值稳定性(当$x\to x_i$时,分子分母同时很大,抵消了不稳定性)。
	
	\section{\normf{Chebyshev插值点选取}}
	在区间$[-1,1]$上,选取$n$个插值点,使得:
	\begin{equation}
	\large
	\max\left| (x-x_1)\times\cdots\times(x-x_n)\right| 
	\end{equation}
	尽可能小,则有:
	\begin{equation}
	\large
	x_i=\cos \frac{\left( 2i-1\right)\pi }{2n},i=1,\cdots,n
	\end{equation}
	插值点集$\{x_i\mid i=1,\cdots,n\}$即为Chebyshev插值点。当在区间$[a,b]$上进行Chebyshev插值点计算时,只需要进行区间线性缩放即可:
	\begin{equation}
	\large
	x_i=\frac{a+b}{2}+\frac{b-a}{2}\times\cos \frac{\left( 2i-1\right)\pi }{2n},i=1,\cdots,n
	\end{equation}
	
	\section{\normf{插值测试}}
	\subsection{\normf{Rongo函数}}
	\begin{equation}
	\large
	f(x)=\frac{1}{1+x^2}\quad x\in[-5,5]
	\end{equation}
	分别对该函数选取了30、50、70、100的等距节点和Chebyshev节点进行拉格朗日重心插值,结果如图\ref{fig:Rongo函数的重心拉格朗日插值结果}所示。
	\begin{figure}[h!]
	\centering
	\subfigure[\normf{30个插值节点的插值结果}]{
		\centering
		\includegraphics[width=0.8\linewidth]{../data/rongo/results30.pdf}
	}
	\subfigure[\normf{50个插值节点的插值结果}]{
		\centering
		\includegraphics[width=0.8\linewidth]{../data/rongo/results50.pdf}
	}
	\subfigure[\normf{70个插值节点的插值结果}]{
		\centering
		\includegraphics[width=0.8\linewidth]{../data/rongo/results70.pdf}
	}
	\subfigure[\normf{100个插值节点的插值结果}]{
		\centering
		\includegraphics[width=0.8\linewidth]{../data/rongo/results100.pdf}
	}
	\caption{\normf Rongo函数的重心拉格朗日插值结果}
	\label{fig:Rongo函数的重心拉格朗日插值结果}
	\end{figure}
	
	\subsection{\normf{三角函数}}
	\begin{equation}
	\large
	f(x)=\sin x\quad x\in[-5,5]
	\end{equation}
	分别对该函数选取了30、50、70、100的等距节点和Chebyshev节点进行拉格朗日重心插值,结果如图\ref{fig:三角函数的重心拉格朗日插值结果}所示。
	\begin{figure}[h!]
	\centering
	\subfigure[\normf{30个插值节点的插值结果}]{
		\centering
		\includegraphics[width=0.8\linewidth]{../data/sinx/results30.pdf}
	}
	\subfigure[\normf{50个插值节点的插值结果}]{
		\centering
		\includegraphics[width=0.8\linewidth]{../data/sinx/results50.pdf}
	}
	\subfigure[\normf{70个插值节点的插值结果}]{
		\centering
		\includegraphics[width=0.8\linewidth]{../data/sinx/results70.pdf}
	}
	\subfigure[\normf{100个插值节点的插值结果}]{
		\centering
		\includegraphics[width=0.8\linewidth]{../data/sinx/results100.pdf}
	}
	\caption{\normf 三角函数的重心拉格朗日插值结果}
	\label{fig:三角函数的重心拉格朗日插值结果}
	\end{figure}
	
	\subsection{\normf{指数函数}}
	\begin{equation}
	\large
	f(x)=e^x\quad x\in[-5,5]
	\end{equation}
	分别对该函数选取了30、50、70、100的等距节点和Chebyshev节点进行拉格朗日重心插值,结果如图\ref{fig:指数函数的重心拉格朗日插值结果}所示。
	\begin{figure}[h!]
	\centering
	\subfigure[\normf{30个插值节点的插值结果}]{
		\centering
		\includegraphics[width=0.8\linewidth]{../data/ex/results30.pdf}
	}
	\subfigure[\normf{50个插值节点的插值结果}]{
		\centering
		\includegraphics[width=0.8\linewidth]{../data/ex/results50.pdf}
	}
	\subfigure[\normf{70个插值节点的插值结果}]{
		\centering
		\includegraphics[width=0.8\linewidth]{../data/ex/results70.pdf}
	}
	\subfigure[\normf{100个插值节点的插值结果}]{
		\centering
		\includegraphics[width=0.8\linewidth]{../data/ex/results100.pdf}
	}
	\caption{\normf 指数函数的重心拉格朗日插值结果}
	\label{fig:指数函数的重心拉格朗日插值结果}
	\end{figure}
	
	\subsection{\normf{对数函数}}
	\begin{equation}
	\large
	f(x)=\ln\left( 1+x^2\right) \quad x\in[-5,5]
	\end{equation}
	分别对该函数选取了30、50、70、100的等距节点和Chebyshev节点进行拉格朗日重心插值,结果如图\ref{fig:对数函数的重心拉格朗日插值结果}所示。
	\begin{figure}[h!]
	\centering
	\subfigure[\normf{30个插值节点的插值结果}]{
		\centering
		\includegraphics[width=0.8\linewidth]{../data/ln/results30.pdf}
	}
	\subfigure[\normf{50个插值节点的插值结果}]{
		\centering
		\includegraphics[width=0.8\linewidth]{../data/ln/results50.pdf}
	}
	\subfigure[\normf{70个插值节点的插值结果}]{
		\centering
		\includegraphics[width=0.8\linewidth]{../data/ln/results70.pdf}
	}
	\subfigure[\normf{100个插值节点的插值结果}]{
		\centering
		\includegraphics[width=0.8\linewidth]{../data/ln/results100.pdf}
	}
	\caption{\normf 对数函数的重心拉格朗日插值结果}
	\label{fig:对数函数的重心拉格朗日插值结果}
	\end{figure}
	
	\subsection{\normf{绝对值函数}}
	\begin{equation}
	\large
	f(x)=\left| x\right| \quad x\in[-5,5]
	\end{equation}
	分别对该函数选取了30、50、70、100的等距节点和Chebyshev节点进行拉格朗日重心插值,结果如图\ref{fig:绝对值函数的重心拉格朗日插值结果}所示。
	\begin{figure}[h!]
	\centering
	\subfigure[\normf{30个插值节点的插值结果}]{
		\centering
		\includegraphics[width=0.8\linewidth]{../data/abs/results30.pdf}
	}
	\subfigure[\normf{50个插值节点的插值结果}]{
		\centering
		\includegraphics[width=0.8\linewidth]{../data/abs/results50.pdf}
	}
	\subfigure[\normf{70个插值节点的插值结果}]{
		\centering
		\includegraphics[width=0.8\linewidth]{../data/abs/results70.pdf}
	}
	\subfigure[\normf{100个插值节点的插值结果}]{
		\centering
		\includegraphics[width=0.8\linewidth]{../data/abs/results100.pdf}
	}
	\caption{\normf 绝对值函数的重心拉格朗日插值结果}
	\label{fig:绝对值函数的重心拉格朗日插值结果}
	\end{figure}
	
	\section{\normf{结果分析}}
	从5个函数的插值结果可以看出,在进行拉格朗日插值时,等距节点和Chebyshev节点的区别在于,等距节点是在插值区间上等距离地选取插值点,而Chebyshev节点是在插值区间上选取使得插值多项式的最大误差最小的插值点。等距节点的优点是容易计算,但是在插值区间的两端,多项式插值的误差会很大,即所谓的Runge现象。该现象表明,在
	大范围内使用高次
	插值,逼近的效果往
	往是不理想的。而Chebyshev节点的优点是可以避免Runge现象,但是计算比较复杂。因此,在实际应用中,需要根据具体问题的特点来选择合适的插值节点。
	
	\section{\normf{附录}}
	\begin{lstlisting}[caption=\normf LagrangeInterpolation类实现]
    ns_na::LagrangeInterpolation::LagrangeInterpolation(const std::vector<Observation> &observations)
            : obsVec(observations), omegas(observations.size()) {
        for (int i = 0; i < static_cast<int>(obsVec.size()); ++i) {
            double omegaInv = 1.0;
            for (int k = 0; k < static_cast<int>(obsVec.size()); ++k) {
                if (k == i) { continue; }
                omegaInv *= obsVec.at(i).x - obsVec.at(k).x;
            }
            omegas.at(i) = 1.0 / omegaInv;
        }
        // LOG_VAR(omegas)
    }

    Observation LagrangeInterpolation::operator()(double x) const {
        for (const auto &obv: obsVec) { if (std::abs(x - obv.x) < 1E-10) { return obv; }}
        double v1 = 0.0, v2 = 0.0;
        for (int i = 0; i < static_cast<int>(obsVec.size()); ++i) {
            double v3 = omegas.at(i) / (x - obsVec.at(i).x);
            v1 += v3 * obsVec.at(i).y;
            v2 += v3;
        }
        return {x, v1 / v2};
    }

    Observations LagrangeInterpolation::operator()(const std::vector<double> &xVec) const {
        Observations resVec(xVec.size());
        for (int i = 0; i < static_cast<int>(xVec.size()); ++i) {
            resVec.at(i) = this->operator()(xVec.at(i));
        }
        return resVec;
    }
	\end{lstlisting}
	\begin{lstlisting}[caption=\normf Chebyshev节点和等距节点生成函数]
    std::vector<double> ChebyshevPoints(double a, double b, int n) {
        double v1 = (a + b) * 0.5, v2 = (b - a) * 0.5;
        std::vector<double> points(n);
        for (int i = 1; i <= n; ++i) {
            points.at(i - 1) = v1 + v2 * std::cos((2.0 * i - 1) * M_PI / (2.0 * n));
        }
        return points;
    }

    std::vector<double> AveragePoints(double a, double b, int n) {
        std::vector<double> points(n);
        double delta = (b - a) / (n - 1);
        double v = a;
        for (int i = 0; i < n; ++i) {
            points.at(i) = v;
            v += delta;
        }
        return points;
    }
	\end{lstlisting}
	
	
	\newpage
	\section*{ACKNOWLEDGMENT}
	\begin{tcolorbox}[colback=white,colframe=white!70!black,title={\bfseries Author Information}]
	\par\noindent
		\parbox[t]{\linewidth}{
	 \noindent\parpic{\includegraphics[height=2in,width=1in,clip,keepaspectratio]{ShuolongChen_grey.jpg}}
	 \noindent{\bfseries Shuolong Chen}\emph{
	 received the B.S. degree in geodesy and geomatics engineering from Wuhan University, Wuhan China, in 2023.
	 He is currently a master candidate at the school of Geodesy and Geomatics, Wuhan University. His area of research currently focuses on integrated navigation systems and multi-sensor fusion.
	 Contact him via e-mail: shlchen@whu.edu.cn.
	 }}
	\end{tcolorbox}
		
		
\end{document}


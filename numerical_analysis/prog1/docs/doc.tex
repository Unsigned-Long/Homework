\documentclass[12pt, onecolumn]{article}

% 引入相关的包
\usepackage{amsmath, listings, fontspec, geometry, graphicx, ctex, color, subfigure, amsfonts, amssymb}
\usepackage{multirow}
\usepackage[table,xcdraw]{xcolor}
\usepackage[ruled]{algorithm2e}
\usepackage[hidelinks]{hyperref}

		\usepackage{graphicx}
		\usepackage[most]{tcolorbox}
\hypersetup{
	colorlinks=true,
	linkcolor=red,
	citecolor=red,
}
\usepackage{booktabs}
\usepackage{multirow}
\usepackage{picins}

% 设定页面的尺寸和比例
\geometry{left = 1.5cm, right = 1.5cm, top = 1.5cm, bottom = 1.5cm}

% 设定两栏之间的间距
\setlength\columnsep{1cm}

% 设定字体,为代码的插入作准备
\newfontfamily\ubuntu{Ubuntu Mono}
\newfontfamily\consolas{Consolas}

% 代码块的风格设定
\lstset{
	language=C++,
	basicstyle=\small\ubuntu,
	keywordstyle=\textbf,
	stringstyle=\itshape,
	commentstyle=\itshape,
	numberstyle=\scriptsize\ubuntu,
	showstringspaces=false,
	numbers=left,
	numbersep=8pt,
	tabsize=2,
	frame=single,
	framerule=1pt,
	columns=fullflexible,
	breaklines,
	frame=shadowbox, 
	backgroundcolor=\color[rgb]{0.97,0.97,0.97}
}

% 字体族的定义
% \fangsong \songti \heiti \kaishu
\newcommand\normf{\fangsong}
\newcommand\boldf{\heiti}
\newcommand\keywords[1]{\boldf{关键词:} \normf #1}

\newcommand\liehat[1]{\left[ #1 \right]_\times}
\newcommand\lievee[1]{\left[ #1 \right]^\vee}
\newcommand\liehatvee[1]{\left[ #1 \right]^\vee_\times}

\newcommand\mlcomment[1]{\iffalse #1 \fi}
%\newcommand\mlcomment[1]{ #1 }

\newcounter{problemname}
\newenvironment{problem}{\stepcounter{problemname}\par\noindent\normf\textbf{\textcolor[rgb]{1,0,0}{题目\arabic{problemname}.} }}{\leavevmode\\\par}
\newenvironment{solution}{\par\noindent\normf\textbf{解答: }}{\leavevmode\\\par}
\newenvironment{note}{\par\noindent\normf\textbf{题目\arabic{problemname}的注记: }}{\leavevmode\\\par}


\begin{document}
	\begin{titlepage}
	    \centering
	    \includegraphics[width=0.4\textwidth]{whu_red.png}\par\vspace{1cm}
	    \vspace{4cm}
	    {\huge\kaishu “数值分析”编程作业\par 线性方程组$\boldsymbol{Ax}=\boldsymbol{b}$的求解 \par}
	    \vspace{3cm}
	    {\Large\kaishu 
	    \begin{center}\begin{tabular}{l}
	    姓名:陈烁龙\\
	    学号:\bfseries 2023202140019\\
	    学院:测绘学院
	    \end{tabular}\end{center}
	     \par}
	    
	
	    \vfill
	
	% Bottom of the page
	    {\large\kaishu\bfseries \today\par}
	\end{titlepage}
		% 换页
 		\thispagestyle{empty}
		\clearpage
		
		% 插入目录、图、表并换页
		\pagenumbering{roman}
		\tableofcontents
		\newpage
		\listoffigures
		\newpage
		\listoftables
		% 罗马字母形式的页码
		
		\clearpage
		% 从该页开始计数
		\setcounter{page}{1}
		% 阿拉伯数字形式的页码
		\pagenumbering{arabic}
	
	\section{\normf{LU分解}}
	\normf
	\subsection{\normf{算法描述}}
	对于线性方程组$\boldsymbol{Ax=b}$中的系数方阵$\boldsymbol{A}$,可以通过将其进行LU分解,得到:
	\begin{equation}
	\boldsymbol{A}=\boldsymbol{LU}
	\end{equation}
	而后在方程求解的过程中,使用杜利特尔(Doolittle)分解求解线性方程组:
	\begin{equation}
	\boldsymbol{Ax}=\boldsymbol{b}\quad\Rightarrow\quad
	\boldsymbol{LUx}=\boldsymbol{b}\quad\Rightarrow\quad
	\boldsymbol{Ly}=\boldsymbol{b}\quad\Rightarrow\quad
	\boldsymbol{Ux}=\boldsymbol{y}\quad\Rightarrow\quad
	\boldsymbol{x}
	\end{equation}
	
	关于矩阵$\boldsymbol{A}$的LU分解方面,在进行高斯消去的过程中,初等行变换过程相当于左乘初等变换矩阵。对于第$k$次变换,相当于左乘:
	\begin{equation}
	\boldsymbol{L}_k=\begin{pmatrix}
	1\\
	0&1\\
	0&-l_{k+1,k}&1\\
	\cdots\\
	0&-l_{n,k}&0&\cdots&1
	\end{pmatrix}
	\quad
	\boldsymbol{L}_k^{-1}=\begin{pmatrix}
		1\\
		0&1\\
		0&l_{k+1,k}&1\\
		\cdots\\
		0&l_{n,k}&0&\cdots&1
		\end{pmatrix}
	\end{equation}
	其中$l_{i,k}=a^{(k)}_{i,k}/a^{(k)}_{k,k},i\in[k+1,n]$。于是,有:
	\begin{equation}
	\boldsymbol{A}^{(1)}=\boldsymbol{L}_1^{-1}\cdot\boldsymbol{L}_2^{-1}\cdots
	\boldsymbol{L}_n^{-1}=\boldsymbol{L}\boldsymbol{A}^{n}=\boldsymbol{LU}
	\end{equation}
	
	当然,在高斯消去法消去过程中可能出现$a_{k,k}^{k}=0$的情况,这时高斯消去法将无法进行。这时,可以在第k步消元前,在系数矩阵第k列的对角线以下的元素中找出绝对值最大的元,再继续消去计算,也即使用列主元Gauss消去法。
	
	
	\subsection{\normf{实验结果}}
	\subsection{\normf{关键代码}}
	
	\newpage
	\section{ACKNOWLEDGMENT}
	\begin{tcolorbox}[colback=white,colframe=white!70!black,title={\bfseries Author Information}]
	\par\noindent
		\parbox[t]{\linewidth}{
	 \noindent\parpic{\includegraphics[height=2in,width=1in,clip,keepaspectratio]{ShuolongChen_grey.jpg}}
	 \noindent{\bfseries Shuolong Chen}\emph{
	 received the B.S. degree in geodesy and geomatics engineering from Wuhan University, Wuhan China, in 2023.
	 He is currently a master candidate at the school of Geodesy and Geomatics, Wuhan University. His area of research currently focuses on integrated navigation systems and multi-sensor fusion.
	 Contact him via e-mail: shlchen@whu.edu.cn.
	 }}
	\end{tcolorbox}
		
		
\end{document}


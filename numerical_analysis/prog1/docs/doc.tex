\documentclass[12pt, onecolumn]{article}

% 引入相关的包
\usepackage{amsmath, listings, fontspec, geometry, graphicx, ctex, color, subfigure, amsfonts, amssymb}
\usepackage{multirow}
\usepackage[table,xcdraw]{xcolor}
\usepackage[ruled]{algorithm2e}
\usepackage[hidelinks]{hyperref}

		\usepackage{graphicx}
		\usepackage[most]{tcolorbox}
\hypersetup{
	colorlinks=true,
	linkcolor=red,
	citecolor=red,
}
\usepackage{booktabs}
\usepackage{multirow}
\usepackage{picins}

% 设定页面的尺寸和比例
\geometry{left = 1.5cm, right = 1.5cm, top = 1.5cm, bottom = 1.5cm}

% 设定两栏之间的间距
\setlength\columnsep{1cm}

% 设定字体,为代码的插入作准备
\newfontfamily\ubuntu{Ubuntu Mono}
\newfontfamily\consolas{Consolas}

% 代码块的风格设定
\lstset{
	language=C++,
	basicstyle=\small\ubuntu,
	keywordstyle=\textbf,
	stringstyle=\itshape,
	commentstyle=\itshape,
	numberstyle=\scriptsize\ubuntu,
	showstringspaces=false,
	numbers=left,
	numbersep=8pt,
	tabsize=2,
	frame=single,
	framerule=1pt,
	columns=fullflexible,
	breaklines,
	frame=shadowbox, 
	backgroundcolor=\color[rgb]{0.97,0.97,0.97}
}

% 字体族的定义
% \fangsong \songti \heiti \kaishu
\newcommand\normf{\fangsong}
\newcommand\boldf{\heiti}
\newcommand\keywords[1]{\boldf{关键词:} \normf #1}

\newcommand\liehat[1]{\left[ #1 \right]_\times}
\newcommand\lievee[1]{\left[ #1 \right]^\vee}
\newcommand\liehatvee[1]{\left[ #1 \right]^\vee_\times}

\newcommand\mlcomment[1]{\iffalse #1 \fi}
%\newcommand\mlcomment[1]{ #1 }

\newcounter{problemname}
\newenvironment{problem}{\stepcounter{problemname}\par\noindent\normf\textbf{\textcolor[rgb]{1,0,0}{题目\arabic{problemname}.} }}{\leavevmode\\\par}
\newenvironment{solution}{\par\noindent\normf\textbf{解答: }}{\leavevmode\\\par}
\newenvironment{note}{\par\noindent\normf\textbf{题目\arabic{problemname}的注记: }}{\leavevmode\\\par}


\begin{document}
	\begin{titlepage}
	    \centering
	    \includegraphics[width=0.4\textwidth]{whu_red.png}\par\vspace{1cm}
	    \vspace{4cm}
	    {\huge\kaishu “数值分析”编程作业\par 线性方程组$\boldsymbol{Ax}=\boldsymbol{b}$的求解 \par}
	    \vspace{3cm}
	    {\Large\kaishu 
	    \begin{center}\begin{tabular}{l}
	    姓名:陈烁龙\\
	    学号:\bfseries 2023202140019\\
	    学院:测绘学院
	    \end{tabular}\end{center}
	     \par}
	    
	
	    \vfill
	
	% Bottom of the page
	    {\large\kaishu\bfseries \today\par}
	\end{titlepage}
		% 换页
 		\thispagestyle{empty}
		\clearpage
		
		% 插入目录、图、表并换页
		\pagenumbering{roman}
		\tableofcontents
		\newpage
%		\listoffigures
%		\newpage
		\listoftables
		% 罗马字母形式的页码
		
		\clearpage
		% 从该页开始计数
		\setcounter{page}{1}
		% 阿拉伯数字形式的页码
		\pagenumbering{arabic}
	
	\section{\normf{代码运行环境说明}}
	
		\normf
		表\ref{tab:代码运行环境}显示了本次编程测试的运行环境。本次代码使用\emph{CPP 17}实现,使用到了线性代数库\emph{Eigen}。代码虽然在\emph{Ubuntu}上实现, 但由于是\emph{Cmake}架构的项目,所以具有跨平台的特征,即可以在\emph{Windows}平台和\emph{Mac}上编译运行。
		
		% Please add the following required packages to your document preamble:
		% \usepackage{multirow}
		\begin{table}[h]
		\centering
		\caption{\normf 代码运行环境}
		\label{tab:代码运行环境}
		\vspace{2mm}
		\begin{tabular}{ccc|cl}
		\toprule
		\multicolumn{1}{c|}{\multirow{2}{*}{Config.}} & OS name & Ubuntu 20.04.6 LTS & Processor              & 12th Gen Intel® Core™ i9-12900H × 20              \\ \cmidrule{2-5} 
		\multicolumn{1}{c|}{}                         & OS Type & 64-bit             & Graphics               & Mesa Intel® Graphics (ADL GT2)                    \\ \midrule[1pt]\midrule[1pt]
		\multicolumn{3}{c|}{Library}                                                 & \multicolumn{2}{c}{Link}                                                   \\ \midrule
		\multicolumn{3}{c|}{Eigen3}                                                  & \multicolumn{2}{c}{https://eigen.tuxfamily.org/index.php?title=Main\_Page}              \\ \bottomrule
		\end{tabular}
		\end{table}
		
	\section{\normf{LU分解}}
	\subsection{\normf{算法描述}}
	对于线性方程组$\boldsymbol{Ax=b}$中的系数方阵$\boldsymbol{A}$,可以通过将其进行LU分解,得到:
	\begin{equation}
	\boldsymbol{A}=\boldsymbol{LU}
	\end{equation}
	而后在方程求解的过程中,使用杜利特尔(Doolittle)分解求解线性方程组:
	\begin{equation}
	\boldsymbol{Ax}=\boldsymbol{b}\quad\Rightarrow\quad
	\boldsymbol{LUx}=\boldsymbol{b}\quad\Rightarrow\quad
	\boldsymbol{Ly}=\boldsymbol{b}\quad\Rightarrow\quad
	\boldsymbol{Ux}=\boldsymbol{y}\quad\Rightarrow\quad
	\boldsymbol{x}
	\end{equation}
	
	关于矩阵$\boldsymbol{A}$的LU分解方面,在进行高斯消去的过程中,初等行变换过程相当于左乘初等变换矩阵。对于第$k$次变换,相当于左乘:
	\begin{equation}
	\boldsymbol{L}_k=\begin{pmatrix}
	1\\
	0&1\\
	0&-l_{k+1,k}&1\\
	\cdots\\
	0&-l_{n,k}&0&\cdots&1
	\end{pmatrix}
	\quad
	\boldsymbol{L}_k^{-1}=\begin{pmatrix}
		1\\
		0&1\\
		0&l_{k+1,k}&1\\
		\cdots\\
		0&l_{n,k}&0&\cdots&1
		\end{pmatrix}
	\end{equation}
	其中$l_{i,k}=a^{(k)}_{i,k}/a^{(k)}_{k,k},i\in[k+1,n]$。于是,有:
	\begin{equation}
	\boldsymbol{A}^{(1)}=\boldsymbol{L}_1^{-1}\cdot\boldsymbol{L}_2^{-1}\cdots
	\boldsymbol{L}_n^{-1}=\boldsymbol{L}\boldsymbol{A}^{n}=\boldsymbol{LU}
	\end{equation}
	
	当然,在高斯消去法消去过程中可能出现$a_{k,k}^{k}=0$的情况,这时高斯消去法将无法进行。这时,可以在第k步消元前,在系数矩阵第k列的对角线以下的元素中找出绝对值最大的元,再继续消去计算,也即使用列主元Gauss消去法。
	
	
	\subsection{\normf{实验结果}}
	\subsubsection{\normf{小型随机矩阵测试}}
	$5\times 5$的系数矩阵$\boldsymbol{A}$:
	\begin{equation}
	\boldsymbol{A}=\begin{pmatrix}
	 0.68038 &-0.60490 &-0.04521&  0.83239& -0.96740\\
	-0.21123 &-0.32955 & 0.25774&  0.27142& -0.51423\\
	 0.56620 & 0.53646 &-0.27043&  0.43459& -0.72554\\
	 0.59688 &-0.44445 & 0.02680& -0.71679&  0.60835\\
	 0.82329 & 0.10794 & 0.90446&  0.21394& -0.68664\\
	\end{pmatrix}
	\end{equation}
	系数矩阵$\boldsymbol{A}$的LU分解结果:
	\begin{equation}
	\boldsymbol{L}=\begin{pmatrix}
	 1.00000&  0.00000&  0.00000&  0.00000 & 0.00000\\
	-0.31047&  1.00000&  0.00000&  0.00000 & 0.00000\\
	 0.83219& -2.00993&  1.00000&  0.00000 & 0.00000\\
	 0.87728& -0.16664&  0.41659&  1.00000 & 0.00000\\
	 1.21006& -1.62345&  5.27118&  2.46993 & 1.00000\\
	\end{pmatrix}
	\end{equation}
	\begin{equation}
	\boldsymbol{U}=\begin{pmatrix}
	 0.68038& -0.60490& -0.04521 & 0.83239& -0.96740\\
	 0.00000& -0.51736&  0.24371 & 0.52985& -0.81457\\
	 0.00000&  0.00000&  0.25702 & 0.80686& -1.55771\\
	 0.00000&  0.00000&  0.00000 &-1.69487&  1.97022\\
	 0.00000&  0.00000&  0.00000 & 0.00000&  2.50623\\
	\end{pmatrix}
	\end{equation}
	真值:
	\begin{equation}
	\boldsymbol{x}_{gt}^\top=\begin{pmatrix}
	-0.19811 &-0.74042& -0.78238 & 0.99785& -0.56349
	\end{pmatrix}
	\end{equation}
	解算结果:
	\begin{equation}
	\boldsymbol{x}_{sl}^\top=\begin{pmatrix}
	-0.19811 &-0.74042& -0.78238 & 0.99785& -0.56349
	\end{pmatrix}
	\end{equation}
	验证了算法的正确性。
	
	\subsubsection{\normf{大型随机矩阵测试}}
	通过随机函数,构建了一个$1000\times 1000$的随机矩阵,并使用所实现的算法进行了求解,统计其平均运行时间和计算误差,结果如下:
	% Please add the following required packages to your document preamble:
	% \usepackage{multirow}
	\begin{table}[h]
	\normf
	\centering
	\caption{\normf{测试结果}}
	\vspace{2mm}
	\begin{tabular}{c|ccc}
	\toprule
	\multirow{2}{*}{矩阵维数} & \multicolumn{2}{c}{运行时间(秒)} & \multirow{2}{*}{计算误差} \\ \cmidrule{2-3}
	                      & LU分解        & 方程求解(LU分解+回代) &                       \\ \midrule
	10                    & 5.89340e-06 & 1.43406e-05   & 4.82033e-15           \\
	100                   & 4.22002e-03 & 4.24337e-03   & 5.08518e-12           \\
	1000                  & 3.37420e+1  & 3.37437e+1    & 4.03739e-10           \\ \bottomrule
	\end{tabular}
	\end{table}
	
	由于该算法的计算量主要集中在LU分解,其时间复杂度为$\mathcal{O}(n^2)$,所以当矩阵的维数增加时,运行时间显著增加。由于计算机的舍入误差,当矩阵的维数增加时,计算次数增加,相应的误差也增加。
	
	\subsection{\normf{关键代码}}
	\begin{lstlisting}[caption=对矩阵$\boldsymbol{A}$的LU分解]
    void LUFactorization(const Eigen::MatrixXd &AMat, Eigen::MatrixXd &LMat, Eigen::MatrixXd &UMat) {
        assert(AMat.rows() == AMat.cols());
        auto n = AMat.rows();

        LMat = Eigen::MatrixXd::Identity(n, n);
        UMat = AMat;
        for (int c = 0; c < n - 1; ++c) {
            Eigen::MatrixXd curLMatInv = Eigen::MatrixXd::Identity(n, n);
            for (int r = c + 1; r < n; ++r) {
                double l = UMat(r, c) / UMat(c, c);
                curLMatInv(r, c) = l;
                UMat.row(r) = UMat.row(r) - l * UMat.row(c);
            }
            LMat = LMat * curLMatInv;
        }
    }
	\end{lstlisting}
	
	\begin{lstlisting}[caption=杜利特尔(Doolittle)分解求解线性方程组]
	 void EquSolvingByLU(const Eigen::MatrixXd &AMat, const Eigen::VectorXd &bVec, Eigen::VectorXd &xVec) {
	     assert(AMat.rows() == AMat.cols());
	     assert(AMat.rows() == bVec.rows());
	     auto n = AMat.rows();
	
	     Eigen::MatrixXd LMat, UMat;
	     ns_na::LUFactorization(AMat, LMat, UMat);
	     // A = LU, LUx = b, Ly = b, Ux = y
	     // ------
	     // Ly = b
	     // ------
	     Eigen::VectorXd yVec(n);
	     yVec(0) = bVec(0) / LMat(0, 0);
	     for (int r = 1; r < n; ++r) {
	         yVec(r) = bVec(r) - (LMat.block(r, 0, 1, r) * yVec.block(0, 0, r, 1))(0, 0);
	     }
	     // ------
	     // Ux = y
	     // ------
	     xVec = Eigen::VectorXd(n);
	     xVec(n - 1) = yVec(n - 1) / UMat(n - 1, n - 1);
	     for (int r = static_cast<int>(n) - 2; r >= 0; --r) {
	         xVec(r) = (yVec(r) - (UMat.block(r, r + 1, 1, n - r - 1)
	                               * xVec.block(r + 1, 0, n - r - 1, 1))(0, 0)) / UMat(r, r);
	     }
	 }
	\end{lstlisting}
	
	\newpage
	\section{\normf{梯度下降}}
	\subsection{\normf{算法描述}}
	\subsection{\normf{实验结果}}
	\subsubsection{\normf{小型稀疏矩阵测试}}
	\subsubsection{\normf{大型稀疏矩阵测试}}
	\subsection{\normf{关键代码}}
	
	\newpage
	\section{\normf{共轭梯度}}
	\subsection{\normf{算法描述}}
	\subsection{\normf{实验结果}}
	\subsubsection{\normf{小型稀疏矩阵测试}}
	\subsubsection{\normf{大型稀疏矩阵测试}}
	\subsection{\normf{关键代码}}
		
	\newpage
	\section{ACKNOWLEDGMENT}
	\begin{tcolorbox}[colback=white,colframe=white!70!black,title={\bfseries Author Information}]
	\par\noindent
		\parbox[t]{\linewidth}{
	 \noindent\parpic{\includegraphics[height=2in,width=1in,clip,keepaspectratio]{ShuolongChen_grey.jpg}}
	 \noindent{\bfseries Shuolong Chen}\emph{
	 received the B.S. degree in geodesy and geomatics engineering from Wuhan University, Wuhan China, in 2023.
	 He is currently a master candidate at the school of Geodesy and Geomatics, Wuhan University. His area of research currently focuses on integrated navigation systems and multi-sensor fusion.
	 Contact him via e-mail: shlchen@whu.edu.cn.
	 }}
	\end{tcolorbox}
		
		
\end{document}


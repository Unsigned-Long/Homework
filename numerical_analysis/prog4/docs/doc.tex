\documentclass[12pt, onecolumn]{article}

% 引入相关的包
\usepackage{amsmath, listings, fontspec, geometry, graphicx, ctex, color, subfigure, amsfonts, amssymb}
\usepackage{multirow}
\usepackage[table,xcdraw]{xcolor}
\usepackage[ruled]{algorithm2e}
\usepackage[hidelinks]{hyperref}

		\usepackage{graphicx}
		\usepackage[most]{tcolorbox}
\hypersetup{
	colorlinks=true,
	linkcolor=red,
	citecolor=red,
}
\usepackage{booktabs}
\usepackage{multirow}
\usepackage{picins}

% 设定页面的尺寸和比例
\geometry{left = 1.5cm, right = 1.5cm, top = 1.5cm, bottom = 1.5cm}

% 设定两栏之间的间距
\setlength\columnsep{1cm}

% 设定字体,为代码的插入作准备
\newfontfamily\ubuntu{Ubuntu Mono}
\newfontfamily\consolas{Consolas}

% 代码块的风格设定
\lstset{
	language=C++,
	basicstyle=\small\ubuntu,
	keywordstyle=\textbf,
	stringstyle=\itshape,
	commentstyle=\itshape,
	numberstyle=\scriptsize\ubuntu,
	showstringspaces=false,
	numbers=left,
	numbersep=8pt,
	tabsize=2,
	frame=single,
	framerule=1pt,
	columns=fullflexible,
	breaklines,
	frame=shadowbox, 
	backgroundcolor=\color[rgb]{0.97,0.97,0.97}
}

% 字体族的定义
% \fangsong \songti \heiti \kaishu
\newcommand\normf{\fangsong}
\newcommand\boldf{\heiti}
\newcommand\keywords[1]{\boldf{关键词:} \normf #1}

\newcommand\liehat[1]{\left[ #1 \right]_\times}
\newcommand\lievee[1]{\left[ #1 \right]^\vee}
\newcommand\liehatvee[1]{\left[ #1 \right]^\vee_\times}

\newcommand\mlcomment[1]{\iffalse #1 \fi}
%\newcommand\mlcomment[1]{ #1 }

\newcounter{problemname}
\newenvironment{problem}{\stepcounter{problemname}\par\noindent\normf\textbf{\textcolor[rgb]{1,0,0}{题目\arabic{problemname}.} }}{\leavevmode\\\par}
\newenvironment{solution}{\par\noindent\normf\textbf{解答: }}{\leavevmode\\\par}
\newenvironment{note}{\par\noindent\normf\textbf{题目\arabic{problemname}的注记: }}{\leavevmode\\\par}


\begin{document}
	\begin{titlepage}
	    \centering
	    \includegraphics[width=0.4\textwidth]{whu_red.png}\par\vspace{1cm}
	    \vspace{4cm}
	    {\huge\kaishu “数值分析”编程作业\par 多项式插值 \par}
	    \vspace{3cm}
	    {\Large\kaishu 
	    \begin{center}\begin{tabular}{l}
	    姓名:陈烁龙\\
	    学号:\bfseries 2023202140019\\
	    学院:测绘学院
	    \end{tabular}\end{center}
	     \par}
	    
	
	    \vfill
	
	% Bottom of the page
	    {\large\kaishu\bfseries \today\par}
	\end{titlepage}
		% 换页
 		\thispagestyle{empty}
		\clearpage
		
		% 插入目录、图、表并换页
		\pagenumbering{roman}
		\tableofcontents
		\newpage
		\listoffigures
		\newpage
		\listoftables
		% 罗马字母形式的页码
		
		\clearpage
		% 从该页开始计数
		\setcounter{page}{1}
		% 阿拉伯数字形式的页码
		\pagenumbering{arabic}
	
	\section{\normf{代码运行环境说明}}
	
		\normf\bfseries
		表\ref{tab:代码运行环境}显示了本次编程测试的运行环境。本次代码使用\emph{CPP 17}实现,使用到了线性代数库\emph{Eigen}。代码虽然在\emph{Ubuntu}上实现, 但由于是\emph{Cmake}架构的项目,所以具有跨平台的特征,即可以在\emph{Windows}平台和\emph{Mac}上编译运行。
		
		绘图使用\emph{Python}中的\emph{matploylib}库实现,具体的绘图脚本在\emph{script}文件夹下。
		
		% Please add the following required packages to your document preamble:
		% \usepackage{multirow}
		\begin{table}[h]
		\centering
		\caption{\normf 代码运行环境}
		\label{tab:代码运行环境}
		\vspace{2mm}
		\begin{tabular}{ccc|cl}
		\toprule
		\multicolumn{1}{c|}{\multirow{2}{*}{Config.}} & OS name & Ubuntu 20.04.6 LTS & Processor              & 12th Gen Intel® Core™ i9-12900H × 20              \\ \cmidrule{2-5} 
		\multicolumn{1}{c|}{}                         & OS Type & 64-bit             & Graphics               & Mesa Intel® Graphics (ADL GT2)                    \\ \midrule[1pt]\midrule[1pt]
		\multicolumn{3}{c|}{Library}                                                 & \multicolumn{2}{c}{Link}                                                   \\ \midrule
		\multicolumn{3}{c|}{Eigen3}                                                  & \multicolumn{2}{c}{https://eigen.tuxfamily.org/index.php?title=Main\_Page}              \\ \bottomrule
		\end{tabular}
		\end{table}
		
		本次的作业已上传到(\url{https://github.com/Unsigned-Long/Homework/tree/master/numerical_analysis/prog4})。
		
	\section{\normf{复化梯形求积公式}}
	给定被积函数$f(x)$,其在区间$[a,b]$的梯形数值积分为:
	\begin{equation}
	\int_{a}^{b}f(x)dx\approx T(a,b)=\frac{b-a}{2}\left(
	f(a)+f(b)
	\right) 
	\end{equation}
	高阶Newton-Cotes公式会出现数值不稳定, 而低阶Newton-Cotes
	公式有时又不能满足精度要求. 解决这
	个矛盾的办法是将积分区间$[a,b]$分成若
	干小区间, 在每个小区间上用低阶求积
	公式计算, 然后将它们加起来, 这就是复
	化(composite)求积方法的思想.
	给定被积函数$f(x)$,其在区间$[a,b]$的复化梯形数值积分为:
	\begin{equation}
	\int_{a}^{b}f(x)dx\approx T_c(a,b,n)=\frac{h}{2}\left( f(a)+2\times\sum_{i=1}^{n-1}f(x_k)+f(b)\right)
	\end{equation}
	其中:
	\begin{equation}
		h=\frac{b-a}{n}\qquad
		x_k=a+k\times h
	\end{equation}
	
	\section{\normf{复化Simpson求积公式}}
	给定被积函数$f(x)$,其在区间$[a,b]$的Simpson数值积分为:
	\begin{equation}
		\int_{a}^{b}f(x)dx\approx S(a,b)=\frac{b-a}{6}\left(
		f(a)+4\times f\left( \frac{a+b}{2}\right) +f(b)
		\right) 
	\end{equation}
	其在区间$[a,b]$的复化Simpson数值积分为:
	\begin{equation}
	\int_{a}^{b}f(x)dx\approx S_c(a,b,n)=\frac{h}{3}\left( f(a)+4\times\sum_{k=1}^{n}f(x_{2k-1})+2\times\sum_{k=1}^{n-1}f(x_{2k})+f(b)\right) 
	\end{equation}
	其中:
	\begin{equation}
		h=\frac{b-a}{2n}\qquad
		x_k=a+k\times h
	\end{equation}
	
	\section{\normf{Romberg求积公式}}
	给定被积函数$f(x)$,求积为区间$[a,b]$,将其分为$2^k$份,则有:
	\begin{equation}
	\begin{aligned}
	T_0^{(0)}&=\frac{b-a}{2}(f(a)+f(b))
	\\
	T_0^{(k)}&=\frac{1}{2}T_0^{(k-1)}+\frac{b-a}{2^k}\sum_{i=1}^{2^{k-1}}f\left(
	a+(2\times i-1)\times\frac{b-a}{2^k}
	\right) 
	\\
	T_m^{(l)}&=\frac{4^m\times T_{m-1}^{(l+1)}-T_{m-1}^{(l)}}{4^m-1}
	\end{aligned}
	\end{equation}
	
	\section{\normf{高斯型求积公式}}
	n次多项式为勒让德多项式定义为:
	\begin{equation}
	P_n(x)=\frac{1}{2^n\times n!}\times \frac{d^n}{dx^n}
	(x^2-1)^n
	\end{equation}
	当$n=0,1,2,3$时,有:
	\begin{equation}
	\begin{aligned}
	P_0(x)=&1
	\\
	P_1(x)=&x
	\qquad real\;roots:0
	\\
	P_2(x)=&\frac{1}{2}(3x^2-1)
	\qquad real\;roots:\pm\sqrt{\frac{1}{3}}
	\\
	P_3(x)=&\frac{1}{2}(5x^3-3x)
	\qquad real\;roots:0,\pm\sqrt{\frac{3}{5}}
	\end{aligned}
	\end{equation}
	给定被积函数$f(x)$,求积为区间$[a,b]$,令:
	\begin{equation}
	x=\frac{b-a}{2}\times t+\frac{b+a}{2}
	\rightarrow
	f(x)=f\left( \frac{b-a}{2}\times t+\frac{b+a}{2}\right) 
	\quad
	a\Leftarrow -1\quad b\Leftarrow 1
	\end{equation}
	将积分区间归化到$[-1,1]$,有高斯积分:
	\begin{equation}
	G(a,b,n)=\sum_{j=1}^{n}A_j\times 
	f\left( 
	\frac{b-a}{2}\times t_j+\frac{b+a}{2}
	\right) 
	\end{equation}
	其中$t_j$为$P_n(x)$的第$j$个实根,$A_j$为:
	\begin{equation}
	A_j=\int_{-1}^{1}\prod_{i=1,i!=j}^{n}\frac{t-t_i}{t_j-t_i}dt
	\end{equation}
	特别的,当$n=2$时:
	\begin{equation}
	\begin{aligned}
	A_1&=\int_{-1}^{1}\frac{t-t_2}{t_1-t_2}dt=\frac{1}{t_1-t_2}\left[
		\frac{1}{2}\times t^2-t_2\times t
		\right]_{-1}^{1}=-\frac{2\times t_2}{t_1-t_2}=1
		\\
		A_2&=\int_{-1}^{1}\frac{t-t_1}{t_2-t_1}dt=-\frac{2\times t_1}{t_2-t_1}=1
		\\
		t_1&=-\sqrt{\frac{1}{3}}\qquad t_2=\sqrt{\frac{1}{3}}
	\end{aligned}
	\end{equation}
	当$n=3$时:
	\begin{equation}
	\begin{aligned}
	A_1
%	&=\int_{-1}^{1}\frac{t-t_2}{t_1-t_2}\cdot\frac{t-t_3}{t_1-t_3}dt
%	\\&=
%	\frac{1}{(t_1-t_2)\cdot(t_1-t_3)}\int_{-1}^{1}(t^2-(t_2+t_3)t+t_2t_3)dt
%	\\&=\frac{1}{(t_1-t_2)\cdot(t_1-t_3)}\left[
%	\frac{1}{3}t^3-\frac{t_2+t_3}{2}t^2+t_2t_3t
%	\right] _{-1}^{1}
%	\\
%	&=\frac{1}{(t_1-t_2)\cdot(t_1-t_3)}\cdot\left(
%	\frac{2}{3}+2t_2t_3
%	\right) 
%	\\
%	&
	=\frac{5}{9}
	\qquad A_2
%	=\frac{1}{(t_2-t_1)\cdot(t_2-t_3)}\cdot\left(
%		\frac{2}{3}+2t_1t_3
%		\right)
		=\frac{8}{9}
		\qquad A_3
%		=\frac{1}{(t_3-t_1)\cdot(t_3-t_2)}\cdot\left(
%			\frac{2}{3}+2t_1t_2
%			\right)
		=\frac{5}{9}
		\qquad t_1=-\sqrt{\frac{3}{5}}\qquad t_2=0\qquad t_3=\sqrt{\frac{3}{5}}
	\end{aligned}
	\end{equation}
	
	\section{\normf{测试}}
	\subsection{\normf{Runge函数}}
	Runge函数为:
	\begin{equation}
	f(x)=\frac{1}{1+x^2},x\in[-5,5]
	\end{equation}
	其积分真值为:
	\begin{equation}
	\int_{-5}^{5}\frac{1}{1+x^2}dx=\left[ \arctan(x) \right] _{-5}^{5}=\arctan(5)-\arctan(-5)=2.746801534
	\end{equation}
	
	\subsection{\normf{第一类不完全椭圆积分}}
		第一类不完全椭圆积分为:
		\begin{equation}
		f(\phi,\alpha)=\frac{1}{\sqrt{1-(\sin\theta\sin\alpha)^2}},\theta\in[0,\phi]
		\end{equation}
		其中$\phi$为幅度,$\alpha$为模角。
	
	\subsection{\normf{Zigzag函数}}
	考虑的Zigzag函数为:
	\begin{equation}
	f(x)=x^2\times\sin(5x)^2,x\in[1,2]
	\end{equation}
	
	
	
	
	
		
	\newpage
	\section*{ACKNOWLEDGMENT}
	\begin{tcolorbox}[colback=white,colframe=white!70!black,title={\bfseries Author Information}]
	\par\noindent
		\parbox[t]{\linewidth}{
	 \noindent\parpic{\includegraphics[height=2in,width=1in,clip,keepaspectratio]{ShuolongChen_grey.jpg}}
	 \noindent{\bfseries Shuolong Chen}\emph{
	 received the B.S. degree in geodesy and geomatics engineering from Wuhan University, Wuhan China, in 2023.
	 He is currently a master candidate at the school of Geodesy and Geomatics, Wuhan University. His area of research currently focuses on integrated navigation systems and multi-sensor fusion.
	 Contact him via e-mail: shlchen@whu.edu.cn.
	 }}
	\end{tcolorbox}
		
		
\end{document}


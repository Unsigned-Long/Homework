\documentclass[12pt, onecolumn]{article}

% 引入相关的包
\usepackage{amsmath, listings, fontspec, geometry, graphicx, ctex, color, subfigure, amsfonts, amssymb}
\usepackage{multirow}
\usepackage[table,xcdraw]{xcolor}
\usepackage[ruled]{algorithm2e}
\usepackage[hidelinks]{hyperref}
\hypersetup{
	colorlinks=true,
	linkcolor=red,
	citecolor=red,
}

% 设定页面的尺寸和比例
\geometry{left = 1.5cm, right = 1.5cm, top = 1.5cm, bottom = 1.5cm}

% 设定两栏之间的间距
\setlength\columnsep{1cm}

% 设定字体,为代码的插入作准备
\newfontfamily\ubuntu{Ubuntu Mono}
\newfontfamily\consolas{Consolas}

% 头部信息
\title{\normf{机器视觉编程1:直线和角点提取} }
\author{\normf{陈烁龙 2023202140019}}
\date{\normf{\today}}

% 代码块的风格设定
\lstset{
	language=C++,
	basicstyle=\small\ubuntu,
	keywordstyle=\textbf,
	stringstyle=\itshape,
	commentstyle=\itshape,
	numberstyle=\scriptsize\ubuntu,
	showstringspaces=false,
	numbers=left,
	numbersep=8pt,
	tabsize=2,
	frame=single,
	framerule=1pt,
	columns=fullflexible,
	breaklines,
	frame=shadowbox, 
	backgroundcolor=\color[rgb]{0.97,0.97,0.97}
}

% 字体族的定义
% \fangsong \songti \heiti \kaishu
\newcommand\normf{\fangsong}
\newcommand\boldf{\heiti}
\newcommand\keywords[1]{\boldf{关键词:} \normf #1}

\newcommand\liehat[1]{\left[ #1 \right]_\times}
\newcommand\lievee[1]{\left[ #1 \right]^\vee}
\newcommand\liehatvee[1]{\left[ #1 \right]^\vee_\times}

\newcommand\mlcomment[1]{\iffalse #1 \fi}
%\newcommand\mlcomment[1]{ #1 }

\newcounter{problemname}
\newenvironment{problem}{\stepcounter{problemname}\par\noindent\normf\textbf{\textcolor[rgb]{1,0,0}{题目\arabic{problemname}.} }}{\leavevmode\\\par}
\newenvironment{solution}{\par\noindent\normf\textbf{解答: }}{\leavevmode\\\par}
\newenvironment{note}{\par\noindent\normf\textbf{题目\arabic{problemname}的注记: }}{\leavevmode\\\par}


\begin{document}
	
	% 插入头部信息
		\maketitle
		% 换页
		\thispagestyle{empty}
		\clearpage
		
		% 插入目录、图、表并换页
		\pagenumbering{roman}
		\tableofcontents
		\newpage
		\listoffigures
%		\newpage
%		\listoftables
		% 罗马字母形式的页码
		
		\clearpage
		% 从该页开始计数
		\setcounter{page}{1}
		% 阿拉伯数字形式的页码
		\pagenumbering{arabic}
	
	\section{\normf{角点提取}}
	\normf
	\subsection{\normf{Harris角点}}
	Harris角点检测算法诞生于1988年,是在Moravec算法(另一种角点检测算法)的基础上提出的,其克服了图像中噪声对角点检测的影响。定义某个角点的能量为:
	\begin{equation}
	\mathcal{E}(u,v)=\sum_{(x,y)\in\mathcal{W}}w(x,y)\left[ 
	I(x+u,y+v)-I(x,y)
	\right] ^2
	\end{equation}
	其中$(x,y)$为窗口$\mathcal{W}$内的点位,$I(\cdot)$为某个像素位置处的强度,$w(x,y)$是二维的高斯窗口函数。将上式展开得到:
	\begin{equation}
	\mathcal{E}(u,v)=\sum_{(x,y)\in\mathcal{W}}w(x,y)\left[ 
		u\cdot I_x+v\cdot I_y
		\right] ^2=\begin{pmatrix}
		u&v
		\end{pmatrix}\boldsymbol{M}\begin{pmatrix}
		u\\v
		\end{pmatrix}
	\end{equation}
	其中:
	\begin{equation}
	\boldsymbol{M}=\sum_{(x,y)\in\mathcal{W}}w(x,y)\begin{pmatrix}
	I_x^2 &I_xI_y\\
	I_xI_y&I_y^2
	\end{pmatrix}
	\end{equation}
	也即为Harris矩阵。设$\lambda_1$和$\lambda_2$为$\boldsymbol{M}$的两个特征值,则有:
	\begin{enumerate}
	\item $\lambda_1$和$\lambda_2$均小:光滑区域;
	\item $\lambda_1\gg \lambda_2$或者$\lambda_2\gg \lambda_1$:边缘区域;
	\item $\lambda_1$和$\lambda_2$均大:角点区域。
	\end{enumerate}
	基于此,若要识别角点,构建如下的响应函数:
	\begin{equation}
	R=\det(\boldsymbol{M})-k\times (\mathrm{trace}(\boldsymbol{M}))^2=\lambda_1\lambda_2-\alpha\times(\lambda_1+\lambda_2)^2
	\end{equation}
	当$R$大于某个阈值时,认为窗口中心为一个角点。
	
	图\ref{fig:Harris角点提取结果}为实验的结果,一共选取了两个场景的照片进行了测试(室内环境和室外环境)。其中图\ref{fig:图像1的Harris响应函数}和图\ref{fig:图像2的Harris响应函数}为计算Harris响应函数得到的图,其中黑色表示负值区域,也即边缘区域;白色表示正值区域,也即角点区域;红色区域表示平坦区域,也响应值小的区域。
	图\ref{fig:图像1的角点提取结果}和图\ref{fig:图像2的角点提取结果}为将Harris算法提取得到的角点绘制到原始灰度图像上的结果。
	在测试中,两个图像都使用了相同的参数:窗口大小,核函数大小,角点阈值为。
	可以发现,似然Harris能够提取到一定的角点,但是:
	\begin{enumerate}
	\item 阈值决定角点的数量。阈值过低会导致大量角点和非角点被提取出来;阈值过小会导致提取的角点数目过少,忽视了其他的角点;
	\item 虽然使用了核函数,但是算法任然对图像中的噪声比较敏感,可能会将噪声误判为角点;
	
	\item 对尺度变化不敏感:Harris角点检测算法在处理尺度变化较大的图像时效果较差;
	
	\item 对旋转变化不敏感:算法对图像的旋转变化不敏感,可能会导致角点检测结果不准确;
	\end{enumerate}
	
	为了解决Harris角点检测算法的一些缺点,研究者们提出了一些改进的方法,如尺度不变特征变换(Scale-Invariant Feature Transform,SIFT)和加速稳健特征(Speeded Up Robust Features,SURF)等。这些改进算法在角点检测的准确性和鲁棒性上有所提升。具体的算法测试案例见下文。
	
	\begin{figure}[t]
	\centering
	\subfigure[\normf{图像1的Harris响应函数}]{
				\centering
				\includegraphics[width=0.45\linewidth]{../data/harris/img1/response.jpg}
				\label{fig:图像1的Harris响应函数}
			}
	\subfigure[\normf{图像2的Harris响应函数}]{
				\centering
					\includegraphics[width=0.45\linewidth]{../data/harris/img2/response.jpg}
					
					\label{fig:图像2的Harris响应函数}
				}
	\subfigure[\normf{图像1的角点提取结果}]{
		\centering
		\includegraphics[width=0.45\linewidth]{../data/harris/img1/marked_image.jpg}
		\label{fig:图像1的角点提取结果}
	}
	\subfigure[\normf{图像2的角点提取结果}]{
		\centering
			\includegraphics[width=0.45\linewidth]{../data/harris/img2/marked_image.jpg}
			
			\label{fig:图像2的角点提取结果}
		}
	\caption{\normf Harris角点提取结果}
	\label{fig:Harris角点提取结果}
	\end{figure}
	
	
	
	\subsection{\normf{FAST角点}}
	FAST角点检测是一种快速的特征点检测方法。它的基本原理是使用一个周长为16个像素点(半径为3的圆)来判定其圆心像素p是否为角点1。FAST角点定义如下:若某像素与其周围邻域内足够多的像素点相差较大,则该像素可能是角点。
	
	FAST算法的流程如下:
	\begin{enumerate}
	\item 选择一个像素点作为中心点。
	\item 将中心点的亮度值设为$I_p$。
	\item 以该像素点为中心的一个半径等于3像素的离散化圆上的16个像素,分别记为$p_i,i\in[0,1,\cdots,16]$,亮度值分别记为$I_{p_1}\sim I_{p_{16}}$。
	\item 选定一个阈值$t$。
	\item 计算$p_1$、$p_9$、$p_5$和$p_{13}$与中心$p$的像素差,若它们的绝对值有至少3个超过阈值$t$,则认为该像素是候选角点。
	\item 在候选角点的情况下,计算$p_1$到$p_{16}$这16个点与中心$p$的像素差,若它们有至少连续9个超过阈值t,则认为该像素是角点。
	\end{enumerate}
	FAST角点检测方法具有计算速度快、精度高等优点。图\ref{fig:FAST角点提取结果}为FAST角点提取结果。
	\begin{figure}[t]
		\centering
		\subfigure[\normf{图像1的FAST角点提取结果}]{
			\centering
			\includegraphics[width=0.45\linewidth]{../data/fast/img1/marked_image.jpg}
			\label{fig:图像1的FAST角点提取结果}
		}
		\subfigure[\normf{图像2的FAST角点提取结果}]{
			\centering
				\includegraphics[width=0.45\linewidth]{../data/fast/img2/marked_image.jpg}
				
				\label{fig:图像2的FAST角点提取结果}
			}
		\caption{\normf FAST角点提取结果}
		\label{fig:FAST角点提取结果}
		\end{figure}
		
	FAST角点检测虽然速度较快,但是其存在如下的缺点:
	\begin{enumerate}
	\item 检测到的很多特征点是连在一起的;
	\item 检测出来的角点不是最优的,这是因为它的效果是依靠角点外围的排列和分布;
	\end{enumerate}
	
	\subsection{\normf{ORB角点}}
	ORB(Oriented FAST and Rotated BRIEF)是一种特征点检测和提取算法,由Ethan Rublee等人于2011年首次提出。它是在FAST角点检测和BRIEF描述子的基础上进行改进得到的。ORB特征点具有实时性好、提取效果不错等优势,因此被广泛应用于需要实时处理任务的场景中(如著名的ORB-SLAM),它旨在提供一种快速高效的替代方法,以取代SIFT算法,是目前最快速稳定的特征点检测和提取算法(ORB是SIFT算法的100倍,是SURF的10倍)。特点:角度不变性,尺度不变性,计算速度快。
	
	ORB特征点的提取过程如下:
	\begin{enumerate}
	\item 首先,使用FAST角点检测算法提取图像中的候选角点。
	\item 对每个候选角点计算其方向,以便后续进行旋转不变性处理。
	\item 使用BRIEF描述子对每个候选角点进行描述。
	\end{enumerate}
	
	图\ref{fig:ORB角点提取结果}为FAST角点提取结果。与FAST角点检测器相比,ORB角点检测器具有以下优势:
	\begin{enumerate}
	\item 方向性:ORB角点检测器可以计算关键点的方向,从而提供旋转不变性。
	\item 描述子:ORB角点检测器使用BRIEF描述子对每个关键点进行描述,这些描述子是一种二进制描述子,用于表示特征点。
	\end{enumerate}
	
		\begin{figure}[t]
			\centering
			\subfigure[\normf{图像1的ORB角点提取结果}]{
				\centering
				\includegraphics[width=0.45\linewidth]{../data/orb/img1/marked_image.jpg}
				\label{fig:图像1的ORB角点提取结果}
			}
			\subfigure[\normf{图像2的ORB角点提取结果}]{
				\centering
					\includegraphics[width=0.45\linewidth]{../data/orb/img2/marked_image.jpg}
					
					\label{fig:图像2的ORB角点提取结果}
				}
			\caption{\normf ORB角点提取结果}
			\label{fig:ORB角点提取结果}
			\end{figure}
	
	\subsection{\normf{SIFT角点}}
	2004年,不列颠哥伦比亚大学的D.Lowe在他的论文《Distinctive Image Features from Scale-Invariant Keypoints》中提出了一种新算法——尺度不变特征变换(SIFT),该算法提取关键点并计算其描述符。
	
	SIFT算法主要涉及五个步骤:
	\begin{enumerate}
	\item 尺度空间极值检测:找到跨尺度和空间的局部最大值;
	
	\item 特征点定位:一旦找到潜在的关键点位置,就必须对其进行细化以获得更准确的结果。使用尺度空间的泰勒级数展开来获得更准确的极值位置,如果该极值处的强度小于阈值,则将其拒绝。
	
	\item 方向分配:为每个关键点分配一个方向,以实现图像旋转的不变性。
	
	\item 特征点描述符:选取关键点周围的邻域,分为小的子块。对于每个子块,创建直方图,而后表示为向量以形成关键点描述符。
	
	\item 特征点匹配:通过识别最近的邻居来匹配两个图像之间的关键点。但在某些情况下,第二个最接近的匹配可能非常接近第一个。这可能是由于噪音或其他原因造成的。在这种情况下,采用最近距离与次近距离之比。如果大于0.8,则被拒绝。
	\end{enumerate}
	
	SIFT算法的主要优点在于:
	\begin{enumerate}
	\item 尺度不变性:SIFT算法可以在图像的不同尺度下检测和描述特征,使其对图像的缩放、旋转和仿射变换具有不变性。
	\item 旋转不变性:SIFT算法可以检测和描述图像中的旋转特征,使其对图像的旋转具有不变性。
	\item 局部性:SIFT算法提取的特征是局部特征,对遮挡和杂乱背景具有鲁棒性。
	\item 独特性:SIFT算法提取的特征在图像中是独一无二的,可以与大量对象进行匹配。
	\item 数量:SIFT算法可以生成许多特征点,即使是小物体也可以生成许多特征点。
	\end{enumerate}
	
	图\ref{fig:SIFT角点提取结果}为SIFT角点提取结果。
		\begin{figure}[t]
			\centering
			\subfigure[\normf{图像1的SIFT角点提取结果}]{
				\centering
				\includegraphics[width=0.45\linewidth]{../data/sift/img1/marked_image.jpg}
				\label{fig:图像1的SIFT角点提取结果}
			}
			\subfigure[\normf{图像2的SIFT角点提取结果}]{
				\centering
					\includegraphics[width=0.45\linewidth]{../data/sift/img2/marked_image.jpg}
					
					\label{fig:图像2的SIFT角点提取结果}
				}
			\caption{\normf SIFT角点提取结果}
			\label{fig:SIFT角点提取结果}
			\end{figure}
	
	\subsection{\normf{SURF角点}}
	2006 年,Bay, H.、Tuytelaars, T. 和 Van Gool, L 三人发表了另一篇论文《SURF:加速鲁棒特征》,引入了一种称为 SURF 的新算法。顾名思义,它是 SIFT 的加速版本。
	
	在 SIFT 中,Lowe 用高斯差分逼近高斯拉普拉斯算子来寻找尺度空间。SURF 更进一步,使用 Box Filter 来近似 LoG。这种近似的一大优点是,可以借助积分图像轻松计算与盒式滤波器的卷积。并且可以针对不同的规模并行完成。分析表明,它比 SIFT 快 3 倍,而性能与 SIFT 相当。SURF擅长处理模糊和旋转的图像,但不擅长处理视点变化和光照变化。
	
		
		图\ref{fig:SURF角点提取结果}为SIFT角点提取结果。
		\begin{figure}[t]
			\centering
			\subfigure[\normf{图像1的SURF角点提取结果}]{
				\centering
				\includegraphics[width=0.45\linewidth]{../data/surf/img1/marked_image.jpg}
				\label{fig:图像1的SURF角点提取结果}
			}
			\subfigure[\normf{图像2的SURF角点提取结果}]{
				\centering
					\includegraphics[width=0.45\linewidth]{../data/surf/img2/marked_image.jpg}
					
					\label{fig:图像2的SURF角点提取结果}
				}
			\caption{\normf SURF角点提取结果}
			\label{fig:SURF角点提取结果}
			\end{figure}
				
	\section{\normf{直线提取}}
	\normf
	
	\subsection{\normf LSD算法1}
	
	图\ref{fig:LSD1直线提取结果}为LSD1直线提取结果。
	\begin{figure}[t]
		\centering
		\subfigure[\normf{图像1的LSD1直线提取结果}]{
			\centering
			\includegraphics[width=0.45\linewidth]{../data/lsd_sample/img1/marked_image.jpg}
		}
		\subfigure[\normf{图像2的LSD1直线提取结果}]{
			\centering
				\includegraphics[width=0.45\linewidth]{../data/lsd_sample/img2/marked_image.jpg}
			}
		\caption{\normf LSD1直线提取结果}
		\label{fig:LSD1直线提取结果}
		\end{figure}
	
	\subsection{\normf LSD算法2}
	
		图\ref{fig:LSD2直线提取结果}为LSD2直线提取结果。
	\begin{figure}[t]
		\centering
		\subfigure[\normf{图像1的LSD2直线提取结果}]{
			\centering
			\includegraphics[width=0.45\linewidth]{../data/lsd_complex/img1/marked_image.jpg}
		}
		\subfigure[\normf{图像2的LSD2直线提取结果}]{
			\centering
				\includegraphics[width=0.45\linewidth]{../data/lsd_complex/img2/marked_image.jpg}
			}
		\caption{\normf LSD2直线提取结果}
		\label{fig:LSD2直线提取结果}
		\end{figure}
	
	\section{\normf{附录:代码}}
	\begin{lstlisting}[caption={\normf 点和直线的数据结构}]
	    struct Entity {
	    public:
	        static pangolin::ColourWheel WHEEL;
	
	        using Ptr = std::shared_ptr<Entity>;
	
	        virtual std::string Type() { return "Entity"; }
	
	        virtual void Draw(const cv::Mat &img) = 0;
	    };
	
	    struct Corner : public Entity {
	    public:
	        using Ptr = std::shared_ptr<Corner>;
	
	    public:
	        cv::Point2f p;
	
	    public:
	        explicit Corner(cv::Point2f pos);
	
	        static Ptr Create(const cv::Point2f &pos);
	
	        std::string Type() override;
	
	        void Draw(const cv::Mat &img) override;
	
	        friend std::ostream &operator<<(std::ostream &os, const Corner &corner);
	
	    public:
	        // Serialization
	        template<class Archive>
	        void serialize(Archive &archive) {
	            archive(CEREAL_NVP(p));
	        }
	
	    };
	
	    struct Line : public Entity {
	    public:
	        using Ptr = std::shared_ptr<Line>;
	
	    public:
	        cv::Point2f p1, p2;
	
	        Line(cv::Point2f p1, cv::Point2f p2);
	
	        static Ptr Create(const cv::Point2f &p1, const cv::Point2f &p2);
	
	        std::string Type() override;
	
	        void Draw(const cv::Mat &img) override;
	
	        friend std::ostream &operator<<(std::ostream &os, const Line &line);
	
	    public:
	        // Serialization
	        template<class Archive>
	        void serialize(Archive &archive) {
	            archive(CEREAL_NVP(p1), CEREAL_NVP(p2));
	        }
	
	    };
	\end{lstlisting}
	\begin{lstlisting}[caption={\normf Harris角点检测代码}]
	    cv::Mat dstImg;
	    cv::cornerHarris(img, dstImg, 2, 3, 0.04);
	    mats.insert({"response", ns_mv::ConvertToVisibleMat(dstImg, 0.01, 0.01)});
	
	    // find the min and max value
	    double min, max;
	    cv::minMaxIdx(dstImg, &min, &max);
	    // convert the dstImg to binary image based on the threshold
	    cv::Mat filteredImg;
	    cv::threshold(dstImg, filteredImg, (max + min) * 0.2, 255, cv::ThresholdTypes::THRESH_BINARY);
	    mats.insert({"binary_corner", filteredImg});
	
	    // mark corners to source image
	    cv::Mat markedImg;
	    cv::cvtColor(img, markedImg, cv::COLOR_GRAY2BGR);
	    int rows = img.rows, cols = img.cols;
	    for (int i = 0; i < rows; ++i) {
	        for (int j = 0; j < cols; ++j) {
	            float val = filteredImg.at<float>(i, j);
	            if (val == 255.0) {
	                corners.push_back(Corner::Create(cv::Point2f((float) j, (float) i)));
	            }
	        }
	    }
	
	    for (const auto &elem: corners) {
	        elem->Draw(markedImg);
	    }
	    mats.insert({"marked_image", markedImg});
	\end{lstlisting}
	
	\begin{lstlisting}[caption={\normf FAST角点检测代码}]
	    auto detector = cv::FastFeatureDetector::create(60);
	    std::vector<cv::KeyPoint> keyPoints;
	    detector->detect(img, keyPoints);
	
	    for (const auto &point: keyPoints) {
	        corners.push_back(Corner::Create(point.pt));
	    }
	
	    // mark corners to source image
	    cv::Mat markedImg;
	    cv::cvtColor(img, markedImg, cv::COLOR_GRAY2BGR);
	    for (const auto &cor: corners) {
	        cor->Draw(markedImg);
	    }
	    mats.insert({"marked_image", markedImg});
	\end{lstlisting}
	\begin{lstlisting}[caption={\normf ORB角点检测代码}]
	    auto detector = cv::ORB::create(1000);
	    std::vector<cv::KeyPoint> keyPoints;
	    detector->detect(img, keyPoints);
	
	    for (const auto &point: keyPoints) {
	        corners.push_back(Corner::Create(point.pt));
	    }
	
	    // mark corners to source image
	    cv::Mat markedImg;
	    cv::cvtColor(img, markedImg, cv::COLOR_GRAY2BGR);
	    for (const auto &cor: corners) {
	        cor->Draw(markedImg);
	    }
	    mats.insert({"marked_image", markedImg});
	\end{lstlisting}
	\begin{lstlisting}[caption={\normf SIFT角点检测代码}]
	    auto detector = cv::SIFT::create(1000);
	    std::vector<cv::KeyPoint> keyPoints;
	    detector->detect(img, keyPoints);
	    for (const auto &point: keyPoints) {
	        corners.push_back(Corner::Create(point.pt));
	    }
	
	    // mark corners to source image
	    cv::Mat markedImg;
	    cv::cvtColor(img, markedImg, cv::COLOR_GRAY2BGR);
	    for (const auto &cor: corners) {
	        cor->Draw(markedImg);
	    }
	    mats.insert({"marked_image", markedImg});
	\end{lstlisting}
	\begin{lstlisting}[caption={\normf SURF角点检测代码}]
	    auto detector = cv::xfeatures2d::SURF::create(1000);
	    std::vector<cv::KeyPoint> keyPoints;
	    detector->detect(img, keyPoints);
	
	    for (const auto &point: keyPoints) {
	        corners.push_back(Corner::Create(point.pt));
	    }
	
	    // mark corners to source image
	    cv::Mat markedImg;
	    cv::cvtColor(img, markedImg, cv::COLOR_GRAY2BGR);
	    for (const auto &cor: corners) {
	        cor->Draw(markedImg);
	    }
	    mats.insert({"marked_image", markedImg});
	\end{lstlisting}
	\begin{lstlisting}[caption={\normf LSD1直线提取代码}]
	    auto lsd = cv::createLineSegmentDetector();
	    std::vector<cv::Vec4f> cvLines;
	    lsd->detect(img, cvLines);
	
	    // mark corners to source image
	    for (const auto &elem: cvLines) {
	        lines.push_back(Line::Create({elem[0], elem[1]}, {elem[2], elem[3]}));
	    }
	    cv::Mat markedImg;
	    cv::cvtColor(img, markedImg, cv::COLOR_GRAY2BGR);
	    for (const auto &elem: lines) {
	        elem->Draw(markedImg);
	    }
	    mats.insert({"marked_image", markedImg});
	\end{lstlisting}
	\begin{lstlisting}[caption={\normf LSD2直线提取代码}]
	    auto lsd = cv::line_descriptor::LSDDetector::createLSDDetector();
	    std::vector<cv::line_descriptor::KeyLine> cvLines;
	    lsd->detect(img, cvLines, 2, 1);
	
	    // mark corners to source image
	    for (const auto &elem: cvLines) {
	        lines.push_back(Line::Create({elem.startPointX, elem.startPointY},
	                                     {elem.endPointX, elem.ePointInOctaveY}));
	    }
	    cv::Mat markedImg;
	    cv::cvtColor(img, markedImg, cv::COLOR_GRAY2BGR);
	    for (const auto &elem: lines) {
	        elem->Draw(markedImg);
	    }
	    mats.insert({"marked_image", markedImg});
	\end{lstlisting}
\end{document}


\documentclass[12pt, twocolumn]{article}

% 引入相关的包
\usepackage{amsmath, listings, fontspec, geometry, graphicx, ctex, color, subfigure, amsfonts, amssymb}
\usepackage{multirow}
\usepackage[table,xcdraw]{xcolor}
\usepackage[ruled]{algorithm2e}
\usepackage[hidelinks]{hyperref}

		\usepackage{graphicx}
		\usepackage[most]{tcolorbox}
\hypersetup{
	colorlinks=true,
	linkcolor=red,
	citecolor=red,
}
\usepackage{booktabs}
\usepackage{multirow}
\usepackage{picins}

% 设定页面的尺寸和比例
\geometry{left = 1.5cm, right = 1.5cm, top = 1.5cm, bottom = 1.5cm}

% 设定两栏之间的间距
\setlength\columnsep{1cm}

% 设定字体,为代码的插入作准备
\newfontfamily\ubuntu{Ubuntu Mono}
\newfontfamily\consolas{Consolas}

% 头部信息
\title{\normf{编程:观测值逐次更新的扩展卡尔曼滤波器}}
\author{\normf 姓名:陈烁龙\;\;\;学号:2023202140019\;\;\;学院:测绘学院}
\date{\normf{\today}}

% 代码块的风格设定
\lstset{
	language=C++,
	basicstyle=\small\ubuntu,
	keywordstyle=\textbf,
	stringstyle=\itshape,
	commentstyle=\itshape,
	numberstyle=\scriptsize\ubuntu,
	showstringspaces=false,
	numbers=left,
	numbersep=8pt,
	tabsize=2,
	frame=single,
	framerule=1pt,
	columns=fullflexible,
	breaklines,
	frame=shadowbox, 
	backgroundcolor=\color[rgb]{0.97,0.97,0.97}
}

% 字体族的定义
% \fangsong \songti \heiti \kaishu
\newcommand\normf{\fangsong}
\newcommand\boldf{\heiti}
\newcommand\keywords[1]{\boldf{关键词:} \normf #1}

\newcommand\liehat[1]{\left[ #1 \right]_\times}
\newcommand\lievee[1]{\left[ #1 \right]^\vee}
\newcommand\liehatvee[1]{\left[ #1 \right]^\vee_\times}

\newcommand\mlcomment[1]{\iffalse #1 \fi}
%\newcommand\mlcomment[1]{ #1 }

\newcounter{problemname}
\newenvironment{problem}{\stepcounter{problemname}\par\noindent\normf\textbf{\textcolor[rgb]{1,0,0}{题目\arabic{problemname}.} }}{\leavevmode\\\par}
\newenvironment{solution}{\par\noindent\normf\textbf{解答: }}{\leavevmode\\\par}
\newenvironment{note}{\par\noindent\normf\textbf{题目\arabic{problemname}的注记: }}{\leavevmode\\\par}


\begin{document}
	\begin{titlepage}
	    \centering
	    \includegraphics[width=0.4\textwidth]{whu_red.png}\par\vspace{1cm}
	    \vspace{4cm}
	    {\huge\kaishu “机器视觉”讨论一\par 自动驾驶场景下的三维车辆识别与跟踪 \par}
	    \vspace{3cm}
	    {\Large\kaishu 
	    \begin{center}\begin{tabular}{l}
	    姓名:陈烁龙\\
	    学号:\bfseries 2023202140019\\
	    学院:测绘学院
	    \end{tabular}\end{center}
	     \par}
	    
	
	    \vfill
	
	% Bottom of the page
	    {\large\kaishu\bfseries \today\par}
	\end{titlepage}
		% 换页
 		\thispagestyle{empty}
		\clearpage
		
		% 插入目录、图、表并换页
		\pagenumbering{roman}
		\tableofcontents
		\newpage
		\listoffigures
		% 罗马字母形式的页码
		
		\clearpage
		% 从该页开始计数
		\setcounter{page}{1}
		% 阿拉伯数字形式的页码
		\pagenumbering{arabic}
	
	\section{\normf{研究背景与现状}}
	\normf
	物体检测是一种计算机视觉技术,它可以在图像或视频中识别和定位感兴趣的目标,并给出它们的类别和位置信息。
	物体检测车载着诸多的应用场景,比如视频监控、自动驾驶、人脸识别等。
	
	物体检测可以分为二维物体检测和三维物体检测两种不同形式。二维物体检测是在图像平面上进行的,它可以识别和定位图像中的目标,但无法提供目标的深度信息,如图\ref{fig:物体检测}上图所示。三维物体检测则更进一步,可以在识别和定位目标的同时,提供目标的深度信息,如图\ref{fig:物体检测}下图所示。因此,三维物体检测比二维物体检测更适合需要精确距离信息的应用场景,例如自动驾驶、机器人导航等。但是,三维物体检测需要使用更多的传感器和计算资源,并且通常比二维物体检测更具挑战性。
	\begin{figure}[h]
		\centering
		\includegraphics[width=\linewidth,trim= 0 0 0 125, clip]{img/2d.png}
		
		\vspace{2 mm}
		
		\includegraphics[width=\linewidth]{img/3d.png}
		\caption{\normf 二维物体检测(上)和三维物体检测(下)}
		\label{fig:物体检测}
	\end{figure}
	
	当前三维物体检测主要可以分为以下几类:
	\begin{enumerate}
	\item 单目3D目标检测:以单幅图像作为输入,输出物体的三维边界框。由于图像不包含深度信息,因此这个问题是病态的;
	
	\item 多视角3D目标检测:主要研究方向是密集算法,这些方法使用稠密的特征向量进行特征融合、目标预测。而其中,基于BEV的方法占据了一大部分。
	\end{enumerate}
	单目3D目标检测的典型代表工作有FCOS3D \cite{wang2021fcos3d} (见图\ref{fig:FCOS3D三维物体检测算法})和SMOKE \cite{liu2020smoke},他们都是基于原来的2D物体检测进行扩展,使用全卷积网络直接回归每个对象的深度。与此不同,OFT (见图\ref{fig:OFT三维物体检测算法})和CaDDN \cite{reading2021categorical} 则将2D影像转到BEV(Bird-eye-view)空间,然后基于探测到的BEV特征进行3D目标探测。还有的一些方法则基于深度估计结果,将2D图像转换为3D伪点云,用LiDAR的那一套来进行3D目标检测。

	\begin{figure}[h]
		\centering
		\includegraphics[width=\linewidth,trim= 0 90 0 0, clip]{img/FCOS3D.png}
		\caption{\normf FCOS3D三维物体检测算法}
		\label{fig:FCOS3D三维物体检测算法}
	\end{figure}
	\begin{figure}[h]
		\centering
		\includegraphics[width=\linewidth,trim= 0 0 0 0, clip]{img/OFT.png}
		\caption{\normf OFT三维物体检测算法}
		\label{fig:OFT三维物体检测算法}
	\end{figure}
	
	多视角3D目标检测算法中,BEVFormer \cite{li2022bevformer} (见图\ref{fig:BEVFormer三维物体检测算法})采用可变注意力(deformable attention)完成BEV特征生成和稠密时空特征的融合。而BEVDet \cite{huang2021bevdet},BEVDepth \cite{li2023bevdepth},BEVStereo \cite{li2023bevstereo}则使用lift-splat操作进行视图转换来实现物体检测。DETR3D \cite{wang2022detr3d} (见图\ref{fig:DETR3D三维物体检测算法})不同于上面的稠密方法,其基于稀疏参考点进行特征采样和融合,大大提高了物体识别的效率。在DETR3D的基础上,Graph DETR3D \cite{chen2022graph} 引入图网络来实现更好的空间特征融合。
	\begin{figure}[h]
		\centering
		\includegraphics[width=\linewidth,trim= 0 120 0 0, clip]{img/BEVFormer.png}
		\caption{\normf BEVFormer三维物体检测算法}
		\label{fig:BEVFormer三维物体检测算法}
	\end{figure}
	\begin{figure}[h]
		\centering
		\includegraphics[width=\linewidth,trim= 0 200 0 0, clip]{img/DETR3D.jpg}
		\caption{\normf DETR3D三维物体检测算法}
		\label{fig:DETR3D三维物体检测算法}
	\end{figure}
	
	然而,无论是基于BEV的方法,还是基于稀疏的方法,都存在不足之处。基于BEV的方法图像到BEV的视角转换需要密集的特征采样或重排,这对于低成本的边缘设备部署来说是复杂且计算昂贵的。同时,最大感知范围受BEV特征图大小的限制,难以在感知范围、效率和精度之间进行权衡。另外BEV特征的高度维度被压缩,纹理线索丢失。
	在基于稀疏的方法中,DETR3D对每一个检测目标框,只有一个3D参考的样本特征,是的可靠性存在不足。SRCN3D \cite{shi2022srcn3d} 采样多视图特征,但效率不高,不能精确对齐来自不同视图的特征点。同时注意到,现有的稀疏三维检测方法没有充分利用丰富的时间信息,与目前最先进的基于BEV的方法相比,在性能上存在很大差距。
	
	\section{\normf{方法设计}}
	\subsection{\normf{方案设想}}
	基于上文的分析,我们提出了如下的方法,以权衡两类方法的利弊。针对当前基于稀疏的方法中时间信息利用程度不高的现状,我们仍然使用基于稀疏的方法,但是:
	\begin{enumerate}
	\item 对于每个3D锚点,我们分配多个4D关键点(多了一个时间维),然后将其投影到多视图/尺度/时间戳图像特征中以采样相应的特征;(区别于DETR3D:对每一个检测目标框,只有一个3D参考的样本特征);
	
	\item 将不同视角/尺度、不同时间戳和不同关键点的采样特征分层融合,生成高质量的实例特征;
	
	\item 引入实例级深度重权重模块,缓解三维到二维投影中的不确定性问题。
	\end{enumerate}
	
	该方法由于是基于稀疏的方法,能够高效实现3D检测。同时使用多个4D关键点而非单个3D关键点,关键点数目更多,维度信息更大,能够有效实现3D检测。另外,该方法无需依赖密集的视图变换和全局关注,计算量小,对设备部署更加友好。
	
	\subsection{\normf{关键技术和可行性}}
	对于4D特征点的生成:为获取具有代表性的实例特征,最关键的环节在于特征点的选取。为此,对于每个锚点,可以将4D关键点分为固定特征点和可学习特征点。将固定的特征点直接放在锚盒的立体中心和六面中心上,可学习的特征点随着不同的实例特征而变化,从而可以运用神经网络找到每个实例的最具代表性的特征。
	
	在稀疏采样方面,为充分利用特征信息,生成高质量的实例特征,将采样特征进行分层融合。具体来说,对于每个特征点,首先用特征点生成中预测的权重融合不同视图和尺度的特征。时间戳维度在序列中的线性层融合。对于每个锚点实例,融合多点特征以生成实例特征。

	另外,通过引入一个深度重权重,来缓解在3D到2D的变换中不同的3D点可以对应于相同的2D坐标这个问题。对于每个3D锚点,可以计算它在不同视图下的投影深度,并根据深度的分布和方差来给每个视图分配一个权重,用于重新加权实例特征。

	\section{\normf{结论与展望}}
	由于该方法是基于稀疏的方法,所以相较于稠密BEV方法,更加高效。同时,在构建特征点的时候,考虑了时间信息,所以精度更高。最后,引入的引入了实例层次的深度重权重模块,能够缓解三维到二维投影中的不确定性问题。

	当然,该方案也存在一些待改进的地方。比如,可在深度重加权模块里面使用多视角视图中的双目视图信息,进一步改善深度信息的可靠性。在训练模型中,将相机的参数也纳入进来,来改善3D物体识别的泛化能力。
	
	\bibliographystyle{IEEEtran}
	\bibliography{reference}
	
	\section*{ACKNOWLEDGMENT}
	\begin{tcolorbox}[colback=white,colframe=white!70!black,title={\bfseries Author Information}]
	\par\noindent
		\parbox[t]{\linewidth}{
	 \noindent\parpic{\includegraphics[height=2in,width=1in,clip,keepaspectratio]{ShuolongChen_grey.jpg}}
	 \noindent{\bfseries Shuolong Chen}\emph{
	 received the B.S. degree in geodesy and geomatics engineering from Wuhan University, Wuhan China, in 2023.
	 He is currently a master candidate at the school of Geodesy and Geomatics, Wuhan University. His area of research currently focuses on integrated navigation systems and multi-sensor fusion.
	 Contact him via e-mail: shlchen@whu.edu.cn.
	 }}
	\end{tcolorbox}
		
		
\end{document}


\documentclass[12pt, onecolumn]{article}

% 引入相关的包
\usepackage{amsmath, listings, fontspec, geometry, graphicx, ctex, color, subfigure, amsfonts, amssymb}
\usepackage{multirow}
\usepackage[table,xcdraw]{xcolor}
\usepackage[ruled]{algorithm2e}
\usepackage[hidelinks]{hyperref}

		\usepackage{graphicx}
		\usepackage[most]{tcolorbox}
\hypersetup{
	colorlinks=true,
	linkcolor=red,
	citecolor=red,
}
\usepackage{booktabs}
\usepackage{multirow}
\usepackage{picins}

% 设定页面的尺寸和比例
\geometry{left = 1.5cm, right = 1.5cm, top = 1.5cm, bottom = 1.5cm}

% 设定两栏之间的间距
\setlength\columnsep{1cm}

% 设定字体,为代码的插入作准备
\newfontfamily\ubuntu{Ubuntu Mono}
\newfontfamily\consolas{Consolas}

% 头部信息
\title{\normf{编程:观测值逐次更新的扩展卡尔曼滤波器}}
\author{\normf 姓名:陈烁龙\;\;\;学号:2023202140019\;\;\;学院:测绘学院}
\date{\normf{\today}}

% 代码块的风格设定
\lstset{
	language=C++,
	basicstyle=\small\ubuntu,
	keywordstyle=\textbf,
	stringstyle=\itshape,
	commentstyle=\itshape,
	numberstyle=\scriptsize\ubuntu,
	showstringspaces=false,
	numbers=left,
	numbersep=8pt,
	tabsize=2,
	frame=single,
	framerule=1pt,
	columns=fullflexible,
	breaklines,
	frame=shadowbox, 
	backgroundcolor=\color[rgb]{0.97,0.97,0.97}
}

% 字体族的定义
% \fangsong \songti \heiti \kaishu
\newcommand\normf{\fangsong}
\newcommand\boldf{\heiti}
\newcommand\keywords[1]{\bfseries{关键词:} \normf #1}

\newcommand\liehat[1]{\left[ #1 \right]_\times}
\newcommand\lievee[1]{\left[ #1 \right]^\vee}
\newcommand\liehatvee[1]{\left[ #1 \right]^\vee_\times}

\newcommand\mlcomment[1]{\iffalse #1 \fi}
%\newcommand\mlcomment[1]{ #1 }

\newcounter{problemname}
\newenvironment{problem}{\stepcounter{problemname}\par\noindent\normf\textbf{\textcolor[rgb]{1,0,0}{题目\arabic{problemname}.} }}{\leavevmode\\\par}
\newenvironment{solution}{\par\noindent\normf\textbf{解答: }}{\leavevmode\\\par}
\newenvironment{note}{\par\noindent\normf\textbf{题目\arabic{problemname}的注记: }}{\leavevmode\\\par}


\begin{document}
	\begin{titlepage}
	    \centering
	    \includegraphics[width=0.4\textwidth]{whu_red.png}\par\vspace{1cm}
	    \vspace{4cm}
	    {\huge\kaishu\bfseries “机器视觉”编程作业\par 图像去畸变 \par}
	    \vspace{3cm}
	    {\Large\kaishu 
	    \begin{center}\begin{tabular}{l}
	    姓名:陈烁龙\\
	    学号:\bfseries 2023202140019\\
	    学院:测绘学院
	    \end{tabular}\end{center}
	     \par}
	    
	
	    \vfill
	
	% Bottom of the page
	    {\large\kaishu\bfseries \today\par}
	\end{titlepage}
		% 换页
 		\thispagestyle{empty}
		\clearpage
		
		% 插入目录、图、表并换页
		\pagenumbering{roman}
		\tableofcontents
		\newpage
		\listoffigures
%		\newpage
%		\listoftables
		% 罗马字母形式的页码
		
		\clearpage
		% 从该页开始计数
		\setcounter{page}{1}
		% 阿拉伯数字形式的页码
		\pagenumbering{arabic}
	
	
	
	\section{\normf{角点检测}}
	\normf\bfseries
	
	\subsection{\normf{相机模型}}
	设提取到的角点为序列$\mathcal{P}=\{\boldsymbol{p}_{i,j}\mid i\in[0,r-1],j\in[0,c-1]\}$。在棋盘格上设置世界坐标系$\{w\}$,其$x$轴向下,$y$轴向右,$z$轴向外,坐标原点与左上角的角点重合。由于角点在同一平面上,且间距相等,有:
	\begin{equation}
	{^w\boldsymbol{p}_{i,j}}(z)=0
	\end{equation}
	和
	\begin{equation}
	{^w\boldsymbol{p}_{i+1,j+1}}={^w\boldsymbol{p}_{i,j}}+\begin{pmatrix}
	l&l&0
	\end{pmatrix}^\top
	\end{equation}
	其中$l$为相邻两个角点之间的间距(横向和纵向间距相等)。
	
	设相机坐标系为$\{c\}$,则角点${\boldsymbol{p}_{i,j}}$在相机坐标系下的位置为:
	\begin{equation}
	{^c\boldsymbol{p}_{i,j}}={^{c}_{w}\boldsymbol{R}}\cdot{^w\boldsymbol{p}_{i,j}}+{^c\boldsymbol{p}_{w}}
	\end{equation}
	将该点规化到归一化像素平面上,得到:
	\begin{equation}
	\begin{pmatrix}
	x_{i,j}\\y_{i,j}
	\end{pmatrix}=\begin{pmatrix}
	\begin{aligned}
	\frac{{^c\boldsymbol{p}_{i,j}}(x)}{{^c\boldsymbol{p}_{i,j}}(z)}
	\end{aligned}
	&
	\begin{aligned}		\frac{{^c\boldsymbol{p}_{i,j}}(y)}{{^c\boldsymbol{p}_{i,j}}(z)}
	\end{aligned}
	\end{pmatrix}^\top
	\end{equation}
	进行加畸变操作,得到:
	\begin{equation}
	\begin{cases}
	\begin{aligned}
	x_{i,j}^d&=x_{i,j}\times\left( 1+k_1\times r_{i,j}^2+k_2\times r_{i,j}^4\right) 
	\\
	y_{i,j}^d&=y_{i,j}\times\left( 1+k_1\times r_{i,j}^2+k_2\times r_{i,j}^4\right) 
	\end{aligned}
	\end{cases}
	\end{equation}
	其中$r_{i,j}=\sqrt{x_{i,j}^2+y_{i,j}^2}$,$k_1$和$k_2$为径向畸变参数。
	最后投到像平面上,有:
	\begin{equation}
	\begin{pmatrix}
	u_{i,j}\\v_{i,j}\\1
	\end{pmatrix}=
	\begin{pmatrix}
	f_x&0&c_x\\
	0&f_y&c_y\\
	0&0&1
	\end{pmatrix}
	\begin{pmatrix}
		x_{i,j}^d\\y_{i,j}^d\\1
	\end{pmatrix}
	\end{equation}
	
	\subsection{\normf{参数求解}}
	在不考虑畸变模型的情况下,有:
	\begin{equation}
	\begin{pmatrix}
	u_{i,j}\\v_{i,j}\\1
	\end{pmatrix}=\frac{1}{{^c\boldsymbol{p}_{i,j}}(z)}\cdot
	\begin{pmatrix}
	f_x&0&c_x\\
	0&f_y&c_y\\
	0&0&1
	\end{pmatrix}\begin{pmatrix}
	{^{c}_{w}\boldsymbol{R}(1)}&
	{^{c}_{w}\boldsymbol{R}(2)}&
	{^c\boldsymbol{p}_{w}}
	\end{pmatrix}
	\begin{pmatrix}
	{^w\boldsymbol{p}_{i,j}}(x)\\{^w\boldsymbol{p}_{i,j}}(y)\\1
	\end{pmatrix}
	\end{equation}
	在不考虑畸变模型的情况下,有:
	\begin{equation}
	\begin{pmatrix}
	u_{i,j}-c_x\\v_{i,j}-c_y\\1
	\end{pmatrix}=\frac{f}{{^c\boldsymbol{p}_{i,j}}(z)}\cdot
	\begin{pmatrix}
	{^{c}_{w}\boldsymbol{R}(1)}&
	{^{c}_{w}\boldsymbol{R}(2)}&
	{^c\boldsymbol{p}_{w}}
	\end{pmatrix}
	\begin{pmatrix}
	{^w\boldsymbol{p}_{i,j}}(x)\\{^w\boldsymbol{p}_{i,j}}(y)\\1
	\end{pmatrix}
	\end{equation}
	将上式中的映射矩阵记为:
	\begin{equation}
	\begin{pmatrix}
		{^{c}_{w}\boldsymbol{R}(1)}&
		{^{c}_{w}\boldsymbol{R}(2)}&
		{^c\boldsymbol{p}_{w}}
		\end{pmatrix}=\begin{pmatrix}
		h_{11}&h_{12}&h_{13}\\
		h_{21}&h_{22}&h_{23}\\
		h_{31}&h_{32}&h_{33}
		\end{pmatrix}
	\end{equation}
	则有:
	\begin{equation}
	{^c\boldsymbol{p}_{i,j}}(z)=h_{31}\times{^w\boldsymbol{p}_{i,j}}(x)+h_{32}\times{^w\boldsymbol{p}_{i,j}}(y)+h_{33}
	\end{equation}
	同时存在约束:
	\begin{equation}
	{^{c}_{w}\boldsymbol{R}(1)}^\top\cdot{^{c}_{w}\boldsymbol{R}(1)}=0
	\qquad
	\Vert{^{c}_{w}\boldsymbol{R}(1)}\Vert_2=1
	\end{equation}
	待求解参数:
	\begin{equation}
	\boldsymbol{x}^\prime=\{f,{^{c}_{w}\boldsymbol{R}(1)}^\top,{^{c}_{w}\boldsymbol{R}(1)}^\top,{^c\boldsymbol{p}_{w}}^\top\}
	\end{equation}
	
	\newpage
	\bibliographystyle{IEEEtran}
	\bibliography{reference}
		
	\newpage
	\section*{ACKNOWLEDGMENT}
	\begin{tcolorbox}[colback=white,colframe=white!70!black,title={\bfseries Author Information}]
	\par\noindent
		\parbox[t]{\linewidth}{
	 \noindent\parpic{\includegraphics[height=2in,width=1in,clip,keepaspectratio]{ShuolongChen_grey.jpg}}
	 \noindent{\bfseries Shuolong Chen}\emph{
	 received the B.S. degree in geodesy and geomatics engineering from Wuhan University, Wuhan China, in 2023.
	 He is currently a master candidate at the school of Geodesy and Geomatics, Wuhan University. His area of research currently focuses on integrated navigation systems and multi-sensor fusion.
	 Contact him via e-mail: shlchen@whu.edu.cn.
	 }}
	\end{tcolorbox}
		
		
\end{document}
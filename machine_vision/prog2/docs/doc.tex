\documentclass[12pt, onecolumn]{article}

% 引入相关的包
\usepackage{amsmath, listings, fontspec, geometry, graphicx, ctex, color, subfigure, amsfonts, amssymb}
\usepackage{multirow}
\usepackage[table,xcdraw]{xcolor}
\usepackage[ruled]{algorithm2e}
\usepackage[hidelinks]{hyperref}

		\usepackage{graphicx}
		\usepackage[most]{tcolorbox}
\hypersetup{
	colorlinks=true,
	linkcolor=red,
	citecolor=red,
}
\usepackage{booktabs}
\usepackage{multirow}
\usepackage{picins}

% 设定页面的尺寸和比例
\geometry{left = 1.5cm, right = 1.5cm, top = 1.5cm, bottom = 1.5cm}

% 设定两栏之间的间距
\setlength\columnsep{1cm}

% 设定字体,为代码的插入作准备
\newfontfamily\ubuntu{Ubuntu Mono}
\newfontfamily\consolas{Consolas}

% 头部信息
\title{\normf{编程:观测值逐次更新的扩展卡尔曼滤波器}}
\author{\normf 姓名:陈烁龙\;\;\;学号:2023202140019\;\;\;学院:测绘学院}
\date{\normf{\today}}

% 代码块的风格设定
\lstset{
	language=C++,
	basicstyle=\small\ubuntu,
	keywordstyle=\textbf,
	stringstyle=\itshape,
	commentstyle=\itshape,
	numberstyle=\scriptsize\ubuntu,
	showstringspaces=false,
	numbers=left,
	numbersep=8pt,
	tabsize=2,
	frame=single,
	framerule=1pt,
	columns=fullflexible,
	breaklines,
	frame=shadowbox, 
	backgroundcolor=\color[rgb]{0.97,0.97,0.97}
}

% 字体族的定义
% \fangsong \songti \heiti \kaishu
\newcommand\normf{\fangsong}
\newcommand\boldf{\heiti}
\newcommand\keywords[1]{\bfseries{关键词:} \normf #1}

\newcommand\liehat[1]{\left[ #1 \right]_\times}
\newcommand\lievee[1]{\left[ #1 \right]^\vee}
\newcommand\liehatvee[1]{\left[ #1 \right]^\vee_\times}

\newcommand\mlcomment[1]{\iffalse #1 \fi}
%\newcommand\mlcomment[1]{ #1 }

\newcounter{problemname}
\newenvironment{problem}{\stepcounter{problemname}\par\noindent\normf\textbf{\textcolor[rgb]{1,0,0}{题目\arabic{problemname}.} }}{\leavevmode\\\par}
\newenvironment{solution}{\par\noindent\normf\textbf{解答: }}{\leavevmode\\\par}
\newenvironment{note}{\par\noindent\normf\textbf{题目\arabic{problemname}的注记: }}{\leavevmode\\\par}


\begin{document}
	\begin{titlepage}
	    \centering
	    \includegraphics[width=0.4\textwidth]{whu_red.png}\par\vspace{1cm}
	    \vspace{4cm}
	    {\huge\kaishu\bfseries “机器视觉”编程作业\par 图像去畸变 \par}
	    \vspace{3cm}
	    {\Large\kaishu 
	    \begin{center}\begin{tabular}{l}
	    姓名:陈烁龙\\
	    学号:\bfseries 2023202140019\\
	    学院:测绘学院
	    \end{tabular}\end{center}
	     \par}
	    
	
	    \vfill
	
	% Bottom of the page
	    {\large\kaishu\bfseries \today\par}
	\end{titlepage}
		% 换页
 		\thispagestyle{empty}
		\clearpage
		
		% 插入目录、图、表并换页
		\pagenumbering{roman}
		\tableofcontents
		\newpage
		\listoffigures
%		\newpage
%		\listoftables
		% 罗马字母形式的页码
		
		\clearpage
		% 从该页开始计数
		\setcounter{page}{1}
		% 阿拉伯数字形式的页码
		\pagenumbering{arabic}
	
	
	\section{\normf{针孔相机模型}}
	\normf\bfseries
	
	设提取到的角点为序列$\mathcal{P}=\{\boldsymbol{p}_{i,j}\mid i\in[0,r-1],j\in[0,c-1]\}$。在棋盘格上设置世界坐标系$\{w\}$,其$x$轴向下,$y$轴向右,$z$轴向外,坐标原点与左上角的角点重合。由于角点在同一平面上,且间距相等,有:
	\begin{equation}
	{^w\boldsymbol{p}_{i,j}}(z)=0
	\end{equation}
	和
	\begin{equation}
	{^w\boldsymbol{p}_{i+1,j+1}}={^w\boldsymbol{p}_{i,j}}+\begin{pmatrix}
	l&l&0
	\end{pmatrix}^\top
	\end{equation}
	其中$l$为相邻两个角点之间的间距(横向和纵向间距相等)。
	
	设相机坐标系为$\{c\}$,则角点${\boldsymbol{p}_{i,j}}$在相机坐标系下的位置为:
	\begin{equation}
	{^c\boldsymbol{p}_{i,j}}={^{c}_{w}\boldsymbol{R}}\cdot{^w\boldsymbol{p}_{i,j}}+{^c\boldsymbol{p}_{w}}
	\end{equation}
	将该点规化到归一化像素平面上,得到:
	\begin{equation}
	\begin{pmatrix}
	x_{i,j}\\y_{i,j}
	\end{pmatrix}=\begin{pmatrix}
	\begin{aligned}
	\frac{{^c\boldsymbol{p}_{i,j}}(x)}{{^c\boldsymbol{p}_{i,j}}(z)}
	\end{aligned}
	&
	\begin{aligned}		\frac{{^c\boldsymbol{p}_{i,j}}(y)}{{^c\boldsymbol{p}_{i,j}}(z)}
	\end{aligned}
	\end{pmatrix}^\top
	\end{equation}
	进行加畸变操作(在本次实验中,只考虑径向畸变),得到:
	\begin{equation}
	\begin{cases}
	\begin{aligned}
	x_{i,j}^d&=x_{i,j}\times\left( 1+k_1\times r_{i,j}^2+k_2\times r_{i,j}^4\right) 
	\\
	y_{i,j}^d&=y_{i,j}\times\left( 1+k_1\times r_{i,j}^2+k_2\times r_{i,j}^4\right) 
	\end{aligned}
	\end{cases}
	\end{equation}
	其中$r_{i,j}=\sqrt{x_{i,j}^2+y_{i,j}^2}$,$k_1$和$k_2$为径向畸变参数。
	最后投到像平面上,有:
	\begin{equation}
	\begin{pmatrix}
	u_{i,j}\\v_{i,j}\\1
	\end{pmatrix}=
	\begin{pmatrix}
	f_x&0&c_x\\
	0&f_y&c_y\\
	0&0&1
	\end{pmatrix}
	\begin{pmatrix}
		x_{i,j}^d\\y_{i,j}^d\\1
	\end{pmatrix}
	\end{equation}
	
	\section{\normf{基于霍夫变换的径向畸变校正}}
	本次的畸变去除算法采用的是基于霍夫变换的径向畸变去除算法,该方法由Cucchiara等人\cite{cucchiara2003hough}提出。
	该方法最大的假设为径向变形具有扭曲不通过光学中心的直线的特,其进行图像去畸变的两个步骤为:
	\begin{enumerate}
	\item 自动或半自动识别候选图像中的候选线条;
	
	\item 迭代变化径向畸变参数,以便
		使候选线的直线度最大化;
	\end{enumerate}
	由于检测线条和测量其直线度的最著名、最有效的方法是 Hough 变换 (HT) \cite{illingworth1988survey},因此该方法称为 HTRDC,即表示基于 HT变换 的径向畸变校正,其流图如图\ref{fig:htrdc}所示。
	
	\begin{figure}[h]
		\centering
		\includegraphics[width=0.8\linewidth]{images/htrdc.png}
		\caption{\normf 基于霍夫变换的径向畸变校正算法流程图}
		\label{fig:htrdc}
	\end{figure}
	
	具体来说,该方法首先用Canny 边缘检测\cite{canny1986computational}提取 ROI,并利用滞后非最大值抑制方法提取 "强 "边缘。在边缘检测之后,以多分辨率的方式对 ROI
	变形:从初始的 最小k值和 最大k值开始,将范围划分为 分 n 步进行,每一步 i 的失真参数为:
	\begin{equation}
	k^i=k^{min}+i\times\frac{k^{max}-k^{min}}{n}
	\end{equation}
	然后,计算直线的 Hough 变换。Hough 空间的定义如下:
	\begin{equation}
	HT^i_j=\{h^i_j(\rho,\theta),\rho\in\left[ -R,+R\right],\theta\in\left[ 0,\pi\right]  \}
	\end{equation}
	其中$R=\sqrt{w^2_{ROI}+h^2_{ROI}}$,$w$和$h$为ROI的宽度和高度。
	基于标准的 HT参数线方程,对其进行修改,使得权重与 ROI(x,y) 值成正比,如算法\ref{alg:alg}所示。
	\begin{algorithm}
	\caption{\normf 修改后的HT参数线方程}
	\label{alg:alg}
	\ForEach{$(x,y)\in ROI^i_j$}{
	\ForEach{$\theta=a\cdot\delta\Delta\theta$}{
		$\rho\gets x\cdot\cos\theta+y\cdot\sin\theta$
		
		$HT^i_j( \rho,\theta)\gets HT^i_j( \rho,\theta)+ROI^i_j(x,y) $
	}
	}
	\end{algorithm}
	需要注意的是,算法\ref{alg:alg}中的量化值$(\Delta \rho,\Delta\theta)$的选择,即 Hough 空间的分辨率,必须与 Hough 空间的分辨率相关联。
	
	接着使用高斯滤波器对 HT 空间进行后处理,以平滑由于图像和 HT 空间的量化而可能产生的振荡。
	而后对未扭曲 ROI 的平直度进行评估。评估是通过找到 HT 空间的全局最大值来进行的,也即:
	\begin{equation}
	M^i_j=\max_{(\rho,\theta)}\left(\bar{HT^i_j}( \rho,\theta) \right) 
	\end{equation}
	
	至此,完成了一次迭代。在下一次迭代时,畸变参数的最小值选取为:
	\begin{equation}
	k^{min}_{j+1}=\max\left(0,MK_j-\frac{k^{max}_{j}-k^{min}_{j}}{n} \right) 
	\end{equation}
	最大值选取为:
	\begin{equation}
	k^{max}_{j+1}=\min\left(1,MK_j+\frac{k^{max}_{j}-k^{min}_{j}}{n} \right) 
	\end{equation}
	而后重复上述的过程。当 新计算出的 k值范围小于固定阈值时,递归结束,此时$MK_j$被选为最佳 k值。
	
	另外,需要注意的是,k的计算值取决于
	最初选定的 ROI。如果 ROI 中存在较强的边缘,则改变径向变形的效果会更加明显,与之对应的,该方法效果会更好。
	
	\section{\normf 实验结果}
	本次实验选取了4张存在径向畸变的影像,实验的结果如图\ref{fig:图像1的实验结果}、\ref{fig:图像2的实验结果}、\ref{fig:图像3的实验结果}、\ref{fig:图像4的实验结果}所示。
	可以看到,图片中的大部分畸变都已经被去除。具体来说,图\ref{fig:图像1的实验结果}、\ref{fig:图像2的实验结果}、\ref{fig:图像3的实验结果}的去畸变效果比较突出,而图\ref{fig:图像4的实验结果}的去畸变效果存在不足之处,图片中的隐式直线没有被完全对齐到一起,原因在于该图像的径向畸变程度明显比其他图像大,且边缘特征非常稀疏(虽然角点很多),导致算法的“样本”过少,参数估计(径向畸变参数k)的准确地降低。
	
	HTRDC 方法在很大程度上依赖于径向畸变的概念和定义。
	由于径向畸变
	与时间、摄像机运动或场景无关,因此可以在处理开始时找到最佳的变形参数
	参数,而且每个摄像机只需使用一次。唯一的要求是场景中包含
	唯一的要求是场景中包含一些对比强烈的直线。在很多场景下都能使用该方法进行径向畸变的去除,如室内监控、家庭机器人、人体 3D 建模、基于视觉的交通控制和管理
	车视频中的车道检测等。
	
	\begin{figure}[h!]
	\centering
	\subfigure[\normf{畸变图像}]{
		\centering
		\includegraphics[width=0.4\linewidth]{images/test.jpg}
	}
	\subfigure[\normf{去畸变图像}]{
		\centering
			\includegraphics[width=0.48\linewidth]{results/test.jpg}
		}
	\caption{\normf 图像1的实验结果}
	\label{fig:图像1的实验结果}
	\end{figure}
	\begin{figure}[h!]
	\centering
	\subfigure[\normf{畸变图像}]{
		\centering
		\includegraphics[width=0.425\linewidth]{images/test2.png}
	}
	\subfigure[\normf{去畸变图像}]{
		\centering
			\includegraphics[width=0.45\linewidth]{results/test2.png}
		}
	\caption{\normf 图像2的实验结果}
	\label{fig:图像2的实验结果}
	\end{figure}
	\begin{figure}[h!]
	\centering
	\subfigure[\normf{畸变图像}]{
		\centering
		\includegraphics[width=0.41\linewidth]{images/test3.jpg}
	}
	\subfigure[\normf{去畸变图像}]{
		\centering
			\includegraphics[width=0.45\linewidth]{results/test3.jpg}
		}
	\caption{\normf 图像3的实验结果}
	\label{fig:图像3的实验结果}
	\end{figure}
	\begin{figure}[h!]
	\centering
	\subfigure[\normf{畸变图像}]{
		\centering
		\includegraphics[width=0.45\linewidth]{images/board.jpg}
	}
	\subfigure[\normf{去畸变图像}]{
		\centering
			\includegraphics[width=0.45\linewidth]{results/board2-clip.jpg}
		}
	\caption{\normf 图像4的实验结果}
	\label{fig:图像4的实验结果}
	\end{figure}
	
	\section{\normf 总结}
	该种基于霍夫变换的去畸变方法是一种稳健的
	(半)自动校正图像和视频中的径向透镜畸变的方法。
	该方法基于 Hough
	变换,其特点是也适用于未知摄像机的视频,因此无法
	先验校准。因此,该方法依赖于
	假设,即纯径向畸变会将现实世界中的直线
	为投影为曲线。畸变参数最佳值的计算
	失真参数的最佳值是通过多分辨率
	的方式进行。该方法的精度取决于多分辨率的尺度
	多分辨率和 Hough 空间的分辨率。
	
	\section{\normf 附录}
	表\ref{tab:运行环境}展示了本次编程的运行环境。
	% Please add the following required packages to your document preamble:
	% \usepackage{multirow}
	\begin{table}[h]
	\centering
	\caption{\normf 运行环境}
	\label{tab:运行环境}
	\begin{tabular}{ccc|cl}
	\midrule[1pt]
	\multicolumn{1}{c|}{\multirow{2}{*}{Config.}} & OS name & Ubuntu 20.04.6 LTS & Processor          & 12th Gen Intel® Core™ i9-12900H × 20         \\ \cmidrule{2-5} 
	\multicolumn{1}{c|}{}                         & OS Type & 64-bit             & Graphics           & Mesa Intel® Graphics (ADL GT2)               \\ \midrule[1pt]\midrule[1pt]
	\multicolumn{3}{c|}{Library}                                                 & \multicolumn{2}{c}{Link}                                          \\ \midrule
	\multicolumn{3}{c|}{OpenCV}                                                  & \multicolumn{2}{l}{https://github.com/opencv/opencv.git}          \\
	\multicolumn{3}{c|}{OpenCV Contrib}                                          & \multicolumn{2}{l}{https://github.com/opencv/opencv\_contrib.git} \\ \midrule[1pt]
	\end{tabular}
	\end{table}
	
	\begin{lstlisting}[caption={\normf 标定算法实现}]
def sfrs_calibrate(img):
        rsz = cv2.resize(img, (0, 0), fx=.6, fy=.6)
        k = conf.CAL_START_COEFF # modify k factor according to image size
        res = {}
        add_k = conf.CAL_COEFF_INC_FACTOR # modify add_k factor according to image size
        for i in range(conf.CAL_EPOCH_NUMB):
            # apply distortion with the current coeff k
            c, _, _, _, _ = apply_distortion(rsz, k)
            # get straight lines in the new image using hough
            gray = cv2.cvtColor(c, cv2.COLOR_BGR2GRAY)
            edges = cv2.Canny(gray, 20, 110)
            lines = cv2.HoughLines(edges, 1, np.pi / 180, 160)
            # evaluating distortion using number of lines
            if lines is not None:
                res[k] = len(lines)
            else:
                res[k] = 0
            print("epoch: ", i, "k: ", float(k), "score: ", res[k])
            k += add_k
        best = max(res, key=res.get)
        print("the winner is: ")
        print("epoch: ", float(best), "score: ", res[best])
        if float(best) != 0:
            output, _, _, _, _ = apply_distortion(rsz, float(best))
        else:
            output = rsz
        return output
	\end{lstlisting}
	
	\begin{lstlisting}[caption={\normf 基于标定结果进行图像去畸变}]
def apply_distortion(img, k1):
    width = img.shape[1]
    height = img.shape[0]
    corr_x, corr_y = new_axes(0, 0, width // 2, height // 2, k1)
    maxxu = int(corr_x)
    maxyu = int(corr_y)
    minytop = 0
    minxlef = 0
    minybot = 2 * maxyu
    minxrig = 2 * maxxu
    out_mat = np.zeros((2 * maxyu, 2 * maxxu, len(img.shape)), dtype="uint8")
    for y_iter in range(2 * maxyu):
        for x_iter in range(2 * maxxu):
            xd = - maxxu + x_iter
            yd = - maxyu + y_iter
            if (xd == 0) and (yd == 0):
                corr_x, corr_y = 0, 0
            else:
                corr_x, corr_y = inv_axes(0, 0, xd, yd, k1)
            matxu = corr_x + width // 2
            matyu = corr_y + height // 2
            if len(img.shape) == 3:
                if matxu >= width:
                    out_mat[y_iter, x_iter, :] = 0
                elif matyu >= height:
                    out_mat[y_iter, x_iter, :] = 0
                elif matxu <= 0:
                    out_mat[y_iter, x_iter, :] = 0
                elif matyu <= 0:
                    out_mat[y_iter, x_iter, :] = 0
                else:
                    out_mat[y_iter, x_iter, np.newaxis] = img[matyu, matxu, :]
            else:
                if matxu >= width:
                    out_mat[y_iter, x_iter] = 0
                elif matyu >= height:
                    out_mat[y_iter, x_iter] = 0
                elif matxu <= 0:
                    out_mat[y_iter, x_iter] = 0
                elif matyu <= 0:
                    out_mat[y_iter, x_iter] = 0
                else:
                    out_mat[y_iter, x_iter] = img[matyu, matxu]
            if matyu == 0 and y_iter > minytop:
                minytop = y_iter
            elif matyu == height and y_iter < minybot:
                minybot = y_iter
            elif matxu == 0 and x_iter > minxlef:
                minxlef = x_iter
            elif matxu == width and x_iter < minxrig:
                minxrig = x_iter
    out_mat = np.uint8(out_mat[minytop:minybot, minxlef:minxrig])
    return out_mat, minytop, minybot, minxlef, minxrig
	\end{lstlisting}

	\newpage
	\bibliographystyle{IEEEtran}
	\bibliography{reference}
		
	\newpage
	\section*{ACKNOWLEDGMENT}
	\begin{tcolorbox}[colback=white,colframe=white!70!black,title={\bfseries Author Information}]
	\par\noindent
		\parbox[t]{\linewidth}{
	 \noindent\parpic{\includegraphics[height=2in,width=1in,clip,keepaspectratio]{ShuolongChen_grey.jpg}}
	 \noindent{\bfseries Shuolong Chen}\emph{
	 received the B.S. degree in geodesy and geomatics engineering from Wuhan University, Wuhan China, in 2023.
	 He is currently a master candidate at the school of Geodesy and Geomatics, Wuhan University. His area of research currently focuses on integrated navigation systems and multi-sensor fusion.
	 Contact him via e-mail: shlchen@whu.edu.cn.
	 }}
	\end{tcolorbox}
		
		
\end{document}
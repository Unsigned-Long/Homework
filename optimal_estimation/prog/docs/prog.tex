\documentclass[12pt, onecolumn]{article}

% 引入相关的包
\usepackage{amsmath, listings, fontspec, geometry, graphicx, ctex, color, subfigure, amsfonts, amssymb}
\usepackage{multirow}
\usepackage[table,xcdraw]{xcolor}
\usepackage[ruled]{algorithm2e}
\usepackage[hidelinks]{hyperref}
\hypersetup{
	colorlinks=true,
	linkcolor=red,
	citecolor=red,
}

% 设定页面的尺寸和比例
\geometry{left = 1.5cm, right = 1.5cm, top = 1.5cm, bottom = 1.5cm}

% 设定两栏之间的间距
\setlength\columnsep{1cm}

% 设定字体,为代码的插入作准备
\newfontfamily\ubuntu{Ubuntu Mono}
\newfontfamily\consolas{Consolas}

% 头部信息
\title{\normf{编程:观测值逐次更新的扩展卡尔曼滤波器}}
\author{\normf 姓名:陈烁龙\;\;\;学号:2023202140019\;\;\;学院:测绘学院}
\date{\normf{\today}}

% 代码块的风格设定
\lstset{
	language=C++,
	basicstyle=\scriptsize\ubuntu,
	keywordstyle=\textbf,
	stringstyle=\itshape,
	commentstyle=\itshape,
	numberstyle=\scriptsize\ubuntu,
	showstringspaces=false,
	numbers=left,
	numbersep=8pt,
	tabsize=2,
	frame=single,
	framerule=1pt,
	columns=fullflexible,
	breaklines,
	frame=shadowbox, 
	backgroundcolor=\color[rgb]{0.97,0.97,0.97}
}

% 字体族的定义
% \fangsong \songti \heiti \kaishu
\newcommand\normf{\fangsong}
\newcommand\boldf{\heiti}
\newcommand\keywords[1]{\boldf{关键词:} \normf #1}

\newcommand\liehat[1]{\left[ #1 \right]_\times}
\newcommand\lievee[1]{\left[ #1 \right]^\vee}
\newcommand\liehatvee[1]{\left[ #1 \right]^\vee_\times}

\newcommand\mlcomment[1]{\iffalse #1 \fi}
%\newcommand\mlcomment[1]{ #1 }

\newcounter{problemname}
\newenvironment{problem}{\stepcounter{problemname}\par\noindent\normf\textbf{\textcolor[rgb]{1,0,0}{题目\arabic{problemname}.} }}{\leavevmode\\\par}
\newenvironment{solution}{\par\noindent\normf\textbf{解答: }}{\leavevmode\\\par}
\newenvironment{note}{\par\noindent\normf\textbf{题目\arabic{problemname}的注记: }}{\leavevmode\\\par}


\begin{document}
	
	% 插入头部信息
		\maketitle
		% 换页
		\thispagestyle{empty}
		\clearpage
		
		% 插入目录、图、表并换页
		\pagenumbering{roman}
		\tableofcontents
		\newpage
		\listoffigures
		\newpage
		\listoftables
		% 罗马字母形式的页码
		
		\clearpage
		% 从该页开始计数
		\setcounter{page}{1}
		% 阿拉伯数字形式的页码
		\pagenumbering{arabic}
	
	\section{\normf{问题分析}}
	\normf
	\subsection{\normf 状态转移方程}
	设$t$时刻的状态向量为$\boldsymbol{X}(t)=\left( x(t),y(t),v_x(t),v_y(t)\right) ^\top$,则根据题意,得到该系统的连续时间微分方差为
	:
	\begin{equation}
	\dot{\boldsymbol{X}}(t)=\begin{pmatrix}
	\dot{x}(t)\\\dot{y}(t)\\\dot{v}_x(t)\\\dot{v}_y(t)
	\end{pmatrix}=\begin{pmatrix}
	v_x(t)\\
	v_y(t)\\
	-k_x v^2_x(t)\\
	k_y v^2_y(t)
	\end{pmatrix}+\begin{pmatrix}
	0\\0\\0\\-1
	\end{pmatrix}\begin{pmatrix}
	g
	\end{pmatrix}+\begin{pmatrix}
	0&0\\
	0&0\\
	1&0\\
	0&1
	\end{pmatrix}\begin{pmatrix}
	e_x(t)\\
	e_y(t)
	\end{pmatrix}
	\end{equation}
	将其记为:
	\begin{equation}
	\dot{\boldsymbol{X}}(t)=\boldsymbol{F}(t)+\boldsymbol{B}\cdot\boldsymbol{u}+\boldsymbol{C}\cdot\boldsymbol{e}(t)
	\end{equation}
	线性化得到:
	\begin{equation}
	\dot{\boldsymbol{X}}(t)=\boldsymbol{F^*}+\boldsymbol{A}\cdot\left( \boldsymbol{X}(t)-\boldsymbol{X^*}\right) +\boldsymbol{B}\cdot\boldsymbol{u}+\boldsymbol{C}\cdot\boldsymbol{e(t)}
	\end{equation}
	其中:
	\begin{equation}
	\boldsymbol{F^*}=\begin{pmatrix}
		v_x^*\\
		v_y^*\\
		-k_x {v^*_x}^2\\
		k_y {v^*_y}^2
		\end{pmatrix}
	\qquad
	\boldsymbol{A}=\begin{pmatrix}
		0&0&1&0\\
		0&0&0&1\\
		0&0&-2 k_x v_x^*&0\\
		0&0&0&2 k_y v_y^*
		\end{pmatrix}
		\qquad
		\boldsymbol{X^*}=\begin{pmatrix}
			x^*\\y^*\\v_x^*\\v_y^*
			\end{pmatrix}
	\end{equation}
	进一步化简得到:
	\begin{equation}
	\begin{aligned}
	\dot{\boldsymbol{X}}(t)&=\boldsymbol{A}\cdot\boldsymbol{X}(t)
		+\left( \boldsymbol{F^*}-\boldsymbol{A}\cdot\boldsymbol{X^*}+\boldsymbol{B}\cdot\boldsymbol{u}\right) 
		+\boldsymbol{C}\cdot\boldsymbol{e}(t)
		\\
		&=\boldsymbol{A}\cdot\boldsymbol{X}(t)
				+\boldsymbol{G}
				+\boldsymbol{C}\cdot\boldsymbol{e}(t)
	\end{aligned}
	\end{equation}
	对其进行离散化。由于$\boldsymbol{A}$矩阵与时间无关,所以状态转移矩阵为:
	\begin{equation}
	\boldsymbol{\Phi}(t,\tau)=\boldsymbol{I}+\boldsymbol{A}\cdot(t-\tau)=
	\begin{pmatrix}
	1&0&(t-\tau)&0\\
	0&1&0&(t-\tau)\\
	0&0&1-2 k_x v_x^*\times(t-\tau)&0\\
	0&0&0&1+2 k_y v_y^*\times(t-\tau)\\
	\end{pmatrix}
	\end{equation}
	最终,得到离散化后的状态转移方程:
	\begin{equation}
	\begin{aligned}
	\boldsymbol{X}(t_k)&=\boldsymbol{\Phi}(t_k,t_{k-1})\cdot\boldsymbol{X}(t_{k-1})+\int_{t_{k-1}}^{t_k}\boldsymbol{\Phi}(t_k,\tau)\cdot\boldsymbol{G}\cdot d\tau
		+\int_{t_{k-1}}^{t_k}\boldsymbol{\Phi}(t_k,\tau)\cdot\boldsymbol{C}\cdot\boldsymbol{e}(\tau)\cdot d\tau
		\\&=\boldsymbol{\Phi}(t_k,t_{k-1})\cdot\boldsymbol{X}(t_{k-1})
		+\boldsymbol{c}(t_k,t_{k-1})
		+\boldsymbol{w}(t_{k-1})
	\end{aligned}
	\end{equation}
	令$\Delta t=t_k-t_{k-1}$,则其中:
	\begin{equation}
	\boldsymbol{c}(t_k,t_{k-1})=	\begin{pmatrix}
		\Delta t&0&(\Delta t)^2/2&0\\
		0&\Delta t&0&(\Delta t)^2/2\\
		0&0&\Delta t-k_x v_x^*\times(\Delta t)^2&0\\
		0&0&0&\Delta t+k_y v_y^*\times(\Delta t)^2\\
		\end{pmatrix}\begin{pmatrix}
		0\\0\\
		k_x{v_x^*}^2\\
		-k_y{v_y^*}^2-g
		\end{pmatrix}
	\end{equation}
	$\boldsymbol{w}(k-1)$的方差$\boldsymbol{D}_{w}(k-1)=\int_{t_{k-1}}^{t_k}\boldsymbol{\Phi}(t_k,\tau)\cdot\boldsymbol{C}\cdot\boldsymbol{D}_{e}(\tau)\cdot\boldsymbol{C}^\top\cdot\boldsymbol{\Phi}^\top(t_k,\tau)\cdot d\tau$为:
	\begin{equation}
	\begin{pmatrix}
	\frac{(\Delta t)^3}{3}\sigma_{e_x}^2
	&0&
	\left( \frac{1}{2}(\Delta t)^2-\frac{2}{3}k_xv_x^*(\Delta t)^3\right)\sigma_{e_x}^2 
	&0\\
	0&
	\frac{(\Delta t)^3}{3}\sigma_{e_y}^2
	&0&
	\left( \frac{1}{2}(\Delta t)^2+\frac{2}{3}k_yv_y^*(\Delta t)^3\right)\sigma_{e_y}^2 
	\\
	\cdots&\cdots&
	\left(\Delta t-2k_xv_x^*(\Delta t)^2+\frac{4}{3}(k_xv^*_x)^2(\Delta t)^3 \right) \sigma_{e_x}^2
	&0
	\\
	\cdots&\cdots&\cdots&
	\left(\Delta t+2k_yv_y^*(\Delta t)^2+\frac{4}{3}(k_yv^*_y)^2(\Delta t)^3 \right) \sigma_{e_y}^2
	\end{pmatrix}
	\end{equation}
	\subsection{\normf 测量更新方程}
	令$\boldsymbol{Z}(t_k)=\left( r(t_k),\alpha(t_k)\right) ^\top$,则有:
	\begin{equation}
	\boldsymbol{Z}(t_k)=\begin{pmatrix}
	r(t_k)\\\alpha(t_k)
	\end{pmatrix}=\begin{pmatrix}
	\sqrt{x^2(t_k)+y^2(t_k)}\\
	\arctan\begin{aligned}
	\frac{x(t_k)}{y(t_k)}
	\end{aligned}
	\end{pmatrix}+
	\begin{pmatrix}
	\Delta_r(t_k)\\\Delta_\alpha(t_k)
	\end{pmatrix}
	\end{equation}
	将其记为:
	\begin{equation}
	\boldsymbol{Z}(t_k)=\boldsymbol{L}(t_k)+\boldsymbol{\Delta}_z
	\end{equation}
	线性化后得到:
	\begin{equation}
	\boldsymbol{Z}(t_k)=\boldsymbol{L^*}+\boldsymbol{H}\cdot\left(
	\boldsymbol{X}(t_k)-\boldsymbol{X^*}
	\right) +\boldsymbol{\Delta}_z
	\end{equation}
	其中:
	\begin{equation}
	\boldsymbol{L^*}=\begin{pmatrix}
		\sqrt{(x^*)^2+(y^*)^2}\\
		\arctan\begin{aligned}
		\frac{x^*}{y^*}
		\end{aligned}
		\end{pmatrix}
		\qquad
		\boldsymbol{H}=\begin{pmatrix}
			\begin{aligned}
			\frac{x^*}{\sqrt{(x^*)^2+(y^*)^2}}
			\end{aligned}
		&
			\begin{aligned}
			\frac{y^*}{\sqrt{(x^*)^2+(y^*)^2}}
			\end{aligned}
		&0&0\\
		\begin{aligned}
		\frac{1}{1+(x^*/y^*)^2}\times\frac{1}{y^*}
		\end{aligned}
		&
		\begin{aligned}
		\frac{1}{1+(x^*/y^*)^2}\times-\frac{x^*}{(y^*)^2}
		\end{aligned}
		&0&0
		\end{pmatrix}
	\end{equation}
	整理得到:
	\begin{equation}
	\boldsymbol{Z}(t_k)-\boldsymbol{L^*}+\boldsymbol{H}\cdot\boldsymbol{X^*}
	=\boldsymbol{H}\cdot\boldsymbol{X}(t_k)	+\boldsymbol{\Delta}_z
	\end{equation}
	
	\subsection{\normf 观测值逐次更新的扩展卡尔曼滤波}
	已知$t_{k-1}$时刻的状态$\boldsymbol{X}(t_{k-1})$及其方差$\boldsymbol{D}_{\boldsymbol{X}}(t_{k-1})$,首先通过状态转移方程进行状态预测和方差传播:
	\begin{equation}
	\begin{cases}
	\begin{aligned}
	\boldsymbol{X}(t_k,t_{k-1})&=
		\boldsymbol{\Phi}(t_k,t_{k-1})\cdot\boldsymbol{X}(t_{k-1})
		+\boldsymbol{c}(t_k,t_{k-1})
		\\
		\boldsymbol{D}_{\boldsymbol{X}}(t_k,t_{k-1})&=\boldsymbol{\Phi}(t_k,t_{k-1})\cdot\boldsymbol{D}_{\boldsymbol{X}}(t_k)\cdot\boldsymbol{\Phi}^\top(t_k,t_{k-1})+\boldsymbol{D}_{w}(t_{k-1})
	\end{aligned}
	\end{cases}
	\end{equation}
	而后基于量测值进行逐次的测量更新。首先计算残差:
	\begin{equation}
	\boldsymbol{V}_i(t_k)=\boldsymbol{Z}_i(t_k)-\boldsymbol{L}_i(t_k)
	\end{equation}
	需要注意的是,下一次的残差计算需要使用上一次的估计结果。而后计算卡尔曼增益:
	\begin{equation}
	\boldsymbol{K}_i=\boldsymbol{D}_{\boldsymbol{X}}(t_{k},t_{k-1})\cdot\boldsymbol{h}_i^\top\left( \boldsymbol{h}_i\cdot\boldsymbol{D}_{\boldsymbol{X}}(t_k,t_{k-1})\cdot\boldsymbol{h}_i^\top+\boldsymbol{\Delta}_z^i\right) ^{-1}
	\end{equation}
	需要注意的是,下一次的卡尔曼增益矩阵的计算需要使用上一次的状态方差。最后计算状态更新后的值:
	\begin{equation}
	\begin{cases}
	\begin{aligned}
	\boldsymbol{X}(t_k)&=\boldsymbol{X}(t_k,t_{k-1})+\boldsymbol{K}_i\cdot
		\boldsymbol{V}_i(t_k)
		\\
	\boldsymbol{D}_{\boldsymbol{X}}(t_k)&=\left( \boldsymbol{I}-\boldsymbol{K}_i\cdot\boldsymbol{h}_i\right)\cdot\boldsymbol{D}_{\boldsymbol{X}}(t_k,t_{k-1}) 
	\end{aligned}
	\end{cases}
	\end{equation}
	直至当前所有观测值更新完毕,进入下一时刻进行滤波。
	
	\section{\normf{实验结果}}
	
	\section{\normf{附录:关键代码片段}}
	
	
\end{document}


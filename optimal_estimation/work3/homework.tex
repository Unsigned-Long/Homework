\documentclass[12pt, onecolumn]{article}

% 引入相关的包
\usepackage{amsmath, listings, fontspec, geometry, graphicx, ctex, color, subfigure, amsfonts, amssymb}
\usepackage{multirow}
\usepackage[table,xcdraw]{xcolor}
\usepackage[ruled]{algorithm2e}
\usepackage[hidelinks]{hyperref}
\hypersetup{
	colorlinks=true,
	linkcolor=red,
	citecolor=red,
}

% 设定页面的尺寸和比例
\geometry{left = 1.5cm, right = 1.5cm, top = 1.5cm, bottom = 1.5cm}

% 设定两栏之间的间距
\setlength\columnsep{1cm}

% 设定字体,为代码的插入作准备
\newfontfamily\ubuntu{Ubuntu Mono}
\newfontfamily\consolas{Consolas}

% 头部信息
\title{\normf{最优估计课程作业}}
\author{\normf{陈烁龙 2023202140019}}
\date{\normf{\today}}

% 代码块的风格设定
\lstset{
	language=C++,
	basicstyle=\scriptsize\ubuntu,
	keywordstyle=\textbf,
	stringstyle=\itshape,
	commentstyle=\itshape,
	numberstyle=\scriptsize\ubuntu,
	showstringspaces=false,
	numbers=left,
	numbersep=8pt,
	tabsize=2,
	frame=single,
	framerule=1pt,
	columns=fullflexible,
	breaklines,
	frame=shadowbox, 
	backgroundcolor=\color[rgb]{0.97,0.97,0.97}
}

% 字体族的定义
% \fangsong \songti \heiti \kaishu
\newcommand\normf{\fangsong}
\newcommand\boldf{\heiti}
\newcommand\keywords[1]{\boldf{关键词:} \normf #1}

\newcommand\liehat[1]{\left[ #1 \right]_\times}
\newcommand\lievee[1]{\left[ #1 \right]^\vee}
\newcommand\liehatvee[1]{\left[ #1 \right]^\vee_\times}

\newcommand\mlcomment[1]{\iffalse #1 \fi}
%\newcommand\mlcomment[1]{ #1 }

\newcounter{problemname}
\newenvironment{problem}{\stepcounter{problemname}\par\noindent\normf\textbf{\textcolor[rgb]{1,0,0}{题目\arabic{problemname}.} }}{\leavevmode\\\par}
\newenvironment{solution}{\par\noindent\normf\textbf{解答: }}{\leavevmode\\\par}
\newenvironment{note}{\par\noindent\normf\textbf{题目\arabic{problemname}的注记: }}{\leavevmode\\\par}


\begin{document}
	
	% 插入头部信息
	\maketitle
	% 换页
	\thispagestyle{empty}
	\clearpage
	\setcounter{page}{1}
	% 阿拉伯数字形式的页码
	\pagenumbering{arabic}
	
	\section{\normf{第二章:最优估计基础理论}}
	
	\begin{problem}
		某随机变量$Z$的概率密度函数为$f_{Z\vert\theta}(z\vert \theta)=\begin{cases}
		\theta\exp\left( -\theta z\right) &z\ge0\\
		0&z<0
		\end{cases}$。现在对随机变量$Z$进行独立观测,观测值为$Z_1,\cdots,Z_n$。($i$) 求$\theta$的极大似然估计。($ii$) 若参数$\theta$为随机量,概率密度函数为$f_{\theta}(\theta)=\begin{cases}
		\frac{1}{2}&\theta\in(0,2]\\
		0&\theta\not\in(0,2]
		\end{cases}$,求参数$\theta$的极大验后估计。
	\end{problem}
	\begin{solution}
		\begin{enumerate}
			\item 由于各观测值相互独立,易得似然函数为:
			\begin{equation*}
			p_{Z\vert \theta}(z\vert \theta)=\prod_{i=1}^{n}f_{Z_i\vert\theta}(z_i\vert \theta)=
			\prod_{i=1}^{n}\theta\exp\left( -\theta z_i\right)
			\end{equation*}
			对应的对数似然函数为:
			\begin{equation*}
			\ln p_{Z\vert \theta}(z\vert \theta)=l\cdot\ln\theta-\theta\sum_{i=0}^{n}z_i
			\end{equation*}
			令其对$\theta$的偏导数为0,得到:
			\begin{equation*}
			\theta=\frac{l}{\sum_{i=0}^{n}z_i}
			\end{equation*}
			
			\item 由于$\theta$为随机量,考虑概率密度函数,其易得极大后验为:
			\begin{equation*}
			p_{\theta\vert Z}(\theta\vert z)\propto p_{Z\vert \theta}(z\vert \theta)\cdot p_{\theta}(\theta)=\theta^l\cdot\prod_{i=1}^{n}\exp\left( -\theta z_i\right)\cdot\frac{1}{2}
			\qquad \theta\in(0,2]
			\end{equation*}
			取对数运算后得到:
			\begin{equation*}
			\ln p_{Z\vert \theta}(z\vert \theta)=l\cdot\ln\theta-\theta\sum_{i=0}^{n}z_i-\ln 2
			\end{equation*}
			令其对$\theta$的偏导数为0,得到:
			\begin{equation*}
			\theta=\frac{l}{\sum_{i=0}^{n}z_i}
			\end{equation*}
			虽然参数$\theta$为随机量,但是其服从均匀分布,所以最大验后退化为极大似然估计。
		\end{enumerate}
	\end{solution}
	
	\begin{problem}
		给出线性最小方差的推导过程。
	\end{problem}
	\begin{solution}
		设估计向量$\boldsymbol{\hat{X}}$和观测向量$\boldsymbol{Z}$满足如下的线性关系:
		\begin{equation*}
		\boldsymbol{\hat{X}}=\boldsymbol{B}\boldsymbol{Z}+\boldsymbol{l}
		\end{equation*}
		求得$\boldsymbol{B}$和$\boldsymbol{l}$,也就求得了$\boldsymbol{\hat{X}}$的线性最小方差估计。根据最小方差的估计准则,有:
		\begin{equation*}
		E\left[ \left( \boldsymbol{X}-\boldsymbol{\hat{X}}\right)^\top\left( \boldsymbol{X}-\boldsymbol{\hat{X}}\right) \right]=\min
		\end{equation*}
		由于方差矩阵迹最小的估计和方差矩阵最小的估计等价,故对其进行取迹运算,得到:
		\begin{equation*}
		\mathrm{tr}\left( E\left[ \left( \boldsymbol{X}-\boldsymbol{\hat{X}}\right)^\top\left( \boldsymbol{X}-\boldsymbol{\hat{X}}\right) \right]\right) 
		=\mathrm{tr}\left( E\left[ \left( \boldsymbol{X}-\boldsymbol{B}\boldsymbol{Z}-\boldsymbol{l}\right)^\top \left( \boldsymbol{X}-\boldsymbol{B}\boldsymbol{Z}-\boldsymbol{l}\right)\right]\right) 
		=\min
		\end{equation*}
		分别对$\boldsymbol{B}$和$\boldsymbol{l}$求偏导,并令导数为0:
		\begin{equation*}
		\begin{cases}
		\begin{aligned}
		\frac{\partial\mathrm{tr}(\cdot)}{\partial \boldsymbol{B}}&=
		-2\cdot E\left[ \left( \boldsymbol{X}-\boldsymbol{B}\boldsymbol{Z}-\boldsymbol{l}\right)\boldsymbol{Z}^T \right]=0 
		\\
		\frac{\partial\mathrm{tr}(\cdot)}{\partial \boldsymbol{l}}&=
		-E\left[  \boldsymbol{X}-\boldsymbol{B}\boldsymbol{Z}-\boldsymbol{l}\right]=0
		\end{aligned}
		\end{cases}
		\Rightarrow
		\begin{cases}
		\begin{aligned}
		E\left[ \boldsymbol{XZ^\top}\right] -\boldsymbol{B}E\left[\boldsymbol{Z}\boldsymbol{Z}^\top\right] -\boldsymbol{l}E\left[\boldsymbol{Z} \right]  &= 0
		\\
		E\left[\boldsymbol{X} \right] -\boldsymbol{B}E\left[ \boldsymbol{Z}\right] -\boldsymbol{l}&=0
		\end{aligned}
		\end{cases}
		\end{equation*}
		由于$D_{XZ}=E[\boldsymbol{XZ^\top}]-E[\boldsymbol{X}]E[\boldsymbol{Z}]$,$D_{Z}=E[\boldsymbol{ZZ^\top}]-E^2[\boldsymbol{Z}]$,所以有:
		\begin{equation*}
		\begin{cases}
		\begin{aligned}
		D_{XZ}+E[\boldsymbol{X}]E[\boldsymbol{Z}]-\boldsymbol{B}D_{Z}-\boldsymbol{B}E^2[\boldsymbol{Z}]&=\boldsymbol{l}E\left[\boldsymbol{Z}\right]
		\\
		E\left[\boldsymbol{X} \right]E\left[\boldsymbol{Z}\right] -\boldsymbol{B}E^{2}\left[ \boldsymbol{Z}\right]&=\boldsymbol{l}E\left[\boldsymbol{Z}\right]
		\end{aligned}
		\end{cases}
		\end{equation*}
		得到:
		\begin{equation*}
		D_{XZ}-\boldsymbol{B}D_{Z}=0
		\quad
		\Rightarrow
		\quad
		\boldsymbol{B}=D_{XZ}D_Z^{-1}
		\end{equation*}
		进而得到:
		\begin{equation*}
		\boldsymbol{l}=E\left[\boldsymbol{X} \right] -D_{XZ}D_Z^{-1}E\left[ \boldsymbol{Z}\right]
		\end{equation*}
		所以有:
		\begin{equation*}
		\begin{aligned}
		\boldsymbol{\hat{X}}&=\boldsymbol{B}\boldsymbol{Z}+\boldsymbol{l}
		\\&=
		D_{XZ}D_Z^{-1}\boldsymbol{Z}+E\left[\boldsymbol{X} \right] -D_{XZ}D_Z^{-1}E\left[ \boldsymbol{Z}\right]
		\\&=
		D_{XZ}D_Z^{-1}\left( \boldsymbol{Z}-E\left[ \boldsymbol{Z}\right]\right) +E\left[\boldsymbol{X} \right]
		\end{aligned}
		\end{equation*}
	\end{solution}
	\begin{problem}
		设随机变量$Z$服从正态分布$z\sim N(\theta,1)$,其中参数$\theta$为随机量,其概率密度函数为$f_{\theta}(\theta)=\exp\left(-10(\theta-3)^2 \right) $。现在对随机变量$Z$进行独立观测,观测值$Z_1,\cdots,Z_n$为同精度观测,并且条件独立,设损失函数为均匀损失函数:$L(\theta,\hat{\theta}(Z))=\begin{cases}
		1&\Vert\Delta\theta\Vert\le\Delta/2\\
		0&\Vert\Delta\theta\Vert>\Delta/2
		\end{cases}$,求$\theta$的贝叶斯估计。
		\begin{enumerate}
			\item 给出概率密度函数$p(z_1,\cdots,z_n\vert\theta)$和$p(z_1,\cdots,z_n,\theta)$。
			\item 给出贝叶斯风险的表达式。
			\item 给出$\theta$ 的贝叶斯估计。
		\end{enumerate}
	\end{problem}
	\begin{solution}
		\begin{enumerate}
			\item 由于随机变量$Z$服从正态分布$z\sim N(\theta,1)$,有:
			\begin{equation*}
			p_{Z\vert\theta}(z\vert \theta)=\frac{1}{\sqrt{2\pi}}\exp\left( -\frac{(z-\theta)^2}{2}\right) 
			\end{equation*}
			由于观测值$Z_1,\cdots,Z_n$为同精度观测,并且条件独立,所以:
			\begin{equation*}
			p_{Z\vert\theta}(z_1,\cdots,z_n\vert\theta)=\prod_{i=1}^{n}p_{Z\vert\theta}(z_i\vert \theta)=\prod_{i=1}^{n}\frac{1}{\sqrt{2\pi}}\exp\left( -\frac{(z_i-\theta)^2}{2}\right) =
			\left( \frac{1}{\sqrt{2\pi}}\right)^n\exp\left(
			-\frac{1}{2}\sum_{i=1}^{n}(z_i-\theta)^2
			 \right)  
			\end{equation*}
			和:
			\begin{equation*}
			p_{Z,\theta|}(z_1,\cdots,z_n,\theta)=p_{Z\vert\theta}(z_1,\cdots,z_n\vert\theta)\cdot f_{\theta}(\theta)=
			\left( \frac{1}{\sqrt{2\pi}}\right)^n\exp\left(
			-\frac{1}{2}\sum_{i=1}^{n}(z-\theta)^2-10(\theta-3)^2
			\right)  
			\end{equation*}
			\item 令$\hat{\theta}(Z)$为参数$\theta$ 的贝叶斯估计,$\boldsymbol{z}=\left( Z_1,Z_2,\cdots,Z_n\right)^\top $为观测向量,贝叶斯风险即为损失函数的期望,所以有:
			\begin{equation*}
			\begin{aligned}
			R_B(\theta,\hat{\theta}(\boldsymbol{z}))=E\left[L(\theta,\hat{\theta}(\boldsymbol{z}))
			\right] &=
			\int_{-\infty}^{+\infty}\int_{-\infty}^{+\infty}
			L(\theta,\hat{\theta}(\boldsymbol{z}))\cdot p_{Z,\theta}(\boldsymbol{z},\theta)
			d\theta d\boldsymbol{z}
			\\
			&=\int_{-\infty}^{+\infty}\left\lbrace \int_{-\infty}^{+\infty}
			L(\theta,\hat{\theta}(\boldsymbol{z}))\cdot p_{\theta\vert Z}(\theta\vert \boldsymbol{z})
			d\theta \right\rbrace p_{Z}(\boldsymbol{z})d\boldsymbol{z}
			\\&=\int_{-\infty}^{+\infty}\left\lbrace
			1- \int_{-\Delta/2}^{+\Delta/2}
			p_{\theta\vert Z}(\theta\vert \boldsymbol{z})
			d\theta 
			\right\rbrace p_{Z}(\boldsymbol{z})d\boldsymbol{z}
			\end{aligned}
			\end{equation*}
			\item 由于损失函数为均匀损失函数,所以此时贝叶斯估计即为最大验后估计,即:
			\begin{equation*}
			p_{\theta\vert Z}(\theta\vert \boldsymbol{z})=\max\Leftrightarrow p_{Z\vert \theta}(\boldsymbol{z}\vert\theta )\cdot f_{\theta}(\theta)=\max
			\end{equation*}
			也即使得下目标函数最大:
			\begin{equation*}
			O(\theta)=\left( \frac{1}{\sqrt{2\pi}}\right)^n\exp\left(
			-\frac{1}{2}\sum_{i=1}^{n}(z_i-\theta)^2-10(\theta-3)^2
			\right)  
			\end{equation*}
			即:
			\begin{equation*}
			\frac{1}{2}\sum_{i=1}^{n}(z_i-\theta)^2+10(\theta-3)^2=\min
			\end{equation*}
			对$\theta$求导,得到:
			\begin{equation*}
			n\theta-\sum_{i=1}^{n}z_i+20\theta-60=0
			\end{equation*}
			得到$\theta$ 的贝叶斯估计为:
			\begin{equation*}
			\hat{\theta}_{B}=\hat{\theta}_{MAP}=\frac{\sum_{i=1}^{n}z_i+60}{n+20}
			\end{equation*}
		\end{enumerate}
	\end{solution}
	
\end{document}


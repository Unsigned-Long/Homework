\documentclass[12pt, onecolumn]{article}

% 引入相关的包
\usepackage{amsmath, listings, fontspec, geometry, graphicx, ctex, color, subfigure, amsfonts, amssymb}
\usepackage{multirow}
\usepackage[table,xcdraw]{xcolor}
\usepackage[ruled]{algorithm2e}
\usepackage[hidelinks]{hyperref}
\hypersetup{
	colorlinks=true,
	linkcolor=red,
	citecolor=red,
}

% 设定页面的尺寸和比例
\geometry{left = 1.5cm, right = 1.5cm, top = 1.5cm, bottom = 1.5cm}

% 设定两栏之间的间距
\setlength\columnsep{1cm}

% 设定字体,为代码的插入作准备
\newfontfamily\ubuntu{Ubuntu Mono}
\newfontfamily\consolas{Consolas}

% 头部信息
\title{\normf{最优估计课程作业}}
\author{\normf{陈烁龙 2023202140019}}
\date{\normf{\today}}

% 代码块的风格设定
\lstset{
	language=C++,
	basicstyle=\scriptsize\ubuntu,
	keywordstyle=\textbf,
	stringstyle=\itshape,
	commentstyle=\itshape,
	numberstyle=\scriptsize\ubuntu,
	showstringspaces=false,
	numbers=left,
	numbersep=8pt,
	tabsize=2,
	frame=single,
	framerule=1pt,
	columns=fullflexible,
	breaklines,
	frame=shadowbox, 
	backgroundcolor=\color[rgb]{0.97,0.97,0.97}
}

% 字体族的定义
% \fangsong \songti \heiti \kaishu
\newcommand\normf{\fangsong}
\newcommand\boldf{\heiti}
\newcommand\keywords[1]{\boldf{关键词:} \normf #1}

\newcommand\liehat[1]{\left[ #1 \right]_\times}
\newcommand\lievee[1]{\left[ #1 \right]^\vee}
\newcommand\liehatvee[1]{\left[ #1 \right]^\vee_\times}

\newcommand\mlcomment[1]{\iffalse #1 \fi}
%\newcommand\mlcomment[1]{ #1 }

\newcounter{problemname}
\newenvironment{problem}{\stepcounter{problemname}\par\noindent\normf\textbf{\textcolor[rgb]{1,0,0}{题目\arabic{problemname}.} }}{\leavevmode\\\par}
\newenvironment{solution}{\par\noindent\normf\textbf{解答: }}{\leavevmode\\\par}
\newenvironment{note}{\par\noindent\normf\textbf{题目\arabic{problemname}的注记: }}{\leavevmode\\\par}


\begin{document}
	
	% 插入头部信息
	\maketitle
	% 换页
	\thispagestyle{empty}
	\clearpage
	\setcounter{page}{1}
	% 阿拉伯数字形式的页码
	\pagenumbering{arabic}
	
	\section{\normf{第一章:最优估计数学基础}}
	\begin{problem}
		什么是严格平稳随机过程?什么是宽平稳随机过程,相对于严格平稳随机
		过程,它放宽了什么条件?
	\end{problem}
	\begin{solution}
		\begin{enumerate}
			\item 给定一个随机过程$X(t)$,如果其概率密度函数不随时间变化,即:
			\begin{equation*}
			p_{X}(x)=p_{X}(x+\tau)
			\end{equation*}
			则称该随机过程是严格平稳的。
			
			\item 给定一个随机过程$X(t)$,如果其均值和时间无关,自相关函数与时间起点无关,只与时间间隔有关,即:
			\begin{equation*}
			E(X(t))=\mu_{X}
			\qquad
			R(X(t_i),X(t_j))=f(t_i-t_j)=f(\Delta t)
			\end{equation*}
			则称该随机过程是宽平稳的。
			
			\item 宽平稳随机过程不需要保证概率密度函数一直保持不变,只需要其特征值与时间无关,自相关函数只与时间间隔有关即可。
		\end{enumerate}
	\end{solution}
	
	\begin{problem}
		若随机过程$X(t)=At,t\in(-\infty,+\infty)$,且A服从$(0,1)$上的均匀分布,求$E(X(t))$和$R_{X}(t_1,t_2)$。
	\end{problem}
	\begin{solution}
		由于A服从$(0,1)$上的均匀分布,所以:
		\begin{equation*}
		p_{A}(a)=\begin{cases}
		1&a\in(0,1)\\
		0&a\not\in(0,1)
		\end{cases}
		\end{equation*}
		且有:
		\begin{equation*}
		E(A)=\int_{0}^{1}a\cdot da=\frac{1}{2}
		\qquad
		E(A^2)=\int_{0}^{1}a^2\cdot da=\frac{1}{3}
		\end{equation*}
		所以随机过程$X(t)$的均值函数为:
		\begin{equation*}
		E(X(t))=tE(A)=\frac{t}{2}
		\end{equation*}
		其自相关函数为:
		\begin{equation*}
		R(t_i,t_j)=E(X(t_i)X(t_j))=t_it_jE(A^2)=\frac{t_it_j}{3}
		\end{equation*}
	\end{solution}

	\begin{problem}
		什么是方差和均方差,它们之间有什么异同?
	\end{problem}
	\begin{solution}
		\begin{enumerate}
			\item 方差是描述一个随机变量与其期望的偏离程度,即:
			\begin{equation*}
			D(X)=E((X-E(X))^2)
			\end{equation*}
			方差小,说明$X$的数值稳定;方差大,说明$X$的数值不稳定。方差只能描述观测值与其期望之间的离散程度,并不能反映出观测值中是否包含系统误差,其反映的是观测值的精度。
			
			\item 均方差描述一个随机变量与参考值(通常是真值)的偏离程度,即:
			\begin{equation*}
			MSE(X)=E((X-\widetilde{X})^2)
			\end{equation*}
			其中$\widetilde{X}$为参考值。均方差能够反映出观测值中是否包含系统误差,是衡量观测值的准度的指标。
		\end{enumerate}
	\end{solution}
	
	\begin{problem}
		什么是各态历经性?
	\end{problem}
	\begin{solution}
		当平稳随机过程具有均值历经性和相关函数历经性,则称该平稳随机过程具有各态历经性:
		\begin{enumerate}
			\item 均值历经性:若随机过程的平均以概率1趋近全集期望,则称该平稳随机过程具有均值历经性。
			
			\item 相关函数历经性:若随机过程的时间平均相关干函数以概率1趋近全集相关函数,则称该平稳随机过程具有相关函数历经性。
		\end{enumerate}
	\end{solution}
	
	\begin{problem}
		平稳随机过程的平均功率与均方值和相关函数有什么关系?
	\end{problem}
	\begin{solution}
		由平稳随机过程的平均功率:
		\begin{equation*}
		\Psi=\lim_{T\to\infty}E\left( 
		\frac{1}{2T}\int_{-T}^{T}X^2(t)dt
		\right) 
		\end{equation*}
		可以得到:
		\begin{equation*}
		\Psi=E(X^2(t))=R_{X}(0)
		\end{equation*}
		所以,平稳随机过程的平均功率与均方值相对,等于相关函数在0处的取值。
	\end{solution}
	\section{\normf{第二章:最优估计基础理论}}
	\subsection{\normf{极大似然估计}}
	\setcounter{problemname}{0}
	\begin{problem}
		设总体X的概率密度函数为
		\begin{equation*}
		f(x\vert \theta)=\begin{cases}
		(\theta+1)x^\theta&x\in(0,1)\\
		0&x\not\in(0,1)
		\end{cases}
		\end{equation*}
		其中$\theta>1$是未知参数。$X_1,\cdots,X_l$是来自总体$x$的样本,
		且相互随机独立。用最大似然估计法求$\theta$的估计量。
	\end{problem}
	\begin{solution}
		由于$X_1,\cdots,X_l$是来自总体$x$的样本,且相互随机独立,可以得到似然函数:
		\begin{equation*}
		p(x\vert \theta)=\prod_{i=1}^{l}p_{X}(x)=\prod_{i=1}^{l}(\theta+1)x_i^\theta=
		(\theta+1)^l\cdot\prod_{i=1}^{l}x_i^\theta
		\end{equation*}
		对数似然为:
		\begin{equation*}
		\ln	p(x\vert \theta)=l\cdot\ln(\theta+1)+\theta\cdot \sum_{i=0}^{l}\ln x_i
		\end{equation*}
		对$\theta$求偏导并令其为0:
		\begin{equation*}
		\frac{\partial \ln	p(x\vert \theta)}{\partial \theta}=
		\frac{l}{\theta+1}+\sum_{i=0}^{l}\ln x_i=0
		\end{equation*}
		得到:
		\begin{equation*}
		\theta=-\frac{l}{\sum_{i=0}^{l}\ln x_i}-1
		\end{equation*}
	\end{solution}
	
	\subsection{\normf{极大验后估计}}
	\begin{problem}
		设$Z$服从二项分布,即$Z\sim B(N,p)$,且$N$已知,$p$为未知参数,$p$的先验分布为0到1的均匀分布,即$p\sim U(0,1)$。现在有$l$个独立样本:$\boldsymbol{z}=\begin{pmatrix}
		z_1&z_2&\cdots&z_l
		\end{pmatrix}^\top$,请:
		\begin{enumerate}
			\item 给出分布律$P(Z=z_i\vert p)$和$P(Z=\boldsymbol{z}\vert p)$;
			\item 求联合分布律$P(Z=\boldsymbol{z},p)$;
			\item 求$p$的极大后验估计。
		\end{enumerate}
	\end{problem}
	\begin{solution}
		\begin{enumerate}
			\item 已知Z的分布律为:
			\begin{equation*}
			P(Z=z)=C_N^z\cdot p^z\cdot\left( 1-p\right)^{N-z} \quad\mathrm{s.t.}\quad z=0,1,2,\cdots,N
			\end{equation*}
			$p$的概率密度函数为:
			\begin{equation*}
			P(p)=\begin{cases}
			1&p\in(0,1)\\
			0&p\not\in(0,1)
			\end{cases}
			\end{equation*}
			故有:
			\begin{equation*}
			P(Z=z_i\vert p)=C_N^{z_i}\cdot p^{z_i}\cdot\left( 1-p\right)^{N-{z_i}}
			\quad\mathrm{s.t.}\quad p\in(0,1)
			\end{equation*}
			由于$\boldsymbol{z}$中的样本独立,所以:
			\begin{equation*}
			P(Z=\boldsymbol{z}\vert p)=\prod_{i=1}^{l}P(Z=z_i\vert p)=
			p^{\sum_{i=1}^{l}z_i}\cdot\left( 1-p\right)^{l\cdot N-\sum_{i=1}^{l}z_i}\cdot\prod_{i=1}^{l}C_N^{z_i}
			\end{equation*}
			\item 易得:
			\begin{equation*}
			P(Z=\boldsymbol{z},p)=P(Z=\boldsymbol{z}\vert p)\cdot P(p)=
			p^{\sum_{i=1}^{l}z_i}\cdot\left( 1-p\right)^{l\cdot N-\sum_{i=1}^{l}z_i}\cdot\prod_{i=1}^{l}C_N^{z_i}
			\quad\mathrm{s.t.}\quad p\in(0,1)
			\end{equation*}
			
			\item 由于:
			\begin{equation*}
			P(p\vert Z=\boldsymbol{z})\propto P(Z=\boldsymbol{z}\vert p)\cdot P(p)=P(Z=\boldsymbol{z},p)        
			\end{equation*}
			对齐取对数运算,得到:
			\begin{equation*}
			\ln P(p\vert Z=\boldsymbol{z})=\sum_{i=1}^{l}z_i\cdot\ln p+
			\left( l\cdot N-\sum_{i=1}^{l}z_i\right) \cdot\left( 1-p\right)+
			\sum_{i=1}^{l}\ln C_N^{z_i}
			\end{equation*}
			对$p$求导,得到:
			\begin{equation*}
			\frac{\partial \ln P(p\vert Z=\boldsymbol{z})}{\partial p}=
			\left( \frac{1}{p}+1\right) \cdot\sum_{i=1}^{l}z_i-l\cdot N
			\end{equation*}
			令其为0,得到:
			\begin{equation*}
			p=\frac{\sum_{i=1}^{l}z_i}{l\cdot N-\sum_{i=1}^{l}z_i}
			\end{equation*}
		\end{enumerate}
	\end{solution}

	\subsection{\normf{最小方差估计}}
	\begin{problem}
		已知$X$的先验随机信息为$\mu_X=2,\sigma_X=0.1$,现对$X$进行了独立同精度观测,观测值为$\begin{pmatrix}
		1.9&2.2
		\end{pmatrix}^\top$,观测中误差为$\delta_\Delta=0.2$。观测误差与$X$随机独立。求$X$的最小方差估计。 
	\end{problem}
	\begin{solution}
		由题意知:
		\begin{equation*}
		Z=X+\Delta
		\end{equation*}
		观测值和待估参数是线性关系。由于观测误差与$X$随机独立,所以有:
		\begin{equation*}
		D_Z=D_X+D_\Delta=0.1^2+0.2^2=0.05
		\qquad
		D_{XZ}=1\times D_X=0.01
		\qquad
		E(Z)=E(X)=2
		\end{equation*}
		由线性最小方差参数估计得到:
		\begin{equation*}
		\begin{aligned}
		X_{MV}&=E(X)+D_{XZ}D_Z^{-1}\left( Z-E(Z)\right) \\
		&=2+\begin{pmatrix}
		0.01&0.01
		\end{pmatrix}\begin{pmatrix}
		0.05&0\\0&0.05
		\end{pmatrix}^{-1}\left( 
		\begin{pmatrix}
		1.9\\2.2
		\end{pmatrix}-\begin{pmatrix}
		2\\2
		\end{pmatrix}\right) 
		\\
		&=2+\begin{pmatrix}
		0.01&0.01
		\end{pmatrix}\begin{pmatrix}
		20&0\\0&20
		\end{pmatrix}
		\begin{pmatrix}
		-0.1\\0.2
		\end{pmatrix}
		\\
		&=2.02
		\end{aligned}
		\end{equation*}
	\end{solution}
	
\end{document}


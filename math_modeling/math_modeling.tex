\documentclass[12pt, onecolumn]{article}

% 引入相关的包
\usepackage{amsmath, listings, fontspec, geometry, graphicx, ctex, color, subfigure, amsfonts, amssymb}
\usepackage{multirow}
\usepackage[table,xcdraw]{xcolor}
\usepackage[ruled]{algorithm2e}
\usepackage[hidelinks]{hyperref}
\hypersetup{
	colorlinks=true,
	linkcolor=red,
	citecolor=red,
}

% 设定页面的尺寸和比例
\geometry{left = 1.5cm, right = 1.5cm, top = 1.5cm, bottom = 1.5cm}

% 设定两栏之间的间距
\setlength\columnsep{1cm}

% 设定字体,为代码的插入作准备
\newfontfamily\ubuntu{Ubuntu Mono}
\newfontfamily\consolas{Consolas}

% 头部信息
\title{\normf{基于连续时间的LiDAR/Camera/IMU的时空标定方法}}
\author{\normf{陈烁龙}}
\date{\normf{\today}}

% 代码块的风格设定
\lstset{
	language=C++,
	basicstyle=\scriptsize\ubuntu,
	keywordstyle=\textbf,
	stringstyle=\itshape,
	commentstyle=\itshape,
	numberstyle=\scriptsize\ubuntu,
	showstringspaces=false,
	numbers=left,
	numbersep=8pt,
	tabsize=2,
	frame=single,
	framerule=1pt,
	columns=fullflexible,
	breaklines,
	frame=shadowbox, 
	backgroundcolor=\color[rgb]{0.97,0.97,0.97}
}

% 字体族的定义
% \fangsong \songti \heiti \kaishu
\newcommand\normf{\fangsong}
\newcommand\boldf{\heiti}
\newcommand\keywords[1]{\boldf{关键词:} \normf #1}

\newcommand\liehat[1]{\left[ #1 \right]_\times}
\newcommand\lievee[1]{\left[ #1 \right]^\vee}
\newcommand\liehatvee[1]{\left[ #1 \right]^\vee_\times}

\newcommand\mlcomment[1]{\iffalse #1 \fi}
%\newcommand\mlcomment[1]{ #1 }

\begin{document}
	
	% 插入头部信息
	\maketitle
	% 换页
	\thispagestyle{empty}
	\clearpage
	
	% 插入目录、图、表并换页
	\pagenumbering{roman}
	\tableofcontents
	\newpage
	\listoffigures
	\newpage
	\listoftables
	% 罗马字母形式的页码
	
	\clearpage
	% 从该页开始计数
	\setcounter{page}{1}
	% 阿拉伯数字形式的页码
	\pagenumbering{arabic}
	
	\section{\normf{交叉分配}}
	\normf
	在每个评审阶段,作品通常都是随机分发的,每份作品需要多位评委独立评审。为了增加不同评审专家所给成绩之间的可比性,不同专家评审的作品集合之间应有一些交集。但有的交集大了,则必然有交集小了,则可比性变弱。请针对3000支参赛队和125位评审专家,每份作品由5位专家评审的情况,建立数学模型确定最优的“交叉分发”方案,并讨论该方案的有关指标(自己定义)和实施细节。
	
	给定两个集合$\mathcal{A}$和$\mathcal{B}$,两个集合的交集度量指标为:
	\begin{equation*}
	n\left( \mathcal{A}\cap\mathcal{B}\right) =n\left(\mathcal{A} \right)+ n\left(\mathcal{B} \right)-n\left( \mathcal{A}\cup\mathcal{B}\right)
	\end{equation*}
	
	设专家集合为$\mathcal{J}$,且有$n(\mathcal{J})=125$,作品集合为$\mathcal{W}$,且有$n(\mathcal{W})=3000$。设第$i$个专家评审的作品集合为$\mathcal{J}_i(w)$,有$n(\mathcal{J}_i(w))\in[0,1,\cdots,3000)$,第$k$个作品的评审专家集合为$\mathcal{W}_k(j)$,有$n(\mathcal{W}_k(j))=5$。
	
	设$\mathcal{G}$为两两专家对的集合,对于$\mathcal{G}$内的每个元素(每对专家组合),根据两专家之间是否存在共同的评审作品,对其分类到两个集合:
	\begin{equation*}
	\mathcal{G}\mapsto\begin{cases}
	\mathcal{G}_1&n\left(\mathcal{J}_i(w)\cap\mathcal{J}_j(w) \right)=0\\
	\mathcal{G}_2&n\left(\mathcal{J}_i(w)\cap\mathcal{J}_j(w) \right)> 0\\
	\end{cases}
	\end{equation*}
	其中$\mathcal{G}_1$表示两评审专家不存在相同的评审作品,$\mathcal{G}_2$表示两评审专家存在相同的评审作品。基于此,易得:
	\begin{equation*}
	n(\mathcal{G}_1)=\sum_{i=0}^{n(\mathcal{J})-1}\sum_{j=i+1}^{n(\mathcal{J})-1}
	\delta(n\left(\mathcal{J}_i(w)\cap\mathcal{J}_j(w) \right)-0)
	\end{equation*}
	\begin{equation*}
	n(\mathcal{G}_2)=
	n(\mathcal{G})-n(\mathcal{G}_1)=
	C_{n(\mathcal{J})}^{2}-\sum_{i=0}^{n(\mathcal{J})-1}\sum_{j=i+1}^{n(\mathcal{J})-1}
	\delta(n\left(\mathcal{J}_i(w)\cap\mathcal{J}_j(w) \right)-0)
	\end{equation*}
	其中$\delta(x-x_0)$为狄拉克$\delta$函数:
	\begin{equation*}
	\delta(x-x_0)=\begin{cases}
	1&x=x_0\\0&x\ne x_0
	\end{cases}
	\end{equation*}
	
	
	\begin{figure}[t]
		\centering
		\includegraphics[width=0.8\linewidth]{./imgs/math_modeling.pdf}
		\caption{\normf 所构造指标的无向图示意图}
	\end{figure}
	该无向图中共有$n(\mathcal{J})$个节点,一个节点表示一个专家所评审的作品集合。若两个专家所评审的作品集合的交集非空,则两个对应节点用边连通。边的粗细(权重)与交集中元素个数成正比。
	
	为了增加不同评审专家所给成绩之间的可比性,不同专家评审的作品集合之间应有一些交集,但是若存在交集大的,则必然存在交集小的,就会造成可比性变弱。此在无向图中表现为边的粗细不均匀,且边数较少。因此最优的方案是使得该无向图在保证边的数目足够多(交集数目尽可能多)的情况下,同时边的粗细均匀(避免不同交集元素数目之间差异过大)。
	
	为此,我们在同时考虑边数目和边粗细的条件下,构建如下的评价指标:
	
	\begin{equation*}
	\rho=\frac{
		\begin{aligned}
		n(\mathcal{G}_2)
		\end{aligned}
	}{
		\begin{aligned}
		\left( 	n(\mathcal{G}_1)+n(\mathcal{G}_2)\right) \cdot(1+\sigma)
		\end{aligned}
	}
	\end{equation*}
	其中:
	\begin{equation*}
	\sigma^2=	\frac{
		\begin{aligned}
		\sum_{i=0}^{n(\mathcal{G}_2)-1}
		\left( 
		n\left(\mathcal{G}_2^i \right) -\bar{n}{(\mathcal{G}_2) }
		\right) ^2
		\end{aligned}
	}{
		\begin{aligned}
		n(\mathcal{G}_2)-1
		\end{aligned}
	}
	\qquad
	\bar{n}{(\mathcal{G}_2)}=\frac{1}{n(\mathcal{G}_2)}\cdot\sum_{i=0}^{n(\mathcal{G}_2)-1}n(\mathcal{G}_2^i)
	\end{equation*}
	解释:
	\begin{enumerate}
		\item 边数因子$f_1$:在无向图节点数目一定$n(\mathcal{J})$的情况下,当:
		\begin{equation*}
		f_1=\frac{n(\mathcal{G}_2)}{n(\mathcal{G}_1)+n(\mathcal{G}_2)}
		\end{equation*}
		越大,边数越多,即交集数目越多。
		
		\item 分布因子$f_2$:在无向图边数一定的情况下,当
		\begin{equation*}
		f_2=\frac{1}{1+\sigma}
		\end{equation*}
		越大,即$\sigma$越小时,表示边的粗细越均匀,也即不同交集元素数目之间差异较小。为了避免$f_1$过小但$\sigma$接近0导致评价指标异常的情况发生,在$\sigma$上补偿了一个单位量。
	\end{enumerate}
	
	在完全没有先验信息,如作品质量、专家评分偏好等未知的情况下,如下的随机分配策略是指标$\rho$下的最优分配策略:
		\begin{algorithm}[t]
		\caption{\normf{最优随机策略}}
		\LinesNumbered 
		\KwIn{\normf{作品集合$\mathcal{W}$,评审专家集合$\mathcal{J}$,每份作品需被评审的次数$n$。}}
		\KwOut{\normf{每位专家评审的作品集合,或者每个作品被评审的专家集合。}}
		\ForEach{$w\in\mathcal{W}$}{
			\normf{对当前作品随机选取的评审专家集合} $\mathcal{J}_s=\{\}$
			
			\For{\normf{$n$}}{
				\normf{随机选取一个专家$j\in\mathcal{J}$},满足$j\not\in\mathcal{J}_s$	
				
				$j\rightarrow\mathcal{J}_s$
			}
			\normf{包含$n$个专家的集合$\mathcal{J}_s$即为当前作品$w$随机分配到的评审专家集合。}
		}
	\end{algorithm}
\newpage
	为此,我们考虑了三种不同的典型分配策略:
	\begin{enumerate}
		\item 所提出的随机分配策略;
		\item 滑动窗口分配策略:
			\begin{figure}[h]
			\centering
			\includegraphics[width=0.8\linewidth]{./imgs/win_select.pdf}
			\caption{\normf 滑动窗口分配策略示意图}
		\end{figure}
		首先将评审专家进行有序排列成环,每次将作品分配给窗口内相邻的$5$个评审专家。在对下一个作品进行分配时,将窗口向右滑动一步,直至所有作品被分配完毕。
		\item 块分配策略:将专家分组,每5个专家一个组,共25组。再将作品分组,每120个作品作为一组,共25组。一个组的专家评审一个组的作品。
		\begin{figure}[h]
			\centering
			\includegraphics[width=0.8\linewidth]{./imgs/block_select.pdf}
			\caption{\normf 滑动窗口分配策略示意图}
		\end{figure}
	\end{enumerate}
	
\end{document}


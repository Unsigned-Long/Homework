\documentclass[12pt, twocolumn]{article}

% 引入相关的包
\usepackage{amsmath, listings, fontspec, geometry, graphicx, ctex, color, subfigure, amsfonts, amssymb}
\usepackage{multirow}
\usepackage[table,xcdraw]{xcolor}
\usepackage[ruled]{algorithm2e}
\usepackage[hidelinks]{hyperref}

		\usepackage{graphicx}
		\usepackage[most]{tcolorbox}
\hypersetup{
	colorlinks=true,
	linkcolor=red,
	citecolor=red,
}
\usepackage{booktabs}
\usepackage{multirow}
\usepackage{picins}

% 设定页面的尺寸和比例
\geometry{left = 1.5cm, right = 1.5cm, top = 1.5cm, bottom = 1.5cm}

% 设定两栏之间的间距
\setlength\columnsep{1cm}

% 设定字体,为代码的插入作准备
\newfontfamily\ubuntu{Ubuntu Mono}
\newfontfamily\consolas{Consolas}

% 头部信息
\title{\normf{编程:观测值逐次更新的扩展卡尔曼滤波器}}
\author{\normf 姓名:陈烁龙\;\;\;学号:2023202140019\;\;\;学院:测绘学院}
\date{\normf{\today}}

% 代码块的风格设定
\lstset{
	language=C++,
	basicstyle=\small\ubuntu,
	keywordstyle=\textbf,
	stringstyle=\itshape,
	commentstyle=\itshape,
	numberstyle=\scriptsize\ubuntu,
	showstringspaces=false,
	numbers=left,
	numbersep=8pt,
	tabsize=2,
	frame=single,
	framerule=1pt,
	columns=fullflexible,
	breaklines,
	frame=shadowbox, 
	backgroundcolor=\color[rgb]{0.97,0.97,0.97}
}

% 字体族的定义
% \fangsong \songti \heiti \kaishu
\newcommand\normf{\fangsong}
\newcommand\boldf{\heiti}
\newcommand\keywords[1]{\bfseries{关键词:} \normf #1}

\newcommand\liehat[1]{\left[ #1 \right]_\times}
\newcommand\lievee[1]{\left[ #1 \right]^\vee}
\newcommand\liehatvee[1]{\left[ #1 \right]^\vee_\times}

\newcommand\mlcomment[1]{\iffalse #1 \fi}
%\newcommand\mlcomment[1]{ #1 }

\newcounter{problemname}
\newenvironment{problem}{\stepcounter{problemname}\par\noindent\normf\textbf{\textcolor[rgb]{1,0,0}{题目\arabic{problemname}.} }}{\leavevmode\\\par}
\newenvironment{solution}{\par\noindent\normf\textbf{解答: }}{\leavevmode\\\par}
\newenvironment{note}{\par\noindent\normf\textbf{题目\arabic{problemname}的注记: }}{\leavevmode\\\par}


\begin{document}
	\begin{titlepage}
	    \centering
	    \includegraphics[width=0.4\textwidth]{whu_red.png}\par\vspace{1cm}
	    \vspace{4cm}
	    {\huge\kaishu\bfseries Marginalization\par in Continuous-time-based Estimators\par 基于连续时间的状态估计器中的边缘化 \par}
	    \vspace{3cm}
	    {\Large\kaishu 
	    \begin{center}\begin{tabular}{l}
	    姓名:陈烁龙\\
	    学号:\bfseries 2023202140019\\
	    学院:测绘学院
	    \end{tabular}\end{center}
	     \par}
	    
	
	    \vfill
	
	% Bottom of the page
	    {\large\kaishu\bfseries \today\par}
	\end{titlepage}
		% 换页
 		\thispagestyle{empty}
		\clearpage
		
		% 插入目录、图、表并换页
		\pagenumbering{roman}
		\tableofcontents
		\newpage
		\listoffigures
%		\newpage
%		\listoftables
		% 罗马字母形式的页码
		
		\clearpage
		% 从该页开始计数
		\setcounter{page}{1}
		% 阿拉伯数字形式的页码
		\pagenumbering{arabic}
	
	
	\begin{abstract}
		\normf\bfseries
		近几年来,同时定位与地图构建(Simultaneous Localization And Mapping, SLAM)技术已经较为成熟,且被广泛应用于诸如自动驾驶、机器人领域中,来提供连续的位置服务和精确、一致的先验地图。出于实时性的要求,SLAM算法需要在优化精度和优化效率之间平衡。目前比较主流的方法是采用滑窗(Sliding Window)优化加边缘化(Marginalization)的策略,能够在保证实时性的同时,尽可能地减少局部优化\footnote{\normf 由于滑窗优化本质上是一中局部优化而非全局批处理优化,得到的结果往往是全局次优的,而不是全局最优。}带来的精度损失。
		
		目前绝大多的SLAM算法在进行优化的时候都针对的是离散位姿状态序列,是一种离散时间(Discrete Time,DT)估计方法。而相对应的连续时间(Continuous Time,CT)估计方法则使用时间域上的连续函数(比如B样条函数)表达位姿轨迹,优化的是连续函数的参量。相比于DT估计方法,CT估计方法在考虑了运动连续性的同时,能够更有利于异步、高频多源传感器的融合。CT方法目前在SLAM技术上应用的不是很多,主要是因为其优化量较大,效率较低。
		
		本文聚焦于基于CT的估计器中的边缘化问题,进行CT估计器中边缘化算法的探究、实现和测试\footnote{\normf 由于本人的研究方向是连续时间相关的SLAM技术,所以在本次报告考察中,在不偏离课程主题的同时,对这一问题进行深入的探究。}。
	\end{abstract}
	
	\noindent\keywords{同时定位与地图构建,边缘化,连续时间估计方法}
	
	\section{\normf{概述}}
	\normf\bfseries
	
	\subsection{\normf 边缘化}
	同时定位与地图构建(Simultaneous Localization and Mapping,SLAM)指从未知环境的未知地点出发,在运动过程中通过重复观测到的地图特征定位自身位置和姿态,再根据自身位置增量式的构建地图,从而达到同时定位和地图构建的目的。在一个SLAM场景中,随着时间的推移,系统中的状态越来越多。如果在估计器中考虑所有的状态,会导致优化问题膨胀,进而降低估计器的优化效率,使得SLAM算法不能实时运行。
	
	在现阶段的SLAM技术中,一种简单有效的控制优化规模的方法,是仅保留离当前时刻最近的N个关键帧,去掉时间上更早的关键帧。也即,将优化规模固定在一个时间窗口内,离开这个窗口的则被丢弃。这种方法称为滑动窗口法\cite{高翔2017视觉}。
	
	\newpage
	\bibliographystyle{IEEEtran}
	\bibliography{reference}
		
	\newpage
	\section*{ACKNOWLEDGMENT}
	\begin{tcolorbox}[colback=white,colframe=white!70!black,title={\bfseries Author Information}]
	\par\noindent
		\parbox[t]{\linewidth}{
	 \noindent\parpic{\includegraphics[height=2in,width=1in,clip,keepaspectratio]{ShuolongChen_grey.jpg}}
	 \noindent{\bfseries Shuolong Chen}\emph{
	 received the B.S. degree in geodesy and geomatics engineering from Wuhan University, Wuhan China, in 2023.
	 He is currently a master candidate at the school of Geodesy and Geomatics, Wuhan University. His area of research currently focuses on integrated navigation systems and multi-sensor fusion.
	 Contact him via e-mail: shlchen@whu.edu.cn.
	 }}
	\end{tcolorbox}
		
		
\end{document}